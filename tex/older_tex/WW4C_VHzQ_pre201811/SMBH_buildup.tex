
The build-up of SMBHs via accretion is important due to the desire to
understand the evolution of the black hole mass function (BHMF) and
since there is a remarkable similarity between cosmic star-formation
rate density (SFRD) and black hole accretion rate (BHAR). The SFRD and
BHAR trace each other \citep[with a normalisation factor of
$\sim3000$,][]{Willott2013b, MadauDickinson2014} across redshifts
$z=0-4$.  However, new results \citep[e.g., ][]{Vito2018a, Calhau2018}
suggest that the SFRD and BHAR do {\it not} have the same evolutionary form
at higher redshift, $z>4$. The physical reasons for this are currently very poorly
understood.  For the high-$z$ QSOs that we can measure BHAR and SFR, most
have BHAR $\gg 10^{-3}$ SFR, indeed {\it Herschel}-detected sources at
$z\simeq.8$ have $L_{\rm SF} \simeq L_{\rm AGN}$ \citep[e.g.,
][]{Netzer2014} and {\it Herschel}-detected quasars show no evolution
in the $M_{\rm BH} - M_{\star}$ relation from $z\sim2$ to $z\sim0$.
where $M_{\rm BH}, M_{\rm bulge}, M_{\star}$ are the SMBH mass,
stellar mass of the bulge and galaxy total stellar mass,
respectively. The evolution and potential importance of the SFRD and BHAR 
link is a key outstanding problem in contemporary astrophysics, 
potentially explaining the galaxy-black hole scaling relations. 
However, there may also simply be a non-causal origin of the black-hole 
galaxy scaling relations \citep{Jahnke2011}.





\subsection{Consequences of evolution of the $M_{\rm BH} - M_{\star}$ relation}
From a study of 69 {\it Herschel}-detected broad-line active galactic
nuclei, \citet{Sun2015} find there is no evolution in the $M_{\rm BH}
- M_{\star}$ relation from $z\sim2$ to $z\sim0$, with the ratio of
$\log (M_{\rm BH} / M_{\star}$ constant at -2.85 across this redshift
range. If this ratio holds to $z>5$, and we assume the $M_{\rm BH}$
for the $z>6$ measured from the e.g. \mgii line, are generally
correct, then the galaxy total stellar mass for the $z>6$ objects
should be...

\citet{YangG2018} calculate the long-term SMBH accretion rate as a
function of $M_{\rm star}$ and redshift $[BHAR(M_{\star} ,z)]$ over
ranges of $\log (M_{\rm star} / M_{\odot}) = 9.5–12$ and $z = 0.4–4$.
Our BHAR(M⋆,z) is constrained by high-quality survey data
(GOODS-South, GOODS-North and COSMOS), and by the stellar mass
function and the X-ray luminosity function.  This BHAR/SFR dependence
on $M_{\star}$ does not support the scenario that SMBH and galaxy
growth are in lockstep.