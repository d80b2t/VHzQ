\lstset{upquote=true}

\noindent
Then the following SQL will return the values in
Table~\ref{tab:THE_TABLE}.

\begin{lstlisting}[
           language=SQL,
           showspaces=false,
           basicstyle=\ttfamily,
           numbers=left,
           numberstyle=\tiny,
           commentstyle=\color{gray}
        ]
SELECT 
qso.qsoName,  qso.raJ2000 as ra, qso.decJ2000 as dec, 
aver.apertureID,  aver.aperJky3 as aperJky3Aver, 
aver.aperJky3Err as aperJky3AverErr, aver.sumWeight, 
aver.ppErrBits as ppErrBitsAver, m.mjdObs, 
m.filterID, remeas.aperJky3, 
remeas.aperJky3Err, 
w.weight, remeas.ppErrBits, 
m.project

FROM 
finalQsoCatalogue as qso,  
MapApertureIDshighzQsoMap as ma,  
wserv1000MapRemeasAver as aver,  
wserv1000MapRemeasurement as remeas,  
MapProvenance as v,  
wserv1000MapAverageWeights as w, 
MapFrameStatus as mfs, 
Multiframe as m  

WHERE 
qso.qsoID=ma.objectID and 
ma.apertureID=aver.apertureID and 
aver.apertureID=remeas.apertureID and 
aver.catalogueID=v.combicatID and 
v.avSetupID=1 and 
v.catalogueID=remeas.catalogueID and 
w.combicatID=v.combicatID and 
w.catalogueID=v.catalogueID and 
w.apertureID=aver.apertureID and 
mfs.catalogueID=remeas.catalogueID and 
m.multiframeID=mfs.multiframeID and 
mfs.programmeID=10999 and 
mfs.mapID=1 
order by v.combicatID, m.mjdObs
\end{lstlisting}
