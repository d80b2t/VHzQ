%%%%%%%%%%%%%%%%%%%%%%%%%%%%%%%%%%%%%%%%%%%%%%%%%%%%%%%%%%%%%%%%%%%%%%%%%%%%%
%                                                                           
% Needs:  mn2e.cls, mn2e.bst, mn.sty, format.tex, psfig.sty
%									    
%%%%%%%%%%%%%%%%%%%%%%%%%%%%%%%%%%%%%%%%%%%%%%%%%%%%%%%%%%%%%%%%%%%%%%%%%%%%%
\documentclass[usenatbib]{mnras}

\usepackage{graphicx,fancyhdr,natbib,subfigure}
\usepackage{epsfig, epsf}
\usepackage{amsmath, cancel, amssymb}
\usepackage{lscape, longtable, caption}
\usepackage{multirow}
\usepackage{dcolumn}% Align table columns on decimal point
\usepackage{bm}% bold math
\usepackage{hyperref,ifthen}
\usepackage{verbatim}
\usepackage{color}
\usepackage[usenames,dvipsnames]{xcolor}
%% http://en.wikibooks.org/wiki/LaTeX/Colors



%%%%%%%%%%%%%%%%%%%%%%%%%%%%%%%%%%%%%%%%%%%
%       define Journal abbreviations      %
%%%%%%%%%%%%%%%%%%%%%%%%%%%%%%%%%%%%%%%%%%%
\def\nat{Nat} \def\apjl{ApJ~Lett.} \def\apj{ApJ}
\def\apjs{ApJS} \def\aj{AJ} \def\mnras{MNRAS}
\def\prd{Phys.~Rev.~D} \def\prl{Phys.~Rev.~Lett.}
\def\plb{Phys.~Lett.~B} \def\jhep{JHEP} \def\nar{NewAR}
\def\npbps{NUC.~Phys.~B~Proc.~Suppl.} \def\prep{Phys.~Rep.}
\def\pasp{PASP} \def\aap{Astron.~\&~Astrophys.} \def\araa{ARA\&A}
\def\pasa{PASA}
\def\jcap{\ref@jnl{J. Cosmology Astropart. Phys.}}%
\def\physrep{Phys.~Rep.}


\newcommand{\preep}[1]{{\tt #1} }

%%%%%%%%%%%%%%%%%%%%%%%%%%%%%%%%%%%%%%%%%%%%%%%%%%%%%
%              define symbols                       %
%%%%%%%%%%%%%%%%%%%%%%%%%%%%%%%%%%%%%%%%%%%%%%%%%%%%%
\def \Mpc {~{\rm Mpc} }
\def \Om {\Omega_0}
\def \Omb {\Omega_{\rm b}}
\def \Omcdm {\Omega_{\rm CDM}}
\def \Omlam {\Omega_{\Lambda}}
\def \Omm {\Omega_{\rm m}}
\def \ho {H_0}
\def \qo {q_0}
\def \lo {\lambda_0}
\def \kms {{\rm ~km~s}^{-1}}
\def \kmsmpc {{\rm ~km~s}^{-1}~{\rm Mpc}^{-1}}
\def \hmpc{~\;h^{-1}~{\rm Mpc}} 
\def \hkpc{\;h^{-1}{\rm kpc}} 
\def \hmpcb{h^{-1}{\rm Mpc}}
\def \dif {{\rm d}}
\def \mlim {m_{\rm l}}
\def \bj {b_{\rm J}}
\def \mb {M_{\rm b_{\rm J}}}
\def \mg {M_{\rm g}}
\def \qso {_{\rm QSO}}
\def \lrg {_{\rm LRG}}
\def \gal {_{\rm gal}}
\def \xibar {\bar{\xi}}
\def \xis{\xi(s)}
\def \xisp{\xi(\sigma, \pi)}
\def \Xisig{\Xi(\sigma)}
\def \xir{\xi(r)}
\def \max {_{\rm max}}
\def \gsim { \lower .75ex \hbox{$\sim$} \llap{\raise .27ex \hbox{$>$}} }
\def \lsim { \lower .75ex \hbox{$\sim$} \llap{\raise .27ex \hbox{$<$}} }
\def \deg {^{\circ}}
%\def \sqdeg {\rm deg^{-2}}
\def \deltac {\delta_{\rm c}}
\def \mmin {M_{\rm min}}
\def \mbh  {M_{\rm BH}}
\def \mdh  {M_{\rm DH}}
\def \msun {M_{\odot}}
\def \z {_{\rm z}}
\def \edd {_{\rm Edd}}
\def \lin {_{\rm lin}}
\def \nonlin {_{\rm non-lin}}
\def \wrms {\langle w_{\rm z}^2\rangle^{1/2}}
\def \dc {\delta_{\rm c}}
\def \wp {w_{p}(\sigma)}
\def \PwrSp {\mathcal{P}(k)}
\def \DelSq {$\Delta^{2}(k)$}
\def \WMAP {{\it WMAP \,}}
\def \cobe {{\it COBE }}
\def \COBE {{\it COBE \;}}
\def \HST  {{\it HST \,\,}}
\def \Spitzer  {{\it Spitzer \,}}
\def \ATLAS {VST-AA$\Omega$ {\it ATLAS} }
\def \BEST   {{\tt best} }
\def \TARGET {{\tt target} }
\def \TQSO   {{\tt TARGET\_QSO}}
\def \HIZ    {{\tt TARGET\_HIZ}}
\def \FIRST  {{\tt TARGET\_FIRST}}
\def \zc {z_{\rm c}}
\def \zcz {z_{\rm c,0}}

\newcommand{\ltsim}{\raisebox{-0.6ex}{$\,\stackrel
        {\raisebox{-.2ex}{$\textstyle <$}}{\sim}\,$}}
\newcommand{\gtsim}{\raisebox{-0.6ex}{$\,\stackrel
        {\raisebox{-.2ex}{$\textstyle >$}}{\sim}\,$}}
\newcommand{\simlt}{\raisebox{-0.6ex}{$\,\stackrel
        {\raisebox{-.2ex}{$\textstyle <$}}{\sim}\,$}}
\newcommand{\simgt}{\raisebox{-0.6ex}{$\,\stackrel
        {\raisebox{-.2ex}{$\textstyle >$}}{\sim}\,$}}

\newcommand{\Msun}{M_\odot}
\newcommand{\Lsun}{L_\odot}
\newcommand{\lsun}{L_\odot}
\newcommand{\Mdot}{\dot M}

\newcommand{\sqdeg}{deg$^{-2}$}
\newcommand{\hi}{H\,{\sc i}\ }
\newcommand{\lya}{Ly$\alpha$\ }
%\newcommand{\lya}{Ly\,$\alpha$\ }
\newcommand{\lyaf}{Ly\,$\alpha$\ forest}
%\newcommand{\eg}{e.g.~}
%\newcommand{\etal}{et~al.~}
\newcommand{\lyb}{Ly$\beta$\ }
\newcommand{\cii}{C\,{\sc ii}\ }
\newcommand{\ciii}{C\,{\sc iii}]\ }
\newcommand{\civ}{C\,{\sc iv}\ }
\newcommand{\SiII}{Si\,{\sc ii}\ }
\newcommand{\SiIV}{Si\,{\sc iv}\ }
\newcommand{\mgii}{Mg\,{\sc ii}\ }
\newcommand{\feii}{Fe\,{\sc ii}\ }
\newcommand{\feiii}{Fe\,{\sc iii}\ }
\newcommand{\caii}{Ca\,{\sc ii}\ }
\newcommand{\halpha}{H\,$\alpha$\ }
\newcommand{\hbeta}{H\,$\beta$\ }
\newcommand{\hgamma}{H\,$\gamma$\ }
\newcommand{\hdelta}{H\,$\delta$\ }
\newcommand{\oi}{[O\,{\sc i}]\ }
\newcommand{\oii}{[O\,{\sc ii}]\ }
\newcommand{\oiii}{[O\,{\sc iii}]\ }
\newcommand{\heii}{He\,{\sc ii}\ }
%\newcommand{\heii}{[He\,{\sc ii}]\ }
\newcommand{\nv}{N\,{\sc v}\ }
\newcommand{\nev}{Ne\,{\sc v}\ }
\newcommand{\neiii}{[Ne\,{\sc iii}]\ }
\newcommand{\alii}{Al\,{\sc ii}\ }
\newcommand{\aliii}{Al\,{\sc iii}\ }
\newcommand{\siiii}{Si\,{\sc iii}]\ }


\begin{document}

\title[WW4C I: Very high-$z$ Quasars]
      {The WISE $W4$ Compendium: I. Introducing the WW4C with $z>4$ Quasars}
\author[N.P. Ross et al.]
       {Nicholas P. Ross$^{1}$\thanks{email: npross@roe.ac.uk}\\ 
$^1$Institute for Astronomy, University of Edinburgh, Royal Observatory, Edinburgh, EH9 3HJ, United Kingdom\\
}

\maketitle
\begin{abstract}
Lorem ipsum dolor sit amet, consectetur adipiscing elit. Aliquam porta
sodales est, vel cursus risus porta non. Vivamus vel pretium
velit. Sed fringilla suscipit felis, nec iaculis lacus convallis
ac. Fusce pellentesque condimentum dolor, quis vehicula tortor
hendrerit sed. Class aptent taciti sociosqu ad litora torquent per
conubia nostra, per inceptos himenaeos. Etiam interdum tristique diam
eu blandit. Donec in lacinia libero.
%%
Sed elit massa, eleifend non sodales a, commodo ut felis. Sed id
pretium felis. Vestibulum et turpis vitae quam aliquam convallis. Sed
id ligula eu nulla ultrices tempus. Phasellus mattis erat quis metus
dignissim malesuada. Nulla tincidunt quam volutpat nibh facilisis
euismod. Cras vel auctor neque. Nam quis diam risus.
\end{abstract}


\begin{keywords}
Astronomical data bases: surveys -- 
Quasars: general -- 
galaxies: evolution -- 
galaxies: infrared.
\end{keywords}



%%%%%%%%%%%%%%%%%%%%%%%%%%%%%%%%%%%%%%%%%%%%%%%%%%%%%%%%%%%%%%%%%%
%%%%%%%%%%%%%%%%%%%%%%%%%%%%%%%%%%%%%%%%%%%%%%%%%%%%%%%%%%%%%%%%%%
%%
%%  S E C T I O  N   1         S E C T I O  N   1           S E C T I O  N   1       S E C T I O  N   1
%%  S E C T I O  N   1         S E C T I O  N   1           S E C T I O  N   1       S E C T I O  N   1
%%  S E C T I O  N   1         S E C T I O  N   1           S E C T I O  N   1       S E C T I O  N   1
%%
%%%%%%%%%%%%%%%%%%%%%%%%%%%%%%%%%%%%%%%%%%%%%%%%%%%%%%%%%%%%%%%%%%
%%%%%%%%%%%%%%%%%%%%%%%%%%%%%%%%%%%%%%%%%%%%%%%%%%%%%%%%%%%%%%%%%%
\section{Introduction}
Very high redshift quasars (VH$z$Qs; $z\gtrsim5$) are excellent probes
of the early Universe. This includes studies of the Epoch of
Reionization for hydrogen \citep[see e.g.][for reviews]{Fan2006review,
Mortlock2016}, the formation and build-up of supermassive black holes
\citep[e.g., ][]{Rees1984, WyitheLoeb2003, Volonteri2010, Agarwal2016,
Valiante2018, Latif2018} and early metal enrichment \citep[see e.g.,
][]{Simcoe2012, Chen2017, Bosman2017}.

There is a remarkable similarity between cosmic star-formation rate
density (SFRD) and black hole accretion rate (BHAR) trace each other
\citep[with a normalisation factor of $\sim3000$,][]{Willott2013b,
MadauDickinson2014} across redshifts $z=0-4$.  However, new results
\citet[e.g., ][]{Vito2018a, Calhau2018} suggest that the SFRD and BHAR
do not have the same evolutionary form at higher, $z>4$ redshifts, for
reasons that are currently poorly understood.
%% For some high-$z$ QSOs we can measure BHAR and SFR: most have BHAR $\gg 10^{-3}$ SFR.

Quasars are also known to be prodigious emitters of infrared emission,
thought to be from the thermal emission of dust grains heated by
continuum emission from the accretion disc \citep[][]{Hill2014,
Hickox2017}.  Observations in the mid-infrared, e.g. $\sim$3-30$\mu$m
allow discrimation between AGN\footnote{Historically, ``quasars'' and
``Active Galactic Nuclei (AGN)'' have described different
luminosity/classes of objects, but here we use these terms
interchangeably (with a preference for quasar) in recognition of the
fact that they both describe accreting supermassive black holes
\citep[e.g.][]{Haardt2016book}.}  and passive galaxies due to the
1.6$\mu$m ``bump'' entering the MIR at $z\approx0.8-0.9$ \citep[e.g.,
][]{Wright1994, Sawicki2002, Lacy2004, Stern2005, Richards2006b, Timlin2016} as
well as AGN and star-forming galaxies due to the presence of
Polycyclic Aromatic Hydrocarbon (PAHs) at $\lambda >3\mu$m
\citep[e.g., ][]{Yan2007, Tielens2008}.

\citet{Jiang2006dust} and \citet{Jiang2010} report on the discovery of
a quasar without hot-dust emission in a sample of 21 $z\approx6$
quasars. Such apparently hot-dust-free quasars have no counterparts at
low redshift. Moreover, we demonstrate that the hot-dust abundance in
the 21 quasars builds up in tandem with the growth of the central
black hole,

However, \citet{Blain2013} was compiled with the WISE `All-Sky' 
data release, as opposed to the superior ``AllWISE'' catalogs and 
only examined the 31 known $z>6$ quasars. Our sample has 
148 objects with redshift $z\geq6.00$. 

Here we update \citet{Jiang2010}  and \citet{Blain2013} 
\citep[along with Table 8 of][]{Banados2016}. 

We chose redshift $z=5.0$ as our lower redshift limit due to a combination 
of garnishing a large sample, adequately spanning physical properties 
(e.g. luminosity, age of the Universe) and to incorporate what knowledge 
we've gained over the last couple of decades since $z>5$ quasars were 
discovered. 

WISE mapped the sky in 4 passbands, in bands centered at wavelengths
of 3.4, 4.6, 12, and 23$\mu$m.  In total the release all sky
``ALLWISE'' catalog, contains nearly 750 million detections at
high-significance\footnote{\href{http://wise2.ipac.caltech.edu/docs/release/allwise/expsup/sec2\_1.html}{wise2.ipac.caltech.edu/docs/release/allwise/expsup/sec2\_1.html}}.
\citet{Assef2013}, \citet{Stern2012}

\citet{Brown2014b}, in PASA, is the paper about Recalibrating the Wide-field Infrared Survey Explorer (WISE) W4 Filter,

In this paper, we introduce the ``WISE W4 Compendium'' (WW4C); a
detailed study into the objects that were detected in the longest
waveband, 20-28$\mu$m observed, on the Wide-Field Infrared Survey
Explorer \citep[WISE;][]{Wright2010, Cutri2013} mission.
%%
This study will describe all 40 million objects that are detected in
the WISE W4-band, but will concentrate on those objects most affected
by radiating dust emission and well described by extragalactic, and
AGN, spectral energy distributions (SEDs).  The motivations of the
WW4C are numerous, but the primary science we will pursue is the
identification of bolometric luminous AGN, especially those that might
not be observered in X-ray/UV/optical surveys.
%%
However, we will aim to remain as agnostic as possible to the origin
of the objects that are emit in the mid-infrared.

This paper can be considered an update of \citet{Blain2013} and also
an extension of \citet{Banados2014}, with the latter study reporting
WISE W1, W2, W3 and W4 magnitudes for Panoramic Survey Telescope and Rapid Response System 1 
\citep[Pan-STARRS1, PS1;][]{Kaiser2002, Kaiser2010}, but with no  
further investigation into the reddest WISE waveband for the VH$z$Qs.
%%
\citet{Banados2016} reports and investigates the W1, W2 and W3 properties
of quasars at $z > 5.6$.

$M_{\rm BH}, M_{\rm bulge}, M_{\star}$ are the SMBH mass, stellar mass of the bulge and 
 galaxy total stellar mass, respectively. 

Because of established conventions, we report SDSS $ugriz$ magnitudes on
the AB zero-point system \citep{Oke_Gunn1983, Fukugita1996}, 
while the WISE W1 − 4 magnitudes are calibrated on the Vega system
(Wright et al. 2010). For WISE bands, $m_{\rm AB} = m_{\rm Vega} + m$ where m =
(2.699, 3.339, 5.174, 6.66) for W1, W2, W3 and W4, respectively
\citep{Cutri2011, Brown2014b}. We make use of the Explanatory Supplement to the
WISE All-Sky Data Release, as well as the WISE AllWISE Data Release
Products online.
All optical ($ugriz$ bands) magnitudes are given in the AB system, all near-infrared 
($y/YJHK$) are also given in the AB system. 
Our chosen cosmology is: 
%%
We use a flat $\Lambda$CDM cosmology with $H0 = 67.7$ km s$-1$ Mpc$−1$, 
$\Omega_{\rm M} = 0.307$, and $\Omega_{\Lambda} = 0.693$ (Planck Collaboration et al. 2016).
in order to be consistent with \citet{Banados2016}. 


%%%%%%%%%%%%%%%%%%%%%%%%%%%%%%%%%%%%%%%%%%%%%%%%%%%%%%%%%%%%%%%%%%
%%%%%%%%%%%%%%%%%%%%%%%%%%%%%%%%%%%%%%%%%%%%%%%%%%%%%%%%%%%%%%%%%%
%%
%%  S E C T I O  N   2         S E C T I O  N   2           S E C T I O  N   2       S E C T I O  N   2
%%  S E C T I O  N   2         S E C T I O  N   2           S E C T I O  N   2       S E C T I O  N   2
%%  S E C T I O  N   2         S E C T I O  N   2           S E C T I O  N   2       S E C T I O  N   2
%%
%%%%%%%%%%%%%%%%%%%%%%%%%%%%%%%%%%%%%%%%%%%%%%%%%%%%%%%%%%%%%%%%%%
%%%%%%%%%%%%%%%%%%%%%%%%%%%%%%%%%%%%%%%%%%%%%%%%%%%%%%%%%%%%%%%%%%
\section{Data}

\pagestyle{empty}
\begin{landscape}
%\onecolumn
 % \begin{landscape}
%\small  %\footnotesize \scriptsize \tiny
\begin{table}
\begin{center}
\begin{tabular}
%{|l|l|l|l|r|r|r|r|r|r|r|r|r|r|r|r|r|r|r|r|r|r|r|r|l|}
%{ l l l l r r r r r r r r r r r r r r r r r r r r l }
{ l l l l l l l l l l l l l l l l l l l l l l l l l }
\hline \hline
  \multicolumn{1}{ c }{na} &
  \multicolumn{1}{c }{desig} &
  \multicolumn{1}{c }{ra\_hms} &
  \multicolumn{1}{c }{dec\_dms} &
  \multicolumn{1}{c }{ra} &
  \multicolumn{1}{c }{dec} &
  \multicolumn{1}{c }{redshift} &
  \multicolumn{1}{c }{mag} &
  \multicolumn{1}{c }{M1450} &
  \multicolumn{1}{c }{errM1450} &
  \multicolumn{1}{c }{ra\_WISE} &
  \multicolumn{1}{c }{dec\_WISE} &
  \multicolumn{1}{c }{w1mag} &
  \multicolumn{1}{c }{w1err} &
  \multicolumn{1}{c }{w1snr} &
  \multicolumn{1}{c }{w2mag} &
  \multicolumn{1}{c }{w2err} &
  \multicolumn{1}{c }{w2snr} &
  \multicolumn{1}{c }{w3mag} &
  \multicolumn{1}{c }{w3err} &
  \multicolumn{1}{c }{w3snr} &
  \multicolumn{1}{c }{w4mag} &
  \multicolumn{1}{c }{w4err} &
  \multicolumn{1}{c }{w4snr} &
  \multicolumn{1}{c }{ref} \\
\hline
  PSO & J000.3401+26.8358 & 00:01:21.63 & +26:50:09.17 & 0.34011348 & 26.83588138 & 5.75 & 19.52 & -27.16       & 9.99 & 0.3400887 & 26.835823        &     16.373 & 0.066 & 16.5 & 15.266 & 0.107 & 10.2 & 12.594 & 0.492 & 2.2 & 8.756 & -9.99 & 1.1 & 1/1/1\\
  SDSS & J0002+2550 & 00:02:39.39 & +25:50:34.80 & 0.66411726 & 25.84304425 & 5.82 & 19.39 & -27.31                 & 9.99 & 0.6641312 & 25.8430852      & 16.162 & 0.057 & 19.0 & 15.542 & 0.127 & 8.5 & 12.416 & 0.423 & 2.6 & 8.683 & -9.99 & 1.2 & 5/22/1\\
  SDSS & J0005-0006 & 00:05:52.34 & -00:06:55.80 & 1.4680833 & -0.1154999 & 5.85 & 20.98 & -25.73                       & 9.99 & 1.4683933 & -0.1154292     & 17.299 & 0.16 & 6.8 & 17.043 & -9.99 & 0.2 & 12.445 & -9.99 & -1.1 & 9.008 & -9.99 & -0.3 & 5/12/1\\
  PSO & J002.1073-06.4345 & 00:08:25.77 & -06:26:04.60 & 2.10739 & -6.43456 & 5.93 & 20.41 & -26.32                    & 9.99 & 2.1073696 & -6.4345725      & 16.809 & 0.107 & 10.1 & 15.684 & 0.141 & 7.7 & 11.892 & -9.99 & 1.5 & 8.759 & -9.99 & 0.2 & 1;43/1/1\\
  SDWISE & J0008+3616 & 00:08:51.43 & +36:16:13.49 & 2.2142917 & 36.2704138 & 5.17 & 19.12 & -27.34                 & 9.99 & 2.2142355 & 36.2704366       & 16.045 & 0.052 & 20.7 & 15.373 & 0.092 & 11.8 & 12.043 & -9.99 & 1.8 & 8.786 & -9.99 & 1.1 & Wang2016\\
  PSO & J002.3786+32.8702 & 00:09:30.89 & +32:52:12.94 & 2.37870183 & 32.87026179 & 6.1 & 21.13 & -25.65        & 9.99 & 2.3787018 & 32.8702618       & -99.99 & -9.99 & -9.9 & -99.99 & -9.99 & -9.9 & -99.99 & -9.99 & -9.9 & -9.99 & -9.99 & -9.9 & 1/1/1\\
  SDSS & J0017-1000 & 00:17:14.68 & -10:00:55.4 & 4.3111666 & -10.01539722 & 5.011 & 99.99 & -99.99                 & 9.99 & 4.3111476 & -10.0154269      & 15.936 & 0.055 & 19.7 & 15.167 & 0.094 & 11.5 & 12.026 & 0.334 & 3.2 & 8.52 & -9.99 & 1.2 & DR7\_W16\\
  PSO & J004.3936+17.0862 & 00:17:34.47 & +17:05:10.70 & 4.39361347 & 17.08630447 & 5.8 & 20.69 & -26.01        & 9.99 & 4.3936134 & 17.0863047        & -99.99 & -9.99 & -9.9 & -99.99 & -9.99 & -9.9 & -99.99 & -9.99 & -9.9 & -9.99 & -9.99 & -9.9 & 1/1/1\\
  PSO & J004.8140-24.2991 & 00:19:15.38 & -24:17:56.98 & 4.81408 & -24.29916 & 5.68 & 19.43 & -27.24                 & 9.99 & 4.814116 & -24.2990993        &   16.281 & 0.069 & 15.8 & 15.569 & 0.116 & 9.4 & 12.123 & 0.344 & 3.2 & 8.82 & -9.99 & 0.5 & 1/1/1\\
  VDES & J0020-3653 & 00:20:31.46 & -36:53:41.8 & 5.1311237 & -36.8949476 & 6.9 & 99.99 & -99.99                        & 9.99 & 5.1311237 & -36.8949476     & 16.844 & 0.094 & 11.6 & 16.354 & 0.204 & 5.3 & 12.679 & -9.99 & -0.1 & 8.342 & -9.99 & 0.8 & DES-VHS\_inprep\\
\hline \hline
\end{tabular}
\caption{All 425 $z\geq5.00$ quasars that have been spectroscopically confirmed as of 2018 June. 
  The first ten objects are given here as guidance to the format of the data table. The full table  can be found online.} 
\label{tab:THE_TABLE}
  \end{center}
\end{table}
\normalsize 
  \end{landscape}
\twocolumn
\pagestyle{plain}


Table~\ref{tab:THE_TABLE} gives the salient details for the objects
used in this study. We use all the $z\geq5.00$ quasars that
have been discovered and spectroscopically confirmed as of the time of
writing (2018 June). We report near-infrared ($yYJHK$-bands)
and mid-infrared (WISE W1/2/3/4) photometry.

\subsection{Very high redshift quasars}
In Table~\ref{tab:THE_TABLE} we give the discovery reference for the
VH$z$Qs noting that some objects were discovered independently and
contemporaneously.  The redshifts for the VH$z$Qs generally come from
the measurement of broad UV/optical emission lines. Where 
there are far infra-red emission lines e.g. \cii~158 micron, we report 
these, but at the level of our current analysis broadline redshifts are
sufficient. 

\citet{Mortlock2011}, \citet{McGreer2013}, \citet{Venemans2013},  \citet{Venemans2013}, \citet{Venemans2015a},  \citet{Venemans2015b}, \citet{Banados2016}, \citet{Matsuoka2016}, \citet{Reed2017}, \citet{Wang2017}, \citet{Mazzucchelli2017}, \citet{Ikeda2017}, \citet{Matsuoka2017}, \citet{Tang2017}, \citet{Koptelova2017}, \citet{Matsuoka2017}, \citet{Banados2018}

%% From Banado2016
%% Fan et al. (2000) [2000AJ....120.1167F];     3 = Fan et al. (2001) [2001AJ....122.2833F];     4 = Fan et al. (2003) [2003AJ....125.1649F];     5 = Fan et al. (2004) [2004AJ....128..515F];     6 = Mahabal et al. (2005) [2005ApJ...634L...9M];     7 = Cool et al. (2006) [2006AJ....132..823C];     8 = Fan et al. (2006a) [2006AJ....131.1203F];     9 = Goto et al. (2006) [2006MNRAS.371..769G];    10 = Mcgreer et al. (2006) [2006ApJ...652..157M];    11 = Carilli et al. (2007) [2007ApJ...666L...9C];   12 = Kurk et al. (2007) [2007ApJ...669...32K];    13 = Stern et al. (2007) [2007ApJ...663..677S];    14 = Venemans et al. (2007a) [2007MNRAS.376...76V];    15 = Willott et al. (2007) [2007AJ....134.2435W];    16 = Jiang et al. (2008) [2008AJ....135.1057J];    17 = Wang et al. (2008b) [2008ApJ...687..848W];    18 = Jiang et al. (2009) [2009AJ....138..305J];    19 = Kurk et al. (2009) [2009ApJ...702..833K];    20 = Mortlock et al. (2009) [2009A&A...505...97M];    21 = Willott et al. (2009) [2009AJ....137.3541W];    22 = Carilli et al. (2010) [2010ApJ...714..834C];   23 = Wang et al. (2010) [2010ApJ...714..699W];    24 = Willott et al. (2010a) [2010AJ....140..546W];    25 = Willott et al. (2010b) [2010AJ....139..906W];    26 = Derosa et al. (2011) [2011ApJ...739...56D];    27 = Mortlock et al. (2011) [2011Natur..474..616M];   28 = Wang et al. (2011) [2011ApJ...739L..34W];   29 = Zeimann et al. (2011) [2011ApJ...736...57Z];    30 = Morganson et al. (2012) [2012AJ....143..142M];    31 = Venemans et al. (2012) [2012ApJ...751L..25V];   32 = Mcgreer et al. (2013) [2013ApJ...768..105M];    33 = Venemans et al. (2013) [2013ApJ...779...24V];   34 = Wang et al. (2013) [2013ApJ...773...44W];   35 = Willott et al. (2013b) [2013ApJ...770...13W];    36 = Banados et al. (2014) [2014AJ....148...14B];    37 = Calura et al. (2014) [2014MNRAS.438.2765C];    38 = Leipski et al. (2014) [2014ApJ...785..154L];   39 = Banados et al. (2015a) [2015ApJ...805L...8B];   40 = Banados et al. (2015b) [2015ApJ...804..118B];   41 = Becker et al. (2015) [2015PASA...32...45B];    42 = Carnall et al. (2015) [2015MNRAS.451...16C];    43 = Jiang et al. (2015) [2015AJ....149..188J];    44 = Kashikawa et al. (2015) [2015ApJ...798...28K];   45 = Kim et al. (2015) [2015ApJ...813L..35K];   46 = Reed et al. (2015) [2015MNRAS.454.3952R];    47 = Venemans et al. (2015a) [2015MNRAS.453.2259V];    48 = Venemans et al. (2015b) [2015ApJ...801L..11V];   49 = Willott et al. (2015) [2015ApJ...801..123W];   50 = Wu et al. (2015) [2015Natur..518..512W];    51 = Venemans et al. (2016) [2016ApJ...816...37V];   52 = Wang-Feige et al. (2016) [labaApJ...819...24W];   53 = Matsuoka et al. (2016) [2016arXiv160302281M];   54 = Wang et al. (2016) [2016arXiv160609634W];
\citet{Fan2000}, \citet{Fan2001c}, \citet{Fan2003} , \citet{Fan2004}, \citet{Mahabal2005}, \citet{Cool2006}, \citet{Fan2006}, \citet{Goto2006}, \citet{McGreer2006}, \citet{Carilli2007}, \citet{Kurk2007}, \citet{Stern2007}, \citet{Venemans2007}, \citet{Willott2007}, \citet{Jiang2008}, \citet{Wang2008}         \citet{Jiang2009}         \citet{Kurk2009}          \citet{Mortlock2009}      \citet{Willott2009}       \citet{Carilli2010}       \citet{Wang2010}          \citet{Willott2010a}      \citet{Willott2010b}      \citet{DeRosa2011}        \citet{Mortlock2011}     \citet{Wang2011}          \citet{Zeimann2011}       \citet{Morganson2012}    \citet{Venemans2012}      \citet{McGreer2013}       \citet{Venemans2013}      \citet{Wang2013}         \citet{Willott2013b}     \citet{Banados2014}       \citet{Calura2014}        \citet{Leipski2014}       \citet{Banados2015a}      \citet{Banados2015b}      \citet{Becker2015}        \citet{Carnall2015}       \citet{Jiang2015}        \citet{Kashikawa2015}    \citet{Kim2015}          \citet{Reed2015}          \citet{Venemans2015a}    \citet{Venemans2015b}    \citet{Willott2015}      \citet{Wu2015}           \citet{Venemans2016}      \citet{Wang2016_WISE}    \citet{Matsuoka2016}      
\citet{WangR2016}          


%\newpage
\subsection{Near-infrared photometry}
Due to their selection of being very faint/undetected in the
observed-frame optical but bright in the observed frame near-infrared,
VH$z$Qs are generally detected in the near-infrared $yYJHK$-bands
\citep[$\approx$0.98-2.38$\mu$m; e.g., ][]{Peth2011}.

We query the WFCAM Science Archive
\citep[\href{http://wsa.roe.ac.uk/}{WSA}; ][]{Hambly2008} which
reports NIR photometry from the Wide Field Camera \citep[WFCAM;
][]{Casali2007} on the United Kingdom Infrared Telescope (UKIRT).
Data from the VIRCAM (VISTA InfraRed CAMera) on the VISTA
\citep[Visible and Infrared Survey Telescope for Astronomy;
][]{Emerson2006, Dalton2006} is also given in the WSA.

Quasars are known to vary (both photometrically and spectroscopically)
and very high-$z$ quasars under going super-Eddington accretion during
a rapid BH growth phase are prime candidates for this variation.
Therefore, with repeat observations over many epochs availble via the
WSA, we have a choice to make for how we report the photometry. After
tests, we decide to give the NIR photometry averaged on 4 week
observed timescales. This time bracket is chosen as a good compromise
between maximising signal-to-noise while keeping the cadence high. At
our observed redshifts, we sample $\lesssim1$week in the rest-frame
and luminous quasars are not expected to vary on timescales quicker
than this \citep[e.g., ][]{Lawrence2016_ASPC}. 
 
We give our recipe and SQL query syntax in Appendix~\ref{sec:SQL}.


\subsection{The WISE W4 Catalogue}
There are 747,634,026 total entries in the AllWISE Source Catalog.  Of
these, 40,939,966 (5.476\%) are W4-detected.

\begin{table}
  \begin{center}
    \begin{tabular}{l rr}
      \hline
      \hline
      Description & \#  objects & \% of Total \\         
      \hline  
      AllWISE Source Catalog & 747,634,026  & 100.0000 \\
      All W4 detections          &  40,939,966  &      5.4759 \\
      W4 Only                        &         35,818  &      0.0048\\
     \hline
      \hline
    \end{tabular}
    \caption{}
    \label{tab:ERQ_key_numbers}
  \end{center}
  \vspace{-8pt}
\end{table}


%    \begin{figure}
%      \includegraphics[height=8.0cm,width=8.0cm]
%      {pdf/SDSS_Quasar_Nofz.pdf}
%      \centering
%      \caption[he selection of $z \sim 0.7$ LRGs using the SDSS $riz$-bands]
%              {The selection of $z \sim 0.7$ LRGs using the SDSS $riz$-bands.}
%      \label{fig:fig1}
%    \end{figure}


%    \begin{table}
%    \begin{center}
%    \setlength{\tabcolsep}{4pt}
%    \begin{tabular}{lrrr}
%    Sample Description  & Number in sample & North & South   \\
%    \hline
%    \label{tab:The_LRG_numbers}
%    \end{tabular}
%    \caption{}
%    \end{center}
%    \end{table}

%    \begin{equation}
%      \label{equ:simple_prob}
%      dP = n \, dV.
%    \end{equation}
    
%    \begin{eqnarray}
%      \xi_{LS}(s) &=& 1 + \left(\frac{N_{rd}}{N} \right)^{2} \frac{DD(s)}{RR(s)} -
%      2   \left( \frac{N_{rd}}{N} \right) \frac{DR(s)}{RR(s)} \\
%      &\equiv&  \frac{DD(s)-2DR(s)+RR(s)}{RR(s)},
%      \label{lseq}
%    \end{eqnarray}

%% http://www-astro.physics.ox.ac.uk/~kmb/latex_colour.html
%{\color{red} ...bit of LaTeX text...}
%an example of making an equation the colour DarkSeaGreen (rather a sophisticated shade, I think) one types the following:
%\begin{equation}
%{\color{DarkSeaGreen} x = \log_{10} (\nu/\rm MHz) }
%\end{equation}

%\subsection{Dust Overview}
%http://arxiv.org/pdf/1605.06671.pdf


%%%%%%%%%%%%%%%%%%%%%%%%%%%%%%%%%%%%%%%%%%%%%%%%%%%%%%%%%%%%%%%%%%
%%%%%%%%%%%%%%%%%%%%%%%%%%%%%%%%%%%%%%%%%%%%%%%%%%%%%%%%%%%%%%%%%%
%%
%%  S E C T I O  N   5         S E C T I O  N   5           S E C T I O  N   5       S E C T I O  N   5
%%  S E C T I O  N   5         S E C T I O  N   5           S E C T I O  N   5       S E C T I O  N   5
%%  S E C T I O  N   5         S E C T I O  N   5           S E C T I O  N   5       S E C T I O  N   5
%%
%%%%%%%%%%%%%%%%%%%%%%%%%%%%%%%%%%%%%%%%%%%%%%%%%%%%%%%%%%%%%%%%%%
%%%%%%%%%%%%%%%%%%%%%%%%%%%%%%%%%%%%%%%%%%%%%%%%%%%%%%%%%%%%%%%%%%
\section{Very High-$z$ Quasars Detected in WISE}
\citet{Blain2013} 
Cum sociis natoque penatibus et magnis dis parturient montes, nascetur
ridiculus mus. Mauris at diam quis arcu pretium dictum. Quisque id
risus odio. Pellentesque posuere semper tempor. Donec volutpat quam ut
urna rutrum venenatis dapibus nunc interdum. Sed quis diam ac mauris
cursus accumsan. Maecenas sit amet libero in elit mattis iaculis sed
quis elit. Pellentesque vitae mauris nunc.

\subsection{Very High-$z$ Quasars Detected in WISE W4}

\subsection{Colours}
Cras scelerisque egestas ante vitae lacinia. Nunc velit nunc, tempus
congue lobortis auctor, pellentesque ac ipsum. Donec erat urna,
interdum eget blandit non, malesuada nec tellus. Aliquam erat
volutpat. Maecenas ut eros id lorem suscipit vulputate pulvinar quis
libero. Phasellus ipsum libero, sollicitudin vel luctus in, faucibus
at mauris. Fusce ultrices egestas turpis, eget mattis tellus
pellentesque vel. Donec ut ligula id nisi commodo tristique. Etiam
semper turpis eget purus rhoncus a malesuada massa ultrices. Fusce
luctus vehicula sem, eget scelerisque lacus fermentum sed. Nulla
facilisi. Aenean ut turpis sed quam ullamcorper mattis sit amet ut
justo. Morbi lorem dui, aliquam et pharetra non, luctus eu enim. Nulla
molestie hendrerit mi tincidunt interdum. Nunc a augue nunc.

\subsection{Mullaney SEDs}

\subsection{Number Density of High-$z$ Sources}
Cras quis nisl eget orci ultricies tempor quis id mi. Nullam ut
sollicitudin justo. Maecenas posuere fermentum nunc et
faucibus. Integer consectetur nibh a dolor blandit ornare. In et dui
id sapien suscipit commodo eget vitae dolor. Cum sociis natoque
penatibus et magnis dis parturient montes, nascetur ridiculus
mus. Donec ut elit sed urna semper imperdiet nec ac lorem. Integer
varius nibh a mauris posuere non convallis nibh commodo. Cras posuere
lorem sit amet orci elementum sit amet dapibus risus lacinia. Morbi
vitae lobortis dui. Nulla ac lacus tellus, a volutpat tortor.

\subsection{Consequences of evolution of the $M_{\rm BH} - M_{\star}$ relation}
From a study of 69 {\it Herschel}-detected broad-line active galactic nuclei, 
\citet{Sun2015} find there is no evolution in the $M_{\rm BH} - M_{\star}$ relation 
from $z\sim2$ to $z\sim0$, with the ratio of $\log (M_{\rm BH} / M_{\star}$ constant
at -2.85 across this redshift range. If this ratio holds to $z>5$, and we assume the 
$M_{\rm BH}$ for the $z>6$ measured from the e.g. \mgii line, are generally correct, 
then the galaxy total stellar mass for the $z>6$ objects should be...

\citet{YangG2018} calculate the long-term SMBH accretion rate as a function of 
$M_{\rm star}$ and redshift $[BHAR(M_{\star} ,z)]$ 
over ranges of 
$\log (M_{\rm star} / M_{\odot}) = 9.5–12$ and $z = 0.4–4$. 
Our BHAR(M⋆,z) is constrained by high-quality survey data (GOODS-South, GOODS-North and COSMOS), and by the stellar mass function and the X-ray luminosity function.
 This BHAR/SFR dependence on $M_{\star}$ does not support the scenario that SMBH and galaxy growth are in lockstep.


%%%%%%%%%%%%%%%%%%%%%%%%%%%%%%%%%%%%%%%%%%%%%%%%%%%%%%%%%%%%%%%%%%
%%%%%%%%%%%%%%%%%%%%%%%%%%%%%%%%%%%%%%%%%%%%%%%%%%%%%%%%%%%%%%%%%%
%%
%%  S E C T I O  N   7         S E C T I O  N   7           S E C T I O  N   7       S E C T I O  N   7
%%  S E C T I O  N   7         S E C T I O  N   7           S E C T I O  N   7       S E C T I O  N   7
%%  S E C T I O  N   7         S E C T I O  N   7           S E C T I O  N   7       S E C T I O  N   7
%%
%%%%%%%%%%%%%%%%%%%%%%%%%%%%%%%%%%%%%%%%%%%%%%%%%%%%%%%%%%%%%%%%%%
%%%%%%%%%%%%%%%%%%%%%%%%%%%%%%%%%%%%%%%%%%%%%%%%%%%%%%%%%%%%%%%%%%
\section{Discussion and Conclusions}
\label{sec:conclusions}
Cum sociis natoque penatibus et magnis dis parturient montes, nascetur
ridiculus mus. Duis tempus, lectus nec ultricies mollis, mi orci
feugiat nulla, a bibendum velit orci nec lacus. Duis a odio in nisi
egestas dictum. Nullam vel quam mauris, eget consectetur orci. Morbi
ac mi sit amet neque consectetur tempus ac eget est. Curabitur
malesuada arcu sit amet metus dictum at dapibus arcu accumsan. Fusce
sollicitudin luctus rutrum.

\begin{itemize}
    \item{This sample spans a redshift range of $0.28 < z < 4.36$ and has a bimodal distribution, with peaks 
        at $z\sim0.8$ and $z\sim2.5$.}
   \item{We recover a wide range of quasar spectra in this selection.  
        The majority of the objects have spectra of reddened Type 1
        quasars, Type 2 quasars (both at low and high redshift) and objects
        with strong absorption features.} 
    \item{There is a relatively high fraction of Type 2 objects at low redshift,
        suggesting that a high optical-to-infrared colour can be an efficient
        selection of narrow-line quasars.}
    \item{There are three objects that are detected in the $W4$-band but
        not $W1$ or $W2$ (i.e., ``W1W2-dropouts''), all of which are at
        $z>2.6$.}
    \item{We identify an intriguing class of objects at $z\simeq 2-3$ which are
        characterized by equivalent widths of REW(C\,{\sc iv})
        $\gtrsim150$\AA.  These objects often also have unusual line
        properties.  We speculate that the large REWs may be caused by
        suppressed continuum emission analogous to Type 2 quasars in the
        Unified Model. However, there is no obvious mechanism in the Unified
        Model to suppress the continuum without also suppressing the broad
        emission lines, thus potentially providing an interesting challenge to
        quasar models.} 
\end{itemize}
Nunc lacus nibh, convallis ac lobortis ut, tempus ac lectus. Maecenas
eu elit massa. Nulla vel lacus lorem. Proin et lobortis
tortor. Phasellus ultrices nisl non enim porttitor dictum. Curabitur
nec nunc ac nibh ornare elementum. Nunc ultrices hendrerit
ultricies. Aliquam dapibus semper est et gravida. Etiam cursus, massa
eget tempor elementum, lectus urna feugiat nisi, eget sagittis


\section*{Acknowledgements}

\newpage

\appendix
\section{Near-Infrared WFCAM Science Archive SQL queries}\label{sec:SQL}
Here we give the receipe and SQL that returned the near-infrared photometry 
for the VH$z$Qs. 

\begin{enumerate}
\item http://wsa.roe.ac.uk/ 
\item Login
\item {\tt username:	WSERV1000;  password: 	highzqso;   community: 	nonSurvey} 
\item Freeform SQL Query with  WSERV1000v20180327
\end{enumerate}

\lstset{upquote=true}

\noindent
Then the following SQL will return the values in
Table~\ref{tab:THE_TABLE}.

\begin{lstlisting}[
           language=SQL,
           showspaces=false,
           basicstyle=\ttfamily,
           numbers=left,
           numberstyle=\tiny,
           commentstyle=\color{gray}
        ]
SELECT 
qso.qsoName,  qso.raJ2000 as ra, qso.decJ2000 as dec, 
aver.apertureID,  aver.aperJky3 as aperJky3Aver, 
aver.aperJky3Err as aperJky3AverErr, aver.sumWeight, 
aver.ppErrBits as ppErrBitsAver, m.mjdObs, 
m.filterID, remeas.aperJky3, 
remeas.aperJky3Err, 
w.weight, remeas.ppErrBits, 
m.project

FROM 
finalQsoCatalogue as qso,  
MapApertureIDshighzQsoMap as ma,  
wserv1000MapRemeasAver as aver,  
wserv1000MapRemeasurement as remeas,  
MapProvenance as v,  
wserv1000MapAverageWeights as w, 
MapFrameStatus as mfs, 
Multiframe as m  

WHERE 
qso.qsoID=ma.objectID and 
ma.apertureID=aver.apertureID and 
aver.apertureID=remeas.apertureID and 
aver.catalogueID=v.combicatID and 
v.avSetupID=1 and 
v.catalogueID=remeas.catalogueID and 
w.combicatID=v.combicatID and 
w.catalogueID=v.catalogueID and 
w.apertureID=aver.apertureID and 
mfs.catalogueID=remeas.catalogueID and 
m.multiframeID=mfs.multiframeID and 
mfs.programmeID=10999 and 
mfs.mapID=1 
order by v.combicatID, m.mjdObs
\end{lstlisting}









%%%%%%%%%%%%%%%%%%%%%%%%%%%%%%%%%%%%%%%%%%%%%%%%%%%%%%%%%%%%%%%%%%%%
%%%%%%%%%%%%%%%%%%%%%%%%%%%%%%%%%%%%%%%%%%%%%%%%%%%%%%%%%%%%%%%%%%%%

%\bibliographystyle{apj}
\bibliographystyle{mn2e}
\bibliography{/cos_pc19a_npr/LaTeX/tester_mnras}

\end{document}
