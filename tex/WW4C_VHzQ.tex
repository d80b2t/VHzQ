%%%%%%%%%%%%%%%%%%%%%%%%%%%%%%%%%%%%%%%%%%%%%%%%%%%%%%%%%%%%%%%%%%%%%%%%%%%%%
%%                                                                           
%%  
%%									    
%%%%%%%%%%%%%%%%%%%%%%%%%%%%%%%%%%%%%%%%%%%%%%%%%%%%%%%%%%%%%%%%%%%%%%%%%%%%%
\documentclass[usenatbib]{mnras}

\usepackage{amsmath,, amssymb}
\usepackage{bm}% bold math
\usepackage{blindtext}
\usepackage{cancel, caption, color}
\usepackage{dcolumn}% Align table columns on decimal point
\usepackage{epsfig, epsf}
\usepackage{fancyhdr}
\usepackage[bottom,flushmargin,hang,multiple,para]{footmisc}
\usepackage{graphicx}
\usepackage{lscape, longtable, listings}
\usepackage{multirow}
\usepackage{natbib}
\usepackage{pdflscape}
%\usepackage{subcaption}
\usepackage{subfigure}
\usepackage{textcomp}
\usepackage{hyperref, ifthen}
\usepackage{verbatim}
\usepackage[usenames,dvipsnames]{xcolor}

%% This helps stop the friggin crazy erropr::
%% ! pdfTeX error (ext4): \pdfendlink ended up in different nesting level than \pd
%% fstartlink.
%% \AtBegShi@Output ...ipout \box \AtBeginShipoutBox 
\usepackage{etoolbox}
\makeatletter
\patchcmd\@combinedblfloats{\box\@outputbox}{\unvbox\@outputbox}{}{%
   \errmessage{\noexpand\@combinedblfloats could not be patched}%
}%
 \makeatother


%% http://en.wikibooks.org/wiki/LaTeX/Colors



%%%%%%%%%%%%%%%%%%%%%%%%%%%%%%%%%%%%%%%%%%%
%       define Journal abbreviations      %
%%%%%%%%%%%%%%%%%%%%%%%%%%%%%%%%%%%%%%%%%%%
\def\nat{Nat} \def\apjl{ApJ~Lett.} \def\apj{ApJ}
\def\apjs{ApJS} \def\aj{AJ} \def\mnras{MNRAS}
\def\prd{Phys.~Rev.~D} \def\prl{Phys.~Rev.~Lett.}
\def\plb{Phys.~Lett.~B} \def\jhep{JHEP} \def\nar{NewAR}
\def\npbps{NUC.~Phys.~B~Proc.~Suppl.} \def\prep{Phys.~Rep.}
\def\pasp{PASP} \def\aap{Astron.~\&~Astrophys.} \def\araa{ARA\&A}
\def\pasa{PASA}
\def\jcap{\ref@jnl{J. Cosmology Astropart. Phys.}}%
\def\physrep{Phys.~Rep.}


\newcommand{\preep}[1]{{\tt #1} }

%%%%%%%%%%%%%%%%%%%%%%%%%%%%%%%%%%%%%%%%%%%%%%%%%%%%%
%              define symbols                       %
%%%%%%%%%%%%%%%%%%%%%%%%%%%%%%%%%%%%%%%%%%%%%%%%%%%%%
\def \Mpc {~{\rm Mpc} }
\def \Om {\Omega_0}
\def \Omb {\Omega_{\rm b}}
\def \Omcdm {\Omega_{\rm CDM}}
\def \Omlam {\Omega_{\Lambda}}
\def \Omm {\Omega_{\rm m}}
\def \ho {H_0}
\def \qo {q_0}
\def \lo {\lambda_0}
\def \kms {{\rm ~km~s}^{-1}}
\def \kmsmpc {{\rm ~km~s}^{-1}~{\rm Mpc}^{-1}}
\def \hmpc{~\;h^{-1}~{\rm Mpc}} 
\def \hkpc{\;h^{-1}{\rm kpc}} 
\def \hmpcb{h^{-1}{\rm Mpc}}
\def \dif {{\rm d}}
\def \mlim {m_{\rm l}}
\def \bj {b_{\rm J}}
\def \mb {M_{\rm b_{\rm J}}}
\def \mg {M_{\rm g}}
\def \qso {_{\rm QSO}}
\def \lrg {_{\rm LRG}}
\def \gal {_{\rm gal}}
\def \xibar {\bar{\xi}}
\def \xis{\xi(s)}
\def \xisp{\xi(\sigma, \pi)}
\def \Xisig{\Xi(\sigma)}
\def \xir{\xi(r)}
\def \max {_{\rm max}}
\def \gsim { \lower .75ex \hbox{$\sim$} \llap{\raise .27ex \hbox{$>$}} }
\def \lsim { \lower .75ex \hbox{$\sim$} \llap{\raise .27ex \hbox{$<$}} }
\def \deg {^{\circ}}
%\def \sqdeg {\rm deg^{-2}}
\def \deltac {\delta_{\rm c}}
\def \mmin {M_{\rm min}}
\def \mbh  {M_{\rm BH}}
\def \mdh  {M_{\rm DH}}
\def \msun {M_{\odot}}
\def \z {_{\rm z}}
\def \edd {_{\rm Edd}}
\def \lin {_{\rm lin}}
\def \nonlin {_{\rm non-lin}}
\def \wrms {\langle w_{\rm z}^2\rangle^{1/2}}
\def \dc {\delta_{\rm c}}
\def \wp {w_{p}(\sigma)}
\def \PwrSp {\mathcal{P}(k)}
\def \DelSq {$\Delta^{2}(k)$}
\def \WMAP {{\it WMAP \,}}
\def \cobe {{\it COBE }}
\def \COBE {{\it COBE \;}}
\def \HST  {{\it HST \,\,}}
\def \Spitzer  {{\it Spitzer \,}}
\def \ATLAS {VST-AA$\Omega$ {\it ATLAS} }
\def \BEST   {{\tt best} }
\def \TARGET {{\tt target} }
\def \TQSO   {{\tt TARGET\_QSO}}
\def \HIZ    {{\tt TARGET\_HIZ}}
\def \FIRST  {{\tt TARGET\_FIRST}}
\def \zc {z_{\rm c}}
\def \zcz {z_{\rm c,0}}

\newcommand{\ltsim}{\raisebox{-0.6ex}{$\,\stackrel
        {\raisebox{-.2ex}{$\textstyle <$}}{\sim}\,$}}
\newcommand{\gtsim}{\raisebox{-0.6ex}{$\,\stackrel
        {\raisebox{-.2ex}{$\textstyle >$}}{\sim}\,$}}
\newcommand{\simlt}{\raisebox{-0.6ex}{$\,\stackrel
        {\raisebox{-.2ex}{$\textstyle <$}}{\sim}\,$}}
\newcommand{\simgt}{\raisebox{-0.6ex}{$\,\stackrel
        {\raisebox{-.2ex}{$\textstyle >$}}{\sim}\,$}}

\newcommand{\Msun}{M_\odot}
\newcommand{\Lsun}{L_\odot}
\newcommand{\lsun}{L_\odot}
\newcommand{\Mdot}{\dot M}

\newcommand{\sqdeg}{deg$^{-2}$}
\newcommand{\hi}{H\,{\sc i}\ }
\newcommand{\lya}{Ly$\alpha$\ }
%\newcommand{\lya}{Ly\,$\alpha$\ }
\newcommand{\lyaf}{Ly\,$\alpha$\ forest}
%\newcommand{\eg}{e.g.~}
%\newcommand{\etal}{et~al.~}
\newcommand{\lyb}{Ly$\beta$\ }
\newcommand{\cii}{C\,{\sc ii}\ }
\newcommand{\ciii}{C\,{\sc iii}]\ }
\newcommand{\civ}{C\,{\sc iv}\ }
\newcommand{\SiII}{Si\,{\sc ii}\ }
\newcommand{\SiIV}{Si\,{\sc iv}\ }
\newcommand{\mgii}{Mg\,{\sc ii}\ }
\newcommand{\feii}{Fe\,{\sc ii}\ }
\newcommand{\feiii}{Fe\,{\sc iii}\ }
\newcommand{\caii}{Ca\,{\sc ii}\ }
\newcommand{\halpha}{H\,$\alpha$\ }
\newcommand{\hbeta}{H\,$\beta$\ }
\newcommand{\hgamma}{H\,$\gamma$\ }
\newcommand{\hdelta}{H\,$\delta$\ }
\newcommand{\oi}{[O\,{\sc i}]\ }
\newcommand{\oii}{[O\,{\sc ii}]\ }
\newcommand{\oiii}{[O\,{\sc iii}]\ }
\newcommand{\heii}{He\,{\sc ii}\ }
%\newcommand{\heii}{[He\,{\sc ii}]\ }
\newcommand{\nv}{N\,{\sc v}\ }
\newcommand{\nev}{Ne\,{\sc v}\ }
\newcommand{\neiii}{[Ne\,{\sc iii}]\ }
\newcommand{\alii}{Al\,{\sc ii}\ }
\newcommand{\aliii}{Al\,{\sc iii}\ }
\newcommand{\siiii}{Si\,{\sc iii}]\ }


\begin{document}

\title[Very high-$z$ Quasars]
        {Near and Mid-infrared properties of known $z\geq5$ Quasars}
\author[Ross \& Cross]
       {Nicholas P. Ross$^{1}$\thanks{Corresponding Author: npross@roe.ac.uk} and Nicholas J. G. Cross$^{1}$
\\ 
$^1$Institute for Astronomy, University of Edinburgh, Royal Observatory, Edinburgh, EH9 3HJ, United Kingdom\\
}

\maketitle
\begin{abstract}
In this paper, we, for the first time since the discovery of $z\geq5$
quasars, assemble all spectroscopically confirmed very high redshift
quasars in one catalogue.  In particular we present the near
($zZyYJHK_{s}$ and $K$) infrared and mid-infrared (WISE) properties of
all 463 spectroscopically confirmed redshift $z\geq5.00$ quasars.
Using archival public WFCAM/UKIRT and VIRCAM/VISTA data we check for
photometric variability in the near-infrared that might be expected
from Super-Eddington accretion and find {\it blah}.  We present a
comprehensive series of colour-redshift and colour-colour plots and
make inferences into the hot dust properties of the very high-redshift
quasar population. Extrapolating the known quasar luminosity function
we suggest that $x$\% of the possibly detected $z\geq5$ quasars in the
current datasets have been discovered.
\end{abstract}


\begin{keywords}
Astronomical data bases: surveys -- 
Quasars: general -- 
galaxies: evolution -- 
galaxies: infrared.
\end{keywords}

\iffalse
\section*{TO DOs}
{\bf For NPR: }
\begin{itemize}
\item Concentrate on reporting NIR and W1/W2 figures and tables
\item SED dust plots 
\item Filter curve plot with LSST, Euclid and even wide JWST filters...; 
\item Check for NIR and MIR variabilty with the Wu quasar
\item Check with SpIES/SHELA (but only 3.4/4.6$\mu$m; 
%%\item Almost certainly want to compile M$_{\rm BH}$; THIS CAN WAIT, otherwise is massive paper-creep
%%\item May well wanna try and get M$_{\star}$ too...!! (yuck); THIS CAN WAIT, otherwise is massive paper-creep
\end{itemize}

\noindent
{\bf For NJGC: }
\begin{itemize}
\item Write the NIR data section(s) 
\end{itemize}

\noindent
{\bf For Both: }
\begin{itemize}
\item Need to decide what mags and mag system we are reporting, especially in the 
NIR...
\item Depth/area QLF calculation to see whether the NIR surveys have been fully mined...
\item Question from Sarah Bosman:: are the $y$-magnitudes AB? They seem all quite a bit brighter than the ones given in Banados+16. Do you know why that is?
\end{itemize}
\fi


%%%%%%%%%%%%%%%%%%%%%%%%%%%%%%%%%%%%%%%%%%%%%%%%%%%%%%%%%%%%%%%%%%
%%%%%%%%%%%%%%%%%%%%%%%%%%%%%%%%%%%%%%%%%%%%%%%%%%%%%%%%%%%%%%%%%%
%%
%%  S E C T I O  N   1         S E C T I O  N   1           S E C T I O  N   1       S E C T I O  N   1
%%  S E C T I O  N   1         S E C T I O  N   1           S E C T I O  N   1       S E C T I O  N   1
%%  S E C T I O  N   1         S E C T I O  N   1           S E C T I O  N   1       S E C T I O  N   1
%%
%%%%%%%%%%%%%%%%%%%%%%%%%%%%%%%%%%%%%%%%%%%%%%%%%%%%%%%%%%%%%%%%%%
%%%%%%%%%%%%%%%%%%%%%%%%%%%%%%%%%%%%%%%%%%%%%%%%%%%%%%%%%%%%%%%%%%
\section{Introduction}
Very high redshift quasars (VH$z$Q; defined here to have redshifts
$z\geq5.00$) are excellent probes of the early Universe. This includes
studies of the Epoch of Reionization for hydrogen \citep[see e.g.][for
reviews]{Fan2006review, Mortlock2016}, the formation and build-up of
supermassive black holes \citep[e.g., ][]{Rees1984, WyitheLoeb2003,
Volonteri2010, Agarwal2016, Valiante2018, Latif2018} and early metal
enrichment \citep[see e.g., ][]{Simcoe2012, Chen2017, Bosman2017}.

Super-critical accretion, where $\dot{M} > \dot{M}_{\rm Edd}$, is a
viable mechanism to explain the high, potentially super-Eddington,
luminosity and rapid growth of supermassive black holes in the early
universe \citep[e.g.,][]{AlexanderNatarajan2014, MadauHaardtDotti2014,
Volonteri2015, Pezzulli2016, Lupi2016, Pezzulli2017, Takeo2018}. Thus,
one might well expect VH$z$Qs to vary in luminosity as they
potentially go through phases of super-critical accretion and these
signatures of photometric variability should be looked for, noting the
rest-frame optical emission is redshifted into the observed
near-infrared (NIR) at redshifts $z>5$. Fortunately, data are now in
place from deep, wide-field NIR instruments and surveys such as the
Wide Field Camera (WFCAM) instrument on the United Kingdom Infra-Red
Telescope (UKIRT) in the Northern Hemisphere and the VISTA InfraRed
CAMera (VIRCAM) on the Visible and Infrared Survey Telescope for
Astronomy (VISTA) in the Southern Hemisphere, that are necessary for
identifying VH$z$Qs.

Quasars are known to be prodigious emitters of infrared emission,
thought to be from the thermal emission of dust grains heated by
continuum emission from the accretion disc
\citep[e.g.,][]{Richards2006b, Leipski2014, Hill2014, Hickox2017}.
Observations in the mid-infrared, e.g. $\sim$3-30$\mu$m allow
discrimination between AGN\footnote{Historically, ``quasars'' and
``Active Galactic Nuclei (AGN)'' have described different
luminosity/classes of objects, but here we use these terms
interchangeably, with a preference for quasar, in recognition of the
fact that they both describe accreting supermassive black holes
\citep[e.g.][]{Haardt2016book}.}  and passive galaxies due to the
1.6$\mu$m ``bump'' entering the MIR at $z\approx0.8-0.9$ \citep[e.g.,
][]{Wright1994, Sawicki2002, Lacy2004, Stern2005, Richards2006b,
Timlin2016} as well as between AGN and star-forming galaxies due to
the presence of Polycyclic Aromatic Hydrocarbon (PAHs) at $\lambda
>3\mu$m \citep[e.g., ][]{Yan2007, Tielens2008}.

\citet{Jiang2006dust} and \citet{Jiang2010} report on the discovery of
a quasar without hot-dust emission in a sample of 21 $z\approx6$
quasars. Such apparently hot-dust-free quasars have no counterparts at
low redshift. Moreover, those authors demonstrate that the hot-dust
abundance in the 21 quasars builds up in tandem with the growth of the
central black hole. But understanding how dust first forms and appears
in the central engine remains an open question \citep{WangR2008, WangR2011}. 

WISE mapped the sky in 4 passbands, in bands centered at wavelengths
of 3.4, 4.6, 12, and 23$\mu$m. The all sky ``ALLWISE'' catalog
release, contains nearly 750 million detections at
high-significance\footnote{\href{wise2.ipac.caltech.edu/docs/release/allwise/expsup/sec2\_1.html}{wise2.ipac.caltech.edu/docs/release/allwise/expsup/sec2\_1.html}},
of which over 4.5M AGN candidates have been identified with 90\%
reliability \citep{Assef2018}.  \citet{Blain2013} presented WISE
mid-infrared (MIR) detections of 17 (55\%) of the then known 31
quasars at $z > 6$. However, \citet{Blain2013} was compiled with the
WISE `All-Sky' data release, as opposed to the superior ``AllWISE''
catalogs. That sample only examined the 31 known $z>6$ quasars; our
sample has 148 ({\bf NPR to Double Check!}) objects with redshift $z
\geq 6.00$.

Critically, we now have available to us new W1 and W2 photometry from
the `unWISE Source Catalog' \citep[][]{Schlafly_Meisner2018}, a
WISE-selected catalog that is based on significantly deeper imaging
and has a more extensive modeling of crowded regions than the ALLWISE
release. For the first time in a catalog, unWISE takes advantage of
the ongoing mid-IR Near-Earth Object Wide-Field Infrared Survey
Explorer Reactivation mission \citep[NEOWISE-R; ][]{Mainzer2014}, and
achieves depths $\sim$0.7 mag deeper than ALLWISE (in W1/2).  This
additional depth is a significant advantage in the detection and study
of VH$z$Qs in the 3-5 micron regime.

Here we update \citet{Jiang2010} and \citet{Blain2013} \citep[along
with Table 8 of][]{Banados2016}. Our motivations are numerous and
include: {\it (i)} establishing the first complete catalogue of
$z>5.00$ quasars since the pioneering work from SDSS; {\it (ii)}
analyzing all the WFCAM and VISTA near-infrared photometry for the
quasars; {\it (iii)} making the first study of NIR variability of the
VHzQ population and {\it (iv)} establishing the photometric properties
for upcoming surveys and telescopes, e.g. the Large Synoptic Survey
Telescope (LSST)\footnote{\href{https://www.lsst.org}{{lsst.org}}},
ESA {\it
Euclid}\footnote{\href{https://sci.esa.int/euclid/}{sci.esa.int/euclid/}}
and the {\it James Webb Space Telescope}
(JWST)\footnote{\href{https://www.jwst.nasa.gov/}{jwst.nasa.gov};}$^,$\footnote{\href{https://sci.esa.int/jwst/}{sci.esa.int/jwst};}$^,$\footnote{\href{https://www.asc-csa.gc.ca/eng/satellites/jwst/}{www.asc-csa.gc.ca/eng/satellites/jwst};}$^,$\footnote{\href{https://jwst.stsci.edu/}{jwst.stsci.edu}}.

This paper can be considered an update of \citet{Blain2013} and also
an extension of parts of \citet{Banados2016}, with the latter study
reporting WISE W1, W2, W3 and W4 magnitudes for the Panoramic Survey
Telescope and Rapid Response System 1 \citep[Pan-STARRS1,
PS1;][]{Kaiser2002, Kaiser2010}, but with no further investigation
into the reddest WISE waveband for the VH$z$Qs.  \citet{Banados2016}
reports and investigates the W1, W2 and W3 properties of quasars at $z
> 5.6$. We chose redshift $z=5.00$ as our lower redshift limit due to
a combination of garnishing a large sample, adequately spanning
physical properties (e.g. luminosity, age of the Universe) and to
incorporate what knowledge we have gained over the last couple of
decades since $z>5$ quasars were discovered.

This paper is organized as follows.  In Section 2, we present the
assembled list of the 463 $z\geq5.00$ VH$z$Qs that we have
compiled. We then give a high-level overview of the photometric
surveys and datasets we use and present the photometry of the VH$z$Qs.
In Section 3 we ... and In Section 4 we ... We conclude in Section 5 
and present all the necessary details to obtain our dataset in the 
Appendices. 

We make the decision to present all our photometry and magnitudes on
the AB zero-point system \citep{Oke_Gunn1983, Fukugita1996}.  This
includes the near-infrared, as well as the mid-infrared magnitudes.
Appendix~\ref{sec:filters} gives the AB to Vega transforms for a wide
range of optical, NIR and MIR filters. 

We use a flat $\Lambda$CDM cosmology with $H0 = 67.7$ km s$-1$ Mpc$−1$, $\Omega_{\rm M} = 0.307$, and $\Omega_{\Lambda} = 0.693$ (Planck Collaboration et al. 2016) in order to be consistent with \citet{Banados2016} and all logarithms are to the
base 10. 



%%%%%%%%%%%%%%%%%%%%%%%%%%%%%%%%%%%%%%%%%%%%%%%%%%%%%%%%%%%%%%%%%%
%%%%%%%%%%%%%%%%%%%%%%%%%%%%%%%%%%%%%%%%%%%%%%%%%%%%%%%%%%%%%%%%%%
%%
%%  S E C T I O  N   2         S E C T I O  N   2           S E C T I O  N   2       S E C T I O  N   2
%%  S E C T I O  N   2         S E C T I O  N   2           S E C T I O  N   2       S E C T I O  N   2
%%  S E C T I O  N   2         S E C T I O  N   2           S E C T I O  N   2       S E C T I O  N   2
%%
%%%%%%%%%%%%%%%%%%%%%%%%%%%%%%%%%%%%%%%%%%%%%%%%%%%%%%%%%%%%%%%%%%
%%%%%%%%%%%%%%%%%%%%%%%%%%%%%%%%%%%%%%%%%%%%%%%%%%%%%%%%%%%%%%%%%%
\begin{figure}
%%trim=l b r t
%  \includegraphics[width=18.6cm, clip,trim=14mm 4mm 10mm 10mm]
  \includegraphics[width=8.6cm, clip,trim=32mm 4mm 32mm 10mm]
  {/cos_pc19a_npr/programs/quasars/highest_z/SEDs/filters_vs_QSOstars_z7pnt0.pdf}
  \centering
  \vspace{-12pt}
  \caption[]
  {The spectral bands used by different survey telescopes and that are relevant here.
    The $grizy$ filters are from the Pan-STARRS survey. The $JHK$ are from 
    UKIRT/WFCAM, while $K_{s}$ is a VISTA/VIRCAM filter. The 
    The narrow W-band centered at $\lambda\approx14,500$\AA\ is a CFHT/Wircam filter. 
    [NJC: Do we use the W-band anywhere?]
    The WISE passbands  W1-4 are also presented.
    The quasar spectrum is a composite based on \citet{VdB2001} and 
    \citet{Banados2016}. The L and T dwarf spectra are from \citet{Cushing2006}. 
  }
  \label{fig:filters}
\end{figure}

\vspace{-16pt}
\section{Data}
First we compile the list of all known, spectroscopically confirmed quasars from the literature. Most of these objects are easily identified by their broad Ly$\alpha$ emission line, \nv emission and characteristic shape blueward of 1215\AA\ in the rest-frame. As we shall see, some of the recently discovered objects are close to the galaxy luminosity function characteristic luminosity $M^{*}$, and some have relatively weak or maybe even completely absorbed Ly$\alpha$ \citep[e.g. Figures 7 and 10 in][]{Banados2016}. We leave aside detailed investigation and discussion into spectral features and line strengths, and take as given the published spectra and reshift identifications.

We then obtain optical, near-infrared and mid-infrared photometry for the spectral dataset. The optical data comes from the Panoramic Survey Telescope and Rapid Response System (Pan-STARRS) survey \citep{Chambers2016}. The near-infrared data comes from two sources: first, the WFCAM \citep[][]{Casali2007} on the UKIRT, primarly, but not exclusively, as part of the UKIRT Infrared Deep Sky Survey \citep[UKIDSS; ][]{Lawrence2007}.  And second, data from the VIRCAM on the VISTA \citep[][]{Emerson2006, Dalton2006}. The mid-infrared, $\lambda=3-30\mu$m wavelength data is from the the Wide-Field Infrared Survey Explorer \citep[WISE;][]{Wright2010, Cutri2013} mission. 
%%
For reference, Figure~\ref{fig:filters} displays the wavelength and normalised transmission of the filters in question. 

Tables~\ref{tab:THE_TABLE}, \ref{tab:THE_TABLE_NIR} and~\ref{tab:THE_TABLE_MIR}, represent the culmination of this effort, and we now describe the assembly of their contents in more detail.  

%%%%%%%%%%%%%%%%%%%%%%%%%%%%%%%%%%%%%%%%%%%%%%%%%%%%%%%%%%%%%%%%%%%%%%%%%%%%%%%%
%%
%%
%%     T A B L E    O N E                GENERAL PROPERTIES,          incl.   redshift   and   M_1450
%%
%%
%%%%%%%%%%%%%%%%%%%%%%%%%%%%%%%%%%%%%%%%%%%%%%%%%%%%%%%%%%%%%%%%%%%%%%%%%%%%%%%%
\begin{table*}
\begin{center}
\begin{tabular}
%{|l|l|l|l|r|r|r|r|r|r|r|r|r|r|r|r|r|r|r|r|r|r|r|r|l|}
%{ l l l l r r r r r r r r r r r r r r r r r r r r l }
{ l l   r r  r r   l   l l l   }
\hline \hline
  \multicolumn{1}{ c }{na} &
  \multicolumn{1}{c }{desig} &
  \multicolumn{1}{c }{ra\_hms} &
  \multicolumn{1}{c }{dec\_dms} &
  \multicolumn{1}{c }{ra} &
  \multicolumn{1}{c }{dec} &
  \multicolumn{1}{c }{redshift} &
  \multicolumn{1}{c }{mag} &
  \multicolumn{1}{c }{M1450} &
  \multicolumn{1}{c }{ref} \\
\hline
  PSO & J000.3401+26.8358   & 00:01:21.63 & +26:50:09.17 &   0.340113    &  +26.83588      &   5.75    & 19.52 & -27.16     & 1/1/1\\
  SDSS & J0002+2550             & 00:02:39.39 & +25:50:34.80 &   0.664117    &  +25.84304      &  5.82    & 19.39 & -27.31     & 5/22/1\\
  SDSS & J0005-0006             & 00:05:52.34 & -00:06:55.80 &    1.468083    &  -00.11549      & 5.85    & 20.98 & -25.73     &  5/12/1\\
  PSO & J002.1073-06.4345   & 00:08:25.77 & -06:26:04.60 &    2.107390    &  -06.43456            & 5.93   & 20.41 & -26.32      &   1;43/1/1\\
  SDWISE & J0008+3616         & 00:08:51.43 & +36:16:13.49 &   2.214292    &   +36.27041        & 5.17   & 19.12 & -27.34     &    Wang2016\\
  PSO & J002.3786+32.8702  & 00:09:30.89 & +32:52:12.94 &    2.378702   &   +32.87026     &  6.1     & 21.13  & -25.65    &  1/1/1\\
  SDSS & J0017-1000             & 00:17:14.68 & -10:00:55.4 &      4.311166   &    -10.01540    & 5.011  & 99.99 & -99.99     &   DR7\_W16\\
  PSO & J004.3936+17.0862  & 00:17:34.47 & +17:05:10.70 &    4.393614   &   +17.08631      & 5.8      & 20.69 & -26.01     &  1/1/1\\
  PSO & J004.8140-24.2991  & 00:19:15.38 & -24:17:56.98 &     4.814080   &   -24.29920          & 5.68      & 19.43 & -27.24      &    1/1/1\\
  VDES & J0020-3653            & 00:20:31.46 & -36:53:41.8 &       5.131124   &   -36.89495       & 6.9      & 99.99 & -99.99       &   DES-VHS\_inprep\\
\hline \hline
\end{tabular}
\caption{All 463 $z\geq5.00$ quasars that have been spectroscopically confirmed as of 2018 June. 
  The first ten objects are given here as guidance to the format of the data table. The full table  can be found online.} 
\label{tab:THE_TABLE}
  \end{center}
\end{table*}
\normalsize 



%\input{table_tex/THE_TABLE_v0pnt96_TOP10_redshifts}
%%%%%%%%%%%%%%%%%%%%%%%%%%%%%%%%%%%%%%%%%%%%%%%%%%%%%%%%%%%%%%%%%%%%%%%%%%%%%%%%%
%%
%%
%%     T A B L E     T W O                     NEAR     IR     PROPERTIES         
%%
%%
%%%%%%%%%%%%%%%%%%%%%%%%%%%%%%%%%%%%%%%%%%%%%%%%%%%%%%%%%%%%%%%%%%%%%%%%%%%%%%%%

\medskip
\medskip
\begin{landscape}
\begin{table}
\begin{center}
\begin{tabular}
%%  {|l|l|l|l|r|r|r|r|r|r|r|r|r|r|r|r|r|r|r|r|r|r|r|r|l|}
%%  { l l l l r r r r r r r r r r r r r r r r r r r r l }
  { l l l l l l l l l l l l l l l l l l l l l l l l l }
  \hline \hline
  \multicolumn{1}{ c }{na} &
  \multicolumn{1}{c }{desig} &
  \multicolumn{1}{c }{ra} &
  \multicolumn{1}{c }{dec} &
  \multicolumn{1}{c }{w1mag} &
  \multicolumn{1}{c }{w1err} &
  \multicolumn{1}{c }{w1snr} &
  \multicolumn{1}{c }{w2mag} &
  \multicolumn{1}{c }{w2err} &
  \multicolumn{1}{c }{w2snr} &
  \multicolumn{1}{c }{w3mag} &
  \multicolumn{1}{c }{w3err} &
  \multicolumn{1}{c }{w3snr} &
  \multicolumn{1}{c }{w4mag} &
  \multicolumn{1}{c }{w4err} &
  \multicolumn{1}{c }{w4snr} \\
\hline
  PSO & J000.3401+26.8358   &   0.34011348  & 26.83588138     & 16.373      & 0.066  & 16.5 &    15.266 & 0.107 & 10.2 & 12.594 & 0.492 & 2.2 & 8.756 & -9.99 & 1.1 \\
  SDSS & J0002+2550             &   0.66411726  & 25.84304425     & 16.162      & 0.057  & 19.0 &    15.542 & 0.127 & 8.5 & 12.416 & 0.423 & 2.6 & 8.683 & -9.99 & 1.2 \\
  SDSS & J0005-0006             &    1.4680833   & -0.1154999        & 17.299      & 0.16   &  6.8 &      17.043 & -9.99 & 0.2 & 12.445 & -9.99 & -1.1 & 9.008 & -9.99 & -0.3 \\
  PSO & J002.1073-06.4345   &    2.10739       & -6.43456            &  16.809     & 0.107 & 10.1 &     15.684 & 0.141 & 7.7 & 11.892 & -9.99 & 1.5 & 8.759 & -9.99 & 0.2  \\
  SDWISE & J0008+3616         &    2.2142917   & 36.2704138        &  16.045     & 0.052 & 20.7 &     15.373 & 0.092 & 11.8 & 12.043 & -9.99 & 1.8 & 8.786 & -9.99 & 1.1 \\
  PSO & J002.3786+32.8702   &   2.37870183  & 32.87026179      &  -99.99     & -9.99  & -9.9 &   -99.99 & -9.99 & -9.9 & -99.99 & -9.99 & -9.9 & -9.99 & -9.99 & -9.9 \\
  SDSS & J0017-1000              &   4.3111666   & -10.01539722     & 15.936      & 0.055 & 19.7 &     15.167 & 0.094 & 11.5 & 12.026 & 0.334 & 3.2 & 8.52 & -9.99 & 1.2 \\
  PSO & J004.3936+17.0862   &   4.39361347  & 17.08630447      &  -99.99     & -9.99  & -9.9 &    -99.99 & -9.99 & -9.9 & -99.99 & -9.99 & -9.9 & -9.99 & -9.99 & -9.9 \\
  PSO & J004.8140-24.2991   &   4.81408        & -24.29916           &  16.281    & 0.069 & 15.8 &      15.569 & 0.116 & 9.4 & 12.123 & 0.344 & 3.2 & 8.82 & -9.99 & 0.5 \\
  VDES & J0020-3653             &   5.1311237     & -36.8949476      &  16.844  & 0.094 & 11.6 &        16.354 & 0.204 & 5.3 & 12.679 & -9.99 & -0.1 & 8.342 & -9.99 & 0.8 \\
\hline \hline
\end{tabular}
\caption{The mid-infrared photometric properties from the WISE ALLWISE
catalogue for the 424 very-high redshift quasars.  The first ten
objects are given here as guidance to the format of the data
table. The full table can be found online.  {\it This is the third
table here; the SECOND table is this with the NIR data...}}
\label{tab:THE_TABLE_NIR}

\end{center}
\end{table}
\normalsize 
\end{landscape}

%\input{table_tex/THE_TABLE_v0pnt96_TOP10_MIR}




\subsection{Spectroscopy} 
We have obtained a list of 463 spectroscopically confirmed quasars
with redshifts $z\geq5.00$. We use all the $z\geq5.00$ quasars that
have been discovered and spectroscopically confirmed as of the time of
writing (2018 November).

This list was complied from a range of surveys and papers. 
Specifically, we use data from: 
\citet{Banados2014, Banados2016, Banados2018}, 
\citet{Becker2015}, 
\citet{Calura2014}, 
\citet{Carilli2007, Carilli2010}, 
\citet{Carnall2015}, 
\citet{Cool2006}, 
\citet{DeRosa2011}, 
\citet{Fan2000, Fan2001c, Fan2003, Fan2004, Fan2006, Fan2018}, 
\citet{Goto2006}, 
\citet{Ikeda2017}, 
\citet{Jiang2008, Jiang2009, Jiang2015, Jiang2016},   
\citet{Kashikawa2015}, 
\citet{Koptelova2017}, 
\citet{Kim2015, Kim2018},  
\citet{Kurk2007, Kurk2009}, 
\citet{Leipski2014}, 
\citet{Mahabal2005}, 
\citet{Matsuoka2016,  Matsuoka2018a, Matsuoka2018b},   
\citet{Mazzucchelli2017}, 
\citet{Morganson2012}, 
\citet{Mortlock2009, Mortlock2011},
\citet{McGreer2006, McGreer2013},  
\citet{Reed2015, Reed2017}, 
\citet{Stern2007},  
\citet{Tang2017}, 
\citet{Venemans2007, Venemans2012, Venemans2013, Venemans2015a, Venemans2015b, Venemans2016},
\citet{WangF2016, WangF2017, WangF2018a, WangF2018b},
\citet{Willott2007, Willott2009, Willott2010a, Willott2013b, Willott2015}, 
\citet{Wu2015} 
\citet{YangJ2018a, YangJ2018b}  
and 
\citet{Zeimann2011}, 
The breakdown of how many VH$z$Q each survey reports is given in
Table~\ref{tab:surveys}. We assume the reader is familiar with the
Sloan Digital Sky Survey (SDSS) and the Pan-STARRS1 (PS1; PSO in
Table~\ref{tab:surveys}.  ) survey and these surveys alone identified
over half (54.6\%) of the VH$z$Q population. Data from the
Hyper Suprime-Cam (HSC) on the Subaru telecope is responsible for
13.6\% of our dataset (HSC+SHELLQs in Table~\ref{tab:surveys}). The
combination of surveys is also vital for identifying VH$z$Qs. The
UKIDSS Large Area Survey (ULAS) on its own, or in combination with
other surveys is responsible for 6.5\% of the sample (SUV+ULAS)
including the highest-$z$ object. Where more than one survey is used
for the high-redshift identification (e.g. via shorter-band veto and
longer wavelength detection) we follow the discovery paper naming
convention.


\begin{table}
\begin{tabular}{l r r l}
\hline  \hline
Survey              & \# VH$z$Qs & (\%) & Notes/Survey reference  \\
\hline  
  ATLAS             &     4    &   ( 0.86)    &  \citet{Shanks2015} \\
  CFHQS            &   20    &   ( 4.32)    &  \citet{Willott2007} \\
  DELS               &   16    &   ( 3.46)    &  \citet{Dey2018} \\
  ELAIS              &     1    &   ( 0.22)    &  \citet{Vaisanen2000} \\
  FIRST              &     1    &   ( 0.22)    &  \citet{Becker1995} \\
  HSC                 &    8    &   ( 1.73)    & \citet{Miyazaki2018} \\
  IMS                 &     5    &   ( 1.08)     &  \citet{Kim2015} \\
  MMT               &   12    &   ( 2.59)     &  \citet{McGreer2013} \\
  NDWFS           &     1    &   ( 0.22)    &  \citet{JD1999} \\
  PSO                 &   83   &   (17.93)   &   \citet{Kaiser2002, Kaiser2010} \\
  RD                   &     1   &   ( 0.22)    &  \citet{Mahabal2005} \\
  SDSS                &  170  &    (36.72)    & \citet{EDR} \\
 SDWISE$^{b}$    &   27    &  ( 5.83)    &   \citet{WangF2016} \\
%% SHELLQs    &  $^{c}$63  &  (13.6)   &  \citet{Matsuoka2016} \\  %% if you include 8 objects from the first HSC paper...
  SHELLQs         &    55    &   (11.88)  &  \citet{Matsuoka2016}     \\  
  SUV$^{c}$       &   20     &    ( 4.32)  & \citet{YangJ2017} \\
  UHS               &    1      &  ( 0.22)     &  \citet{WangF2017} \\
  ULAS               &   10   &   ( 2.16)     & \citet{Lawrence2007} \\
  VDES$^{d}$       &   17  &    ( 3.67)     &  \citet{Reed2017} \\
  VHS                 &     1  &     ( 0.22)    & \citet{WangF2018b} \\
  VIK                 &     9    &  ( 1.94)    &  \citet{Edge2013} \\
  VIMOS           &    1      &  ( 0.22)     &   \citet{LeFevre2003} \\
\hline  \hline
\end{tabular}
\caption{The source and number of the VH$z$Q, with the key survey reference also given. 
  Recent survey name and acronyms include: 
  $^{a}$DESI Legacy Imaging Survey; 
  $^{b}$SDWISE = SDSS+WISE; 
  $^{a}$SUV  = SDSS-ULAS/VHS; 
  $^{c}$VDES = VHS/VIKING+DES; 
%  $^{c}$Includes 8 objects with a HyperSuprimeCam \citep[HSC; ][]{Miyazaki2018} designation.
}
      \label{tab:surveys}
\end{table}

\begin{figure}
%%trim=l b r t
  \includegraphics[width=8.0cm, clip, trim=10mm 0mm 0mm 0mm]
  {/cos_pc19a_npr/programs/quasars/highest_z/Nofz/Nofz_0pnt075bins_20181211.png}
  \centering
  \vspace{-12pt}
  \caption[]
  {The redshift distribution $N(z)$ of the VH$z$Q sample. 
    The bins are $\delta-z=0.075$ in width. }
  \label{fig:Nofz}
\end{figure}

The redshifts for the VH$z$Qs generally come from the measurement of
broad UV/optical emission lines. Where there are far infra-red
emission lines e.g. \cii~158 micron, we report these, but at the level
of our current analysis broadline redshifts are sufficient.

\begin{table}
\centering
\begin{tabular}{l r  r}
\hline \hline
$z \geq$  & Age / Myr & No. of objects \\
\hline 
7.50         &    700         &   1   \\
7.00         &    767         &   4   \\
6.78         &    800         &  14   \\
6.50         &    845         &  40   \\
6.19         &   900          &  86   \\
6.00         &   937          &  170   \\
5.70         & 1000          &  267   \\
5.00         & 1180          &  463   \\
\hline \hline
\end{tabular}
\caption{The number of objects at or above a given redshift. 
The age of the Universe in megayears is also given. }
      \label{tab:ages}
\end{table}

The number of objects at or above various redshifts, along with the 
corresponding age of the Universe is given in Table~\ref{tab:ages}. 

The $N(z)$ redshift histogram is given for the sample in Figure~\ref{fig:Nofz}. 
We split the contribution up by survey. For clarity we show the individual 
surveys of SDSS, PS1, HSC, the ULAS detection, and tally the remaining 
surveys together (``various''). In Section~\ref{sec:Lz} we discuss the form 
of the $N(z)$ as well as the coverage of the luminosity-redshift $L-z$ plane. 

        
\subsection{Optical Photometry}
%\subsubsection{Pan-STARRS1 (PS1)} 
We query the Panoramic Survey Telescope and Rapid Response System (Pan-STARRS)\footnote{\href{https://outerspace.stsci.edu/display/PANSTARRS}{https://outerspace.stsci.edu/display/PANSTARRS}} Data Release 1 (DR1) Catalog Archive Server Jobs System (CasJobs) service at \href{http://mastweb.stsci.edu/ps1casjobs/}{mastweb.stsci.edu/ps1casjobs/}. The Pan-STARRS1 (PS1) survey observed the 30,000 deg$^{2}$ of sky north of declination $\delta = -30$ degrees in the five $grizy$ filters.  PS1 is the first part of Pan-STARRS to be completed and is the basis for the DR1. \citet{Chambers2016}, \citet{Magnier2016a}, \citet{Waters2016}, \citet{Magnier2016b}, \citet{Magnier2016c} and \citet{Flewelling2016} describe the instrument, survey, and data analyses.  The principal science product of the PS1 survey is the catalog accessible through the CasJobs interface. 
    
We query and return the mean PSF magnitudes from the $grizy$ filters ({\tt MeanPSFMag}) which are in the AB system for our 463 VH$z$Q sample. Details of our SQL and links to the main tables are given in Appendix~\ref{sec:PS1_SQL}.
    
    \iffalse
    \subsubsection{DECam} 
    The Dark Energy Camera \citep[DECam][]{Flaugher2015} is 
    is a wide-field imager with a 2.2 degree diameter field of view 
    mounted at the prime focus of the Victor M. Blanco 4 m telescope 
    at the Cerro Tololo International Observatory. We use data 
    from both the Dark Energy Survey \citep[DES; ][]{DES2016} and 
    the DESI Legacy Imaging Survey \citep[DECaLS; ][]{Dey2018}. 
    \fi
    

\subsection{Near-infrared photometry}~\label{sec:NIR_data} 
The near-infrared data in this paper comes from the Wide Field Astronomy Unit's
(WFAU) Science Archives for UKIRT-WFCAM, the WFCAM Science Archive
\citep[WSA][]{WSA} and VISTA-VIRCAM, the VISTA Science Archive
\citep[VSA][]{VSA}. These archives were developed for the VISTA Data Flow System
\citep[VDFS][]{VDFS}.

We access both the WSA and the VSA and include all non-proprietary WFCAM data, which covers all
public surveys and PI projects from Semester 05A to 1st January 2017, and all
non-proprietary VISTA data, which covers all public surveys and PI projects from
science verification (20091015) to 1st April 2016. 

The data was processed using a matched-aperture photometry method
where flux is measured at the spectroscopic position of the quasar,
without necessarily knowing if there is a formal detection in the NIR
photometry beforehand.  Full details of the matched-aperture pipeline
will appear in a forthcoming paper, Cross et al.  2018, in prep, and
has also been discussed in \citet{Cross2013}.  

We query the WSA and VSA performing matched-aperature photometry at
the positions of our 463 VH$z$Qs. This database is world-readable and we give the full receipe
and relevant SQL queries for accessing both databases in Appendix~\ref{sec:SQL}.
   
    \subsubsection{Averaging matched photometry}
    The photometry in a single epoch image often has low
    signal-to-noise.  The advantage of matched aperture photometry on QSOs
    is that co-adding is relatively simple if each epoch is taken in the
    same aperture and the aperture photometry has been corrected to
    total. Indeed, the standard aperture corrections work well for point
    sources. Coadding using the matched-aperature photometry, where the
    individual epochs are taken from multiple projects with different
    pointings and orientations, should help with issues such as scattered
    light, pixel distortion and aperture corrections.
    
    We average the aperture corrected calibrated fluxes (e.g. {\bf aperJky3}), and
    then convert to magnitudes. Since we do not have a deep image for
    each set of averages, we cannot calculate non-aperture corrected values, so the
    photometry is only appropriate for point-sources. 

    \begin{equation}
      \bar{F} = \frac{\sum_i^N (w_i\,F_i)}{\sum_i^N w_i}  
      \label{eq:avg}
    \end{equation}
    where $F_i$ is the $i^{th}$ epoch measurement of a parameter to be
    averaged such as the aperture corrected calibrated flux in a $1\arcsec$ aperture
    ({\bf aperJky3}) and $\bar{F}$ is the weighted mean average of this parameter.
    The weight for each epoch $w_i=1/(\sigma_{F})^2$ if the epoch is included and 
    $w_i=0$ if an epoch is excluded for quality control purposes. 
    
    We calculate a set of averaged catalogues, for each pointing and filter, based
    on the requirements in \verb+RequiredMapAverages+, in these cases over time
    spans of 7, 14, 30, 91, 183 days, 365 days, 730 days, over 10 epochs and
    over all epochs. {\bf NJC: Should we add this table to suplemental online
    only material?} The averaging process starts at the first epoch and works on.

    We detect 304 unique quasars in the WFCAM WSA database, 203 quasars are 
    detected in the VISTA VSA database with 114 objects in common with both
    WFCAM and VISTA data.  We give the necessary SQL queries syntax in Appendix~\ref{sec:SQL}.


    \subsubsection{Salient details of WSA and VSA}. 
    %% N.B.!!!    WFAU archives more than just the NIR Surveys -- it archives *all* the WFCAM and VISTA programmes!!
    In this study, we are not just quering the WSA or VSA data tables. 
    We are taking a list of objects (positions) are performing matched aperature (``forced'') photometry on the NIR imaging data. 
    As such, we generate a set of tables that are different in subtle ways to the regular ``Detection'' tables. 

    The two most important tables for our needs are the {\tt [w/v]serv1000MapRemeasurement} and {\tt [w/v]serv1000MapRemeasAver}.
     
    Full documentation can be found at the \href{http://wsa.roe.ac.uk/www/wsa_browser.html}{WSA Schema Browser} 
and the \href{http://horus.roe.ac.uk/vsa/www/vsa_browser.html}{VSA Schema Browser}. 


\subsection{MIR data}
Since we are concerned here with the very large area ($\gg1000$ deg$^2$) surveys, we leave exploration of the VH$z$Q population in e.g. the large Spitzer areal surveys such as the Spitzer IRAC Equatorial Survey \citep[SpIES; ][]{Timlin2016}, the Spitzer-HETDEX Exploratory Large-area Survey \citep[SHELA; ][]{Papovich2016} and the Spitzer-SPT Deep Field \citep[SSDF; ][]{Ashby2013}, to future investigation. This will also include a detailed study of MIR spectra e.g. \citet{Lambrides2018}. 

The MIR data for this study comes from the Wide-field Infrared Survey Explorer (WISE) mission, and we utlize data from the WISE cryogenic and NEOWISE (Mainzer et al. 2011 ApJ, 731, 53) post-cryogenic survey phases. 

We use data from the the beginning of the WISE mission \citep[2010 January; ][]{Wright2010} through the fourth-year of NEOWISE-R operations \citep[2017 December;]{Mainzer2011}.  More specifically, we utilise the recently released ``unWISE Catalog'' \citet{MeisnerSchlafly2019}. The unWISE effort\footnote{http://unwise.me/} is the unblurred coadds of the WISE imaging using the AllWISE and NEOWISE-R stacked data \citep{Lang2014, Meisner2018a, Meisner2018b}. {\bf Few more words here on the unWISE catalog}. 

All fluxes in the unWISE catalog are reported there are in ``Vega nanoMaggies'', with the Vega magnitude of a source is given by $m_{\rm Vega} = 22.5−2.5\log(f)$, where $f$ is the source flux. The absolute calibration for unWISE is ultimately inherited from AllWISE through the calibration of \citet{Meisner2017}. This inheritance depends on details of the PSF normalization at large radii, which is uncertain. Subtracting 4 mmag from the unWISE W1, and 32 mmag from unWISE W2 fluxes improves the agreement between unWISE and AllWISE fluxes.

Thus to convert unWISE Vega magnitudes onto the AB system, we have    
\begin{eqnarray*}
        {\rm W1}_{\rm AB, unWISE}  & = &   22.5 - 2.5 \log(f_{\rm W1}) - 0.004 + 2.699 \\
        {\rm W2}_{\rm AB, unWISE}  &  = &  22.5 - 2.5 \log(f_{\rm W2}) - 0.032 + 3.339 
\end{eqnarray*}
%Henceforth, when talking about we will simply refer to ``W1'' and ``W2'' as shorthand for ${\rm W1}_{\rm AB, unWISE}$ and {\rm W2}_{\rm AB, unWISE}
those two numbers are how far our flux measurements differed on average from the AllWISE flux estimates at high Galactic latitudes.  These tweaks will move you closer to the AllWISE absolute calibration, which we haven't tried to improve upon.
%We note though that 
%Especially in W1, though, this correction is uncertain, because the average offset between AllWISE and unWISE depends on magnitude, and we don't really know
%- what the source of the trend is,
%- which of unWISE and AllWISE is correct, or
%- where the AllWISE tie to AB/Vega was done (though presumably it was done on the bright end?).

For the our MIR variability investigations, we do not use the unWISE coadds, but instead use 
the \href{http://wise2.ipac.caltech.edu/docs/release/allwise/}{AllWISE} catalogue and the \href{http://wise2.ipac.caltech.edu/docs/release/neowise/neowise_2018_release_intro.html}{NEOWISE 2018 Data Release}. NEOWISE 2018 makes available the 3.4 and 4.6 μm (W1 and W2) single-exposure images and extracted source information that was acquired between 2016 December 13 and 2017 December 13 UTC, which was the fourth year of survey operations of the Near-Earth Object Wide-field Infrared Survey Explorer Reactivation Mission (NEOWISE; Mainzer et al. 2014, ApJ, 792, 30). The fourth year NEOWISE data products are concatenated with those from the first three years into a single archive.

The WISE scan pattern leads to coverage of the full-sky approximately once every six months (a``sky pass''), but the satellite was placed in hibernation in 2011 February and then reactivated in 2013 October. Hence, our light curves have a cadence of 6 months with a 32 month sampling gap.



%%%%%%%%%%%%%%%%%%%%%%%%%%%%%%%%%%%%%%%%%%%%%%%%%%%%%%%%%%%%%%%%%%
%%%%%%%%%%%%%%%%%%%%%%%%%%%%%%%%%%%%%%%%%%%%%%%%%%%%%%%%%%%%%%%%%%
%%
%%  S E C T I O  N   3         S E C T I O  N   3           S E C T I O  N   3       S E C T I O  N   3
%%  S E C T I O  N   3         S E C T I O  N   3           S E C T I O  N   3       S E C T I O  N   3
%%  S E C T I O  N   3         S E C T I O  N   3           S E C T I O  N   3       S E C T I O  N   3
%%
%%%%%%%%%%%%%%%%%%%%%%%%%%%%%%%%%%%%%%%%%%%%%%%%%%%%%%%%%%%%%%%%%%
%%%%%%%%%%%%%%%%%%%%%%%%%%%%%%%%%%%%%%%%%%%%%%%%%%%%%%%%%%%%%%%%%%
\section{Results}
Having collated the sample of 463 VH$z$Qs, and obtained their optical,
near- and mid-infrared photometry we report here the various
photometric properites of the quasars.

First, we will concentrate on detection rate in the infrared, go on to
report on the color-redshift and color-color properties of our sample
and then report on how the current sample populates the
luminosity-redshift $Lz$-plane.

    %\subsection{Detection Rates in the optical}

    \subsection{Detection Rates in the NIR}
    Table~\ref{tab:nir_detection} gives the detection rates for the 
    VH$z$Qs in the NIR $YJHK/K_{s}$-bands. 
    The first thing to note is that the coverage of the NIR surveys 
    for example from the UKIDSS LAS and VISTA VHS, does
    not overlap the full area for where the VH$z$Qs are detected. 

    \begin{table}
          \centering

      \begin{tabular}{l r l}
        \hline  \hline
        Selection   & number detected (\%) \\
        \hline  
        Any band ($ZYJHK/K_{s}$   &  394  (92.9) \\
        $Z$-band    &  72  (17.0) \\
        $Y$-band    &  249  (58.7) \\
        $J$-band    &  391  (92.2) \\
        $H$-band    &  258  (60.8) \\
        $K$ or $Ks$-band    &  297  ( ) \\
        \hline  \hline
      \end{tabular}
      \caption{Detection rate of VH$z$Qs in the near-infrared.}
      \label{tab:nir_detection}
    \end{table}

        \subsubsection{Comparing WFCAM and VISTA}
        There are 114 overlapping QSOs between WFCAM and VISTA.  Using the
        {\tt VegaToAB} value\footnote{What is this exactly??} to put these
        objects on the same AB system, and for each object compared the two
        measurements. First, the calculated weighted average (calibrated flux)
        in each filter of both and calculated the ratio and difference between
        each measurement and the average.  Then for each filter we calculated
        the weighted average of the differences (in mag) for each instrument
        to see if there were significant offsets. The results are given in
        Table~\ref{tab:WFCAM_vs_VISTA}.  The only filter with a significant
        offset is the $Y$-band. All of the VISTA averages are negative and all
        of the WFCAM ones are positive.  The $Ks$ versus $K$ band may be
        slightly dodgy, given the different shapes of the filters.
        \begin{table}
          \centering
          \begin{tabular}{l r r}
            \hline  \hline
            abs(VIRCAM & \multirow{2}{*}{millimags} &  no. of  \\
            -  WFCAM)      &                                        &  objects \\
            \hline
            $Z$                 &  19.3 	& 2 \\
            $Y$                 &  66.2 	& 48 \\
            $J$                  &    3.2 	& 105 \\
            $H$                 &  19.3     &  89 \\
            $K_{\rm s}$/$K$ &  12.7     & 93 \\
            \hline  \hline
          \end{tabular}
          \caption{Comparing the magnitudes in different WFCAM/UKIRT and VIRCAM/VISTA near-infrared bands.}
          \label{tab:WFCAM_vs_VISTA}
        \end{table}
        
    \subsection{Detection Rates in the MIR}
    Unlike the NIR coverage, the WISE
    satellite and mission performed an all-sky survey, so the location of
    evey VH$z$Q in our dataset is covered. However, the depth of the WISE
    ALLWISE survey depends heavily on sky location, with locations near
    the Ecliptic Poles having the highest number of exposures.
    
    Before reporting on the detection rates, we investigate this
    effect. Figure~\ref{fig:WISEmag_vs_coverage} shows the WISE magnitude
    versus signal-to-noise, colour coded by {\tt w$x$cov} the mean
    coverage depth, in each corresponding band. In the two shorter bands
    W1/2 we see the clear and expected trend for brighter objects to have
    larger SNR, and also for the higher signal to noise for objects with
    more exposures at a given magnitude. The behaviour for the W3/4 bands
    is different, with two populations clearly evident in W3 and although
    a bit more mixed, also in W4. With the suggested split at SNR$>2$, and
    no obvious R.A./Declination dependence seen, this behaviour is
    explained by the fact that there are non-detectiopns in W3/4 for
    objects (with high W1/2 SNR) that are reported in the ALLWISE
    catalogue.
    
    For the 278 VH$z$Q with coverage detections, the mean number of
    exposures for the W1/2 bands is 32.0 and 31.5, respectively, with a
    minimum number of exposures 17 and 12, and the maximum number or
    exposures being 114 (for both bands).  For the W3/4 filters, the
    corresponding mean, minimum and maximum exposure are 17.4 and 17.5,
    5.8 and 6.8 and 69 (for both bands). These values are direclty from
    the {\tt w$x$cov} enteries in the 
    \href {http://wise2.ipac.caltech.edu/docs/release/allwise/expsup/sec2_1a.html#w1cov}{WISE ALLWISE catalogue}.

    \begin{figure}
      %% trim=l b r t
      \includegraphics[width=8.6cm, clip,trim=6mm 6mm 0mm 6mm]
      {/cos_pc19a_npr/programs/quasars/highest_z/detections/WISEmag_vs_coverage_2x2_v1.pdf}
      \centering
      \vspace{-14pt}
      \caption[]{WISE W1/2/3/4 magnitude against signal-to-noise, 
        colour coded by w$x$cov the mean coverage depth, in each corresponding band.
      }
      \label{fig:WISEmag_vs_coverage}
    \end{figure}
    
    Table~\ref{tab:mir_detection} gives the detection rates for the
    VH$z$Qs in the MIR WISE W1-4 bands. 
    \begin{table}
      \begin{tabular}{l r l}
        \hline  \hline
        Selection   & number detected (\%) \\
        \hline  
        W1 SNR $> 2.0$                                             &  275  (64.9) \\
        W2 SNR $> 2.0$                                            &   255 (60.1) \\
        W1 $\land$ W2 SNR $> 2.0$                         &  \\
        W3 SNR $> 2.0$                                            &  99    (23.3) \\
        W4 SNR $> 2.0$                                            &  29    (6.8) \\
        Any W1/2/3/4 SNR $>2.0$                           & \\
        W1/2 SNR $< 2.0$ $\land$ W3 SNR $>2.0$ & \\
        \hline  \hline
      \end{tabular}
      \caption{}
      \label{tab:mir_detection}
    \end{table}

    \begin{table}
      \begin{tabular}{l r l}
        \hline  \hline
         \multirow{2}{*}{Selection}   & number detected \\ 
                                                   & (\% of full specrta) \\
        \hline  
        From ``Source'', ``Rejects'',                    & 245, 40  (67.2) \\
        W1 SNR $> 2.0$                                       &  279  (65.8) \\
        W2 SNR $> 2.0$                                       &  258 (60.8) \\
        W1 $\land$ W2 SNR $> 2.0$                    & 253   (59.7)  \\
        W3 SNR $> 2.0$                                     &  97    (22.9) \\
        W4 SNR $> 2.0$                                     &  33    (7.8) \\
%        Any W1/2/3/4 SNR $>2.0$                            & \\
        W1/2 SNR $< 2.0$ $\land$ W3 SNR $>2.0$ &  3 (0.7)\\
        \hline  \hline
      \end{tabular}
      \caption{Data from the AllWISE Source Catalog and AllWISE Reject Table, from the 
\href{https://irsa.ipac.caltech.edu/cgi-bin/Gator/nph-scan?submit=Select&projshort=WISE} {{\tt NASA/IPAC  Infrared Science Archive}}}
      \label{tab:mir_detection}
    \end{table}
    
    \begin{figure}
      %% trim=l b r t
      \includegraphics[width=8.6cm, clip,trim=2mm 0mm 2mm 0mm]
      {/cos_pc19a_npr/programs/quasars/highest_z/detections/WISEsnrW1W2W3W4_2by3_v1.pdf}
      \centering
      \vspace{-14pt}
      \caption[]{WISE signal-to-noise measures for the four bands, as well
        as for (W1-W2) colour.  The points are colour coded by redshift.}
      \label{fig:WISEmag_vs_coverage}
    \end{figure}

    \citet{Blain2013} 

    Recently, \citet{Assef2018} released two large catalogues of AGN 
    candidates identified across 30,000 deg$^2$ of extragalactic sky 
    from the WISE AllWISE Data Release. The ``R90'' catalogue, is 
    contains 4.5M AGN candidates at 90\% reliability (and $\approx$150 
    AGN candidates per deg$^2$) while the ``C75'' catalog 
    consists of 20.9M AGN candidates at 75\% completeness (and 
    ($\approx$700 AGN candidates per deg$^2$).  Crossmatching 
    out catalogue of 463 VH$z$Qs with these catalogues, produces 
    42 matches with	the R90 sample and 98 matches with the C75 sample. 
    Both catalogues unsurprisingly match to the ultraluminous quasar 
    SDSS J0100+2802 \citep{Wu2015} while the C75, but not the R90 catalogue 
    mathes to ULAS J1120+0641 \citep{Mortlock2011}. Neither catalogue 
    matches J1342+0928 \citep{Banados2018}. 

    %\subsubsection{
    Very High-$z$ Quasars Detected in WISE W3 and W4.


    \begin{figure}
      \centering
      \includegraphics[width=8.5cm]
      {/cos_pc19a_npr/programs/quasars/highest_z/LightCurves/MIR_LCs/NEOWISER_LC_histogramlog_20180827.png}
      \vspace{-16pt}
      \caption[]
      {Histogram showing the number of NEOWISE-R epochs and detections there are for each 
        VH$z$Q.} 
      \label{fig:MIR_LC_epochs}
    \end{figure}
    
\subsection{Variability}
VH$z$Qs, if accreting at, or above the Eddington Limit, might well have have large values of changing mass accretion rate, $\ddot{m_{\rm accr}}$. A consequence of this would be that these quasar exhibit signs of variability, most likely showing up in their UV/optical rest-frame spectra. We look for evidence of this variability signature in the NIR and MIR light-curves of the VH$z$Qs. As a guide, \civ enters the $Y$-band at redshift $z$=5.32 and exits at $z$=5.99, and enters the $J$-band at redshift $z=6.55$ and exits at $z$=7.57. \mgii enters the $H$-band at redshift $z=4.33$ and exits at $z=5.37$ and enters the $K$-band at redshift $z=6.25$ and exits at $7.50$.

Using the extended datasets described in Section~\ref{sec:NIR_data} and~\ref{sec:NIR_SQL}, we 

\begin{figure}
  \includegraphics[width=8.5cm]
  {/cos_pc19a_npr/programs/quasars/highest_z/LightCurves/MIR_LCs/three_MIR_LC_egs_20180827.pdf}
  \centering
  \caption[]
  {Here we show the MIR NEOWISE-R for J0100+2802 \citep{Wu2015}, J0224-4711 and  J1626+2751. 
    Red points are the W1 band; cyan points the W2 band.} 
  \label{fig::MIR_LC_3egs}
\end{figure}

Figure~\ref{fig:MIR_LC_epochs} gives the number of NEOWISE-R epochs and detections there are for each VH$z$Q, while 
Figure~\ref{fig:MIR_LC_3egs} presents three examples of the MIR lightcurves and
associated colour changes. Here we show J0100+2802 \citep{Wu2015}, J0224-4711 and  J1626+2751. 
{\bf NJC: What about NIR light-curves / combined light-curves}


\subsection{Colours}
Currently, very high-redshift quasars are identified by their morphology, flux and colours in optical and infrared imaging data \citet{Fan1999, Mortlock2012} Quasars are generally selected to be point sources, but  be outliers from the stellar locus in colour space. For VH$z$Qs, the main technique is to look for objects with extreme optical-to-near-infrared colours The lack of proper motion can also help identified quasars \citep[e.g.][]{Lang2009}. 

Figure~\ref{fig:Opt_colourredshift} presents the optical
colour-redshift trends for Late Type M/L/T dwarfs and the VH$z$Qs.
\begin{figure*}
   \includegraphics[width=18.0cm]
   {/cos_pc19a_npr/programs/quasars/highest_z/color_redshift/SpecType_vs_Optcolors_20180704.pdf}
   \centering
   \caption[]
   {Optical colour vs. spectral type and redshift for Late Type M/L/T dwarfs and the VH$z$Qs.
     The stars are M, L, and T dwarfs from the \citet{Best2018} PS1-detected catalog.  
   {\it N.B. Trying to look as good as Fig.~5 from Best et al. (2018). How does one get 
bigger gaps between subplots??}}
   \label{fig:Opt_colourredshift}
 \end{figure*}

\begin{figure*}
   \includegraphics[width=18.0cm]
   {/cos_pc19a_npr/programs/quasars/highest_z/color_redshift/SpecType_vs_NIRcolors_20180704.pdf}
  \centering
   \caption[]
   {Infrared colour-spectral type and redshift plots for Late Type M/L/T dwarfs and the VH$z$Qs.
     {\it NB} I'm really not sure how Best et al. actually get their stellar sequence so clean. 
There are two types of spectral classification,  but restricting it to just SpT\_optn  or SpT\_nir removes
the blue or red end respectively. Hmmm....}
   \label{fig:SpecType_vs_NIRcolors}
 \end{figure*}

\begin{figure*}
   \includegraphics[width=18.0cm]
   {/cos_pc19a_npr/programs/quasars/highest_z/color_redshift/SpecType_vs_W1W2_W2W3colors_20180407.pdf}
  \centering
   \caption[]
   {Infrared colour-spectral type and redshift plots for Late Type M/L/T dwarfs and the VH$z$Qs.
}
   \label{fig:SpecType_vs_W1W2_W2W3colors}
 \end{figure*}



\subsection{L-$z$ Plane}
Having obtained an as-near-to-homogenous set of photometry as we can, 
we are now in a position to calculate the Absolute Magnitudes of the VH$z$Q 
sample and in particulare the absolute magnitude at rest-frame 1450\AA\ , $M_{1450}$, 
which is a key physical quantity and goes directly towards the quasar luminosity 
function and thus the reionization of hydrogen calculation. 

We calculate the Distance Modulus in the normal fashion, 
\begin{equation}
m_{1450} - M_{1450} = 5 \log \left (    \frac{ D_{\rm L}(z)}{\rm Mpc}  \right )  + 25 + K_{\rm corr}(X,z)
\end{equation}
where $m_{1450}$ is the apparent magnitude at 1450\AA\ ,  
%$M_{1450}$ is the absolute magnitude at 1450 \AA\ , 
$D_{\rm L}(z)$ is the luminosity distance and 
$K_{\rm corr}(X,z)$ is the $K$-correction which corrects for the effects of redshifting of the bandpass and the spectrum. 

The $m_{1450}$ apparent magnitude is derived from the $z-$, $y/Y-$ or $J-$band photometery.

The Pan-STARSS1 $z_{\rm PS1}$ and $y_{\rm PS1}$-bands approximatley
sample the redshift ranges $4.53\leq z \leq 5.45$ and $5.28\leq z \leq 6.47$, respectively 
for 1450\AA'\ emission, while the VIRCAM $Y_{\rm VIRCAM}-$ and $J_{\rm VIRCAM}$-bands 
cover $5.50\leq z \leq 6.57$ and $7.06\leq z \leq 8.16$. 

\citet{Ross2013} has a detailed discussion of the $K$-correction (see that papers' Appendix B). 
The key result in that paper is, if quasars are described as having a power-law slope, 
$\alpha^{\nu}$ in spectral flux density, i.e., $f_\nu(\nu) \propto \nu^{\alpha_{\nu}}$ (as is conventional) 
then 
\begin{equation}
K_{\rm corr}(z) = -2.5 (1 + \alpha_{\nu}) \log[1 + z].
\end{equation}
Here the $[-2.5 \log(1 + z)]$ term corrects for the effective narrowing of the filter width with redshift, (the ``bandpass correction'') and the $[-2.5 \alpha^{\nu} \log(1 + z)]$ term takes into account the spectral index correction. The bandpass correction is approximately $\approx -1.945$ at redshift $z=5$ decreasing to $-2.32$ at redshift $z=7.50$. 


%At $z=5.00$, the rest-frame 1450\AA\ emission is redshifted to 8700\AA\ observed, i.e., in the $z$-band, while at $z=6.00$, $z=5.00$


\begin{figure*}
  \includegraphics[width=18.0cm]
  {/cos_pc19a_npr/programs/quasars/highest_z/Lz/VHzQ_Lz_20180702.pdf}
  \centering
  \caption[]
  {The spectral bands used by different survey telescopes and that are relevant here.}
  \label{fig:Lz}
\end{figure*}


\iffalse
\subsection{SEDs and Dust properties of the VH$z$Qs}
There are a range of IR SEDs e.g. \citet{Mullaney2013} etc. etc. etc. 
However, they are, for our purposes all roughly the same. 

\begin{figure*}
  \includegraphics[width=18.0cm]
  {/cos_pc19a_npr/programs/quasars/highest_z/SEDs/RestWavelength_flux_20180702.pdf}
  \centering
  \caption[]
  {The rest-frame properties of the VH$z$Qs. }
  \label{fig:RestWavelength_SEDs}
\end{figure*}
\fi


%%%%%%%%%%%%%%%%%%%%%%%%%%%%%%%%%%%%%%%%%%%%%%%%%%%%%%%%%%%%%%%%%%
%%%%%%%%%%%%%%%%%%%%%%%%%%%%%%%%%%%%%%%%%%%%%%%%%%%%%%%%%%%%%%%%%%
%%
%%  S E C T I O  N   7         S E C T I O  N   7           S E C T I O  N   7       S E C T I O  N   7
%%  S E C T I O  N   7         S E C T I O  N   7           S E C T I O  N   7       S E C T I O  N   7
%%  S E C T I O  N   7         S E C T I O  N   7           S E C T I O  N   7       S E C T I O  N   7
%%
%%%%%%%%%%%%%%%%%%%%%%%%%%%%%%%%%%%%%%%%%%%%%%%%%%%%%%%%%%%%%%%%%%
%%%%%%%%%%%%%%%%%%%%%%%%%%%%%%%%%%%%%%%%%%%%%%%%%%%%%%%%%%%%%%%%%%
\section{Discussion and Conclusions}
\label{sec:conclusions}
In this study, we have, for the first time, ompiled the list of all
$z>5$ spectroscopically confirmed quasars. We have assemble the NIR
($y/Y, J, H, K/K_{s}$) and MIR (WISE W1/2/3/4) photometry for these
objects, given their detection rates and SEDs. We find that: 

%%
We can gain a good appreciation for what these missions will discover
by collating the datasets we currently have. 

\begin{itemize}
    \item Lorem ipsum dolor sit amet, consectetur adipiscing
      elit. Aliquam porta sodales est, vel cursus risus porta non. Vivamus
      vel pretium velit. Sed fringilla suscipit felis, nec iaculis lacus
      convallis ac. 
    \item Fusce pellentesque condimentum dolor, quis vehicula
      tortor hendrerit sed. Class aptent taciti sociosqu ad litora torquent
      per conubia nostra, per inceptos himenaeos. Etiam interdum tristique
      diam eu blandit. Donec in lacinia libero.
    \item Sed elit massa, eleifend non sodales a, commodo ut felis. Sed id
      pretium felis. Vestibulum et turpis vitae quam aliquam convallis. Sed
      id ligula eu nulla ultrices tempus. Phasellus mattis erat quis metus
      dignissim malesuada. Nulla tincidunt quam volutpat nibh facilisis
      euismod. Cras vel auctor neque. Nam quis diam risus.
\end{itemize}
Nunc lacus nibh, convallis ac lobortis ut, tempus ac lectus. Maecenas
eu elit massa. Nulla vel lacus lorem. Proin et lobortis
tortor. Phasellus ultrices nisl non enim porttitor dictum. Curabitur
nec nunc ac nibh ornare elementum. Nunc ultrices hendrerit
ultricies. Aliquam dapibus semper est et gravida. Etiam cursus, massa
eget tempor elementum, lectus urna feugiat nisi, eget sagittis.

\subsection*{Author Contributions}   
N.P.R. initiated the project, compiled the list of $z>5.00$ quasars, wrote most of the analysis code, developed the the plotting scripts, and developed and wrote the initial and subsequent drafts of the manuscript.
%%
N.J.G.C. supplied the critical near-infrared expertise and database for which the bulk of the project relies. N.J.G.C. also contributed directly to the writing of the manuscript.
%%



\subsection*{Availability of Data and computer analysis codes} 
All materials, databases, data tables and code are fully available at: 
\href{https://github.com/d80b2t/VHzQ}{\tt https://github.com/d80b2t/VHzQ}


\section*{Acknowledgements}
NPR acknowledges support from the STFC and the Ernest Rutherford Fellowship scheme. 

We thank Mike Read at the ROE WFAU for help with the WFCAM Science Archiv (WSA), and 
also the VISTA Science Archive (VSA). We thank Bernie Shiao at STScI for help with the Pan-STARRS1 DR1 CasJobs interface. 

This paper heavily used \href{http://www.star.bris.ac.uk/~mbt/topcat/}{TOPCAT} (v4.4)
\citep[][]{Taylor2005, Taylor2011}.
%%
This research made use of \href{http://www.astropy.org}{\tt Astropy}, 
a community-developed core Python package for Astronomy 
\citep{AstropyCollaboration2013, AstropyCollaboration2018}. 

The Pan-STARRS1 Surveys (PS1) and the PS1 public science archive have
been made possible through contributions by the Institute for
Astronomy, the University of Hawaii, the Pan-STARRS Project Office,
the Max-Planck Society and its participating institutes, the Max
Planck Institute for Astronomy, Heidelberg and the Max Planck
Institute for Extraterrestrial Physics, Garching, The Johns Hopkins
University, Durham University, the University of Edinburgh, the
Queen's University Belfast, the Harvard-Smithsonian Center for
Astrophysics, the Las Cumbres Observatory Global Telescope Network
Incorporated, the National Central University of Taiwan, the Space
Telescope Science Institute, the National Aeronautics and Space
Administration under Grant No. NNX08AR22G issued through the Planetary
Science Division of the NASA Science Mission Directorate, the National
Science Foundation Grant No. AST-1238877, the University of Maryland,
Eotvos Lorand University (ELTE), the Los Alamos National Laboratory,
and the Gordon and Betty Moore Foundation.

This project used data obtained with the Dark Energy Camera (DECam)
and the NOAO Data Lab, The Data Lab is operated by the National
Optical Astronomy Observatory, the national center for ground-based
nighttime astronomy in the United States operated by the Association
of Universities for Research in Astronomy (AURA) under cooperative
agreement with the National Science Foundation.

This publication makes use of data products from the Wide-field
Infrared Survey Explorer, which is a joint project of the University
of California, Los Angeles, and the Jet Propulsion
Laboratory/California Institute of Technology, and NEOWISE, which is a
project of the Jet Propulsion Laboratory/California Institute of
Technology. WISE and NEOWISE are funded by the National Aeronautics
and Space Administration.

CasJobs was originally developed by the Johns Hopkins University/
Sloan Digital Sky Survey (JHU/SDSS) team. With their permission, MAST
used version 3.5.16 to construct CasJobs-based tools for GALEX,
Kepler, the Hubble Source Catalog, and PanSTARRS.

This research has made use of the SVO Filter Profile Service
(http://svo2.cab.inta-csic.es/theory/fps/) supported from the Spanish
MINECO through grant AyA2014-55216 
%%
The SVO Filter Profile Service\footnote{Rodrigo, C., Solano, E., Bayo, A. http://ivoa.net/documents/Notes/SVOFPS/index.html}
describes the Spanish VO Filter Profile Service. 
The Filter Profile Service Access Protocol. Rodrigo, C., Solano, E. http://ivoa.net/documents/Notes/SVOFPSDAL/index.html

\newpage

\appendix
%\section{Filter Curves}\label{sec:filters} 
%From the SVO Filter Profile Service\footnote{http://svo2.cab.inta-csic.es/svo/theory/fps/}.

\section{A. Photometric Bands and Conversions}
    Due to the differing normalizations between the
    SDSS and  UKIDSS photometric systems, certain corrections are required.  To present
    our data in the  purest sense, all the NIR magnitudes from UKIDSS
    (originally AB magnitudes)  were corrected to Vega magnitudes as
    suggested in \citet{Hewett2006}.
    
    Although ULAS magnitudes are reported in terms of Vega and SDSS
    magnitudes are reported in AB terms for the most part whenever an
    optical-NIR color was calculated both magnitudes were left in their
    default term.
    
\begin{table*}
  \begin{center}
   \caption{Adapted from Table 9 of \citet{Peth2011}. 
CTIO/DECam, PanSTARRS/PS1, LSST
%References: www.cfht.hawaii.edu/Instruments/Filters/wircam.html
Filter only values. 
All wavelengths in ${\buildrel _{\circ} \over {\mathrm{A}}}$. 
%%
%%
From \citet{GonzalezFernandez2018} 
$Z_{\rm AB}   -  Z_{\rm Vega}  = 0.502$;  
$Y_{\rm AB}  -  Y_{\rm Vega}    = 0.600 $;
$J_{\rm AB}   -  J_{\rm Vega}    = 0.916  $;
$H_{\rm AB}  -  H_{\rm Vega}    = 1.366 $;
$Ks_{\rm AB}  -  Ks_{\rm Vega}  = 1.827 $;
%%
and the CASU Vega to AB conversions v1.3:: 
	Z,Y,J,H,Ks were: 0.524, 0.618, 0.937, 1.384, 1.839. 
%%
So, $\Delta$(vs. Gonzalez-Fernandez)::
	(11.2,     1.1,    5.4,     1.6,    0.1) millimags. 
$\Delta$(vsCASU v1.3)::
	(-10.8, -16.9, -15.6,  -16.4, -11.9) millimags. 
}
    \setlength{\tabcolsep}{4pt}
     \begin{tabular}{l r r r  c l l}
      %% https://www.gemini.edu/sciops/instruments/magnitudes-and-fluxes
      %% http://wise2.ipac.caltech.edu/docs/release/allsky/expsup/sec4_4h.html#conv2ab
      \hline
      \hline
      Band & $\lambda_{\rm eff}  $ 
              &  $\lambda_{\rm min} $ 
              & $\lambda_{\rm max} $ 
              & W$_{\rm eff}$
              & \multicolumn{2}{c}{AB - Vega  Transformations} \\
      \hline
      % {\it u} & 3551  &  3005 &  4000 &  581 & $u$ = $u_{AB}$ - 0.927 \\
      % {\it u_{\rm LSST}} &	3733 &	3182 &	4082 &	?? \\
       $g_{\rm HSC}$    &  	4633   &     3940     &   5546	&  1460       &    $g_{\rm HSC}$         &$  = g_{\rm AB} + 0.097 $ \\
      $g_{\rm LSST}$      &     4730     &	  3877    &	   5665   &  1333   &  $g_{\rm LSST}$       &$ = g_{\rm AB} +  0.083 $ \\   %% from SVO 3921.4 
      $g_{\rm DECam}$  &      4734     &   3939    &    5528   &   1133        &  $g_{\rm DECam} $    &$  = g_{\rm AB} + 0.083 $ \\	     
       $g_{\rm PS1}$        &    4776    &    3943    &    5593   &   1167        &  $g_{\rm PS1}$         &$  = g_{\rm AB} + 0.080 $ \\   %% from SVO; e.g.   (-2.5)*(log10(3909.1/3631.0)) 
      &&&&&&\\
      $r_{\rm HSC}$         &    6104     &   5325    &   7071	&   1503       & $r_{\rm HSC}   $       &$     = r_{\rm AB} - 0.151 $ \\
      $r_{\rm PS1}$         &    6130    & 	  5386    &    7036   &   1318       &  $r_{\rm PS1}   $       &$     = r_{\rm AB} - 0.153 $ \\ %% from SVO; e.g.   (-2.5)*(log10(3151.4/3631.0)) 
      $r_{\rm LSST}$       &     6139    &	  5375    &    7055   &   1338      &  $r_{\rm LSST}   $       &$    = r_{\rm AB} - 0.155 $ \\	
      $r_{\rm DECam}$   &      6345    &    5506    &    7238   &   1379      &  $r_{\rm DECam}$       &$   = r_{\rm AB} - 0.192 $ \\	     
      &&&&&&\\
      $i_{\rm PS1}$         &    7485    &     6778    &    8304   &   1243      &  $i_{\rm PS1}    $       &$   = i_{\rm AB} - 0.369 $ \\  %% 2584.6
      $i_{\rm LSST}$       &     7487    &	  6765    &     8325   &   1209      &  $i_{\rm LSST}   $       &$    = i_{\rm AB} - 0.369 $ \\   %% 2583.9
      $i_{\rm HSC}$         &   7633    &     6791    &     8658	&   1483       &  $i_{\rm PS1}    $       &$   = i_{\rm AB} - 0.396 $ \\  %% 2521.6
      $i_{\rm DECam}$   &      7750   &	  6950    &     8646    &   1371      &  $i_{\rm DECam} $     &$  = i_{\rm AB} - 0.415 $        \\	   
      &&&&&&\\
      $z_{\rm PS1}$        &    8658    &	 8028    &      9346   &      966      &  $z_{\rm PS1}   $      &$    = z_{\rm AB} - 0.508 $       \\
      $z_{\rm LSST}$      &     8669   & 	 8035    &      9375   &      994     &   $z_{\rm LSST}  $      &$    = z_{\rm AB} - 0.509 $     \\
      $Z_{\rm VIRCAM}$  &     8762   & 	 8157    &      9400   &      978    &   $Z_{\rm VIRCAM}  $   &$    = Z_{\rm AB} - 0.513 $     \\
      $Z_{\rm WFCAM}$  &     8802   & 	 8129    &      9457   &      926    &   $Z_{\rm WFCAM}  $   &$    = Z_{\rm AB} - 0.514 $     \\
      $z_{\rm HSC}$        &    8915  &   	8280     &      9498	&    793      & 	$Z_{\rm HSC} $         &$    = Z_{\rm AB} - 0.512$     \\
      $z_{\rm DECam}$  &      9216   & 	 8360    &    10166   &   1502      &  $z_{\rm DECam} $     &$   = z_{\rm AB} - 0.521 $ \\
      &&&&&&\\
      $y_{\rm PS1}$       &       9603    &  9100    &    10838  &     615       &  $y_{\rm PS1}    $       &$   = y_{\rm AB} -  0.541 $ \\
      $y_{\rm LSST}$      &       9677   &	 9089    &    10859  &     810         &  $y_{\rm LSST}  $      &$    = y_{\rm AB} - 0.546 $ \\
      $Y_{\rm DECam}$   &      9876   &	  9355    &      10730   &    676      &  $Y_{\rm DECam}  $   &$  =Y_{\rm AB} - 0.570 $ \\
      $Y_{\rm HSC}$       &      9976   &    9000    & 	10931  &   1386    &  $Y_{\rm HSC}  $   &$  =Y_{\rm AB} - 0.580 $ \\
      $Y_{\rm WFCAM}$    &   10305    &   9790      &   10810   &   1020     & $Y_{\rm WFCAM}$     &$ =  Y_{AB}  - 0.617$           \\
      $Y_{\rm VIRCAM}$     &    10184    &   9427      &   10977   &    905        & $Y_{\rm VIRCAM} $     &$ = Y_{AB}  - 0.601 $          \\
      &&&&&&\\
      $J_{\rm 2MASS}$       &   12350   &       10806  & 	14068  &  1624       &  $J_{\rm 2MASS}  $     & $= J_{AB}    - 0.894  $         \\
      $J_{\rm VIRCAM} $     &   12464   &      11427   &    13759   &  1628     &  $J_{\rm VIRCAM}  $     & $= J_{AB}    - 0.921  $         \\
      $J_{\rm WFCAM} $    &    12483   &     11690  &    13280   &   1590      & $J_{\rm WFCAM}$     & $= J_{AB}    - 0.919 $          \\
      &&&&&&\\
      $W_{\rm Wircam}$   &    14514    &    13890   &    15166   &   1020    & $W_{\rm Wircam} $    & $= W_{AB}  -  1.163$           \\
      &&&&&&\\
      $H_{\rm WFCAM}$    &    16313     &    14920  &    17840   &   2920    & $H_{\rm WFCAM} $   & $= H_{AB}  - 1.379$          \\
      $H_{\rm VIRCAM}$      &    16310    &    14604   &    18422   &   2833     & $H_{\rm VIRCAM}$      & $= H_{AB}  - 1.368 $        \\
      $H_{\rm 2MASS}$      & 16620        & 	14787  &   18231   & 2509      & $H_{\rm 2MASS}$      & $= H_{AB}  - 1.374 $        \\
      &&&&&&\\
      $K$s$_{\rm VIRCAM}$     &    21337    &    19333  &    23674   &   3055     & $K$s$_{\rm VIRCAM}$      & $ = K$s$_{AB} - 1.83  $          \\ 
      $K$s$_{\rm 2MASS}$     &   21590    & 	19544  &   23552        &   2619      & $K$s$_{\rm 2MASS}$      & $ = K$s$_{AB} -  1.84  $          \\ 
      $K_{\rm WFCAM}$     &    22010     &    20290 &    23800         &   3510          & $K_{\rm WFCAM}$     & $ = K_{AB} - 1.90  $          \\ 
      &&&&&&\\
      WISE W1               &    33526    &    27541  &    38724   &    6626    & W1                        &   = W1$_{\rm AB} - 2.699$ \\
      WISE W2               &    46028    &    39633  &    53414   &  10423    & W2                        &   = W2$_{\rm AB} - 3.339$ \\
      WISE W3               &  115608    &    74430  &  172613   &  55056    & W3                        &   = W3$_{\rm AB} - 5.174$ \\
      WISE W4               &  228172    &  195201  &  279107   &  41017    & W4                        &   = W4$_{\rm AB} - 6.66$ \\
      \hline
      \hline
      \label{tab:filter_details}
    \end{tabular}
     \end{center}
\end{table*}
https://www.gemini.edu/sciops/instruments/magnitudes-and-fluxes


%% \section{Notes on Individual Objects}
%% \input{/cos_pc19a_npr/data/highest_z_QSOs/notes_on_individual_objects}



\section{PanSTARRS1 SQL queries}\label{sec:PS1_SQL}
The PS1 Casjobs SQL Server is located at
\href{http://mastweb.stsci.edu/ps1casjobs}{mastweb.stsci.edu/ps1casjobs}.
The top level documentation is given
\href{https://outerspace.stsci.edu/display/PANSTARRS/PS1+Source+extraction+and+catalogs}{here}
while the description of tables is given
\href{https://outerspace.stsci.edu/display/PANSTARRS/PS1+Source+extraction+and+catalogs#PS1Sourceextractionandcatalogs}{here}. The
main tables are the
\href{https://outerspace.stsci.edu/display/PANSTARRS/PS1+ObjectThin+table+fields}{{\tt
objectThin}} and
\href{https://outerspace.stsci.edu/display/PANSTARRS/PS1+MeanObject+table+fields}{{\tt
meanObject}} tables.

\onecolumn
\begin{lstlisting}[
language=SQL,
           showspaces=false,
           basicstyle=\ttfamily,
           numbers=left,
           numberstyle=\tiny,
           commentstyle=\color{gray}
        ]
SELECT s.ra, s.decl, 
       o.objID, o.raMean, o.decMean, 
       o.nDetections, o.ng, o.nr, o.ni, o.nz, o.ny, 
       m.gMeanPSFMag, m.gMeanPSFMagErr,  m.gMeanPSFMagStd, 
       m.rMeanPSFMag, m.rMeanPSFMagErr,  m.rMeanPSFMagStd, 
       m.iMeanPSFMag, m.iMeanPSFMagErr,  m.iMeanPSFMagStd, 
       m.zMeanPSFMag, m.zMeanPSFMagErr,  m.zMeanPSFMagStd, 
       m.yMeanPSFMag, m.yMeanPSFMagErr,  m.yMeanPSFMagStd, 
       s.jmag, s.jmag_error, s.hmag, s.hmag_error, s.kmag, s.kmag_error into mydb.MyTable_0 from MyDB.Ldwarfs as s

cross apply fGetNearbyObjEq(s.ra,s.decl,2.0/60.0) nb
inner join ObjectThin o on o.objid=nb.objid and o.nDetections>1 
inner join MeanObject m on o.objid=m.objid  and o.uniquePspsOBid=m.uniquePspsOBid
\end{lstlisting}
\twocolumn


\section{Near-Infrared WFCAM Science Archive SQL queries}\label{sec:SQL}
Here we give the receipe and SQL that returned the near-infrared photometry 
for the VH$z$Qs from the  WFCAM Science Archive. 

The data are on the WFCAM Science Archive: \href{wsa.roe.ac.uk}{\tt wsa.roe.ac.uk}. 
Access the User Login form \href{WFCAM Science Archive}{\tt wsa.roe.ac.uk/login.html} 
with these credentials::
\begin{itemize}
    \item Username: {\tt WSERV1000} 
    \item password: {\tt highzqso} 
    \item community: {\tt nonsurvey}
\end{itemize}
Then going to the
\href{http://wsa.roe.ac.uk:8080/wsa/SQL_form.jsp}{{\tt Free Form SQL
Query}} page the Database release {\tt WSERV1000v20180716} can be
accessed which contains all the data we use here.

We {\it nota bene} a few things. First, the quantity {\tt aperJky3}
and {\tt aperJky3Err} are found in the {\tt wserv1000MapRemeasAver}
and {\tt wserv1000MapRemeasurement}, so care has to be taken to return
unique column names (otherwise e.g.
\href{http://docs.astropy.org/en/stable/io/fits/}{astropy.io.fits}
will crash).  As such, we alias {\tt aver.aperJky3} to {\tt
aperJky3Aver} and likewise for the error quantity. Aliases will be
necessary in some cases anyway, because some queries can be done
sensibly on multiple instances of the same table. Other times, one may
join tables on quantities such as {\tt catalogueID} or {\tt
apertureID}, where you are meaning the same thing, but aliases would
again be sensible.

Second, the {\tt RA} and {\tt DEC} values returned by the WSA are in radians, if
used directly. To return values in degrees, use a selection with an alias, e.g. 
{\tt RA as RADeg} and {\tt DEC as DECDeg}.

\onecolumn
\input{SQL_examples_WSA}
\twocolumn


\section{Near-Infrared VISTA Science Archive SQL queries}\label{sec:SQL}
In a very similar manner to the WSA, we give here the details on how to access
the VISTA Science Archive (VSA)

At the \href{http://horus.roe.ac.uk/vsa/login.html}{VSA Login}, enter 
with these credentials::
\begin{itemize}
    \item Username: {\tt VSERV1000} 
    \item password: {\tt highzqso} 
    \item community: {\tt proprietary}
\end{itemize}
Then head to the \href{http://horus.roe.ac.uk:8080/vdfs/VSQL_form.jsp}{Freeform SQL Query} page where the database release to use is {\tt VSERV1000v20180716}. 

\onecolumn
\input{SQL_examples_VSA}
\twocolumn









%%%%%%%%%%%%%%%%%%%%%%%%%%%%%%%%%%%%%%%%%%%%%%%%%%%%%%%%%%%%%%%%%%%%
%%%%%%%%%%%%%%%%%%%%%%%%%%%%%%%%%%%%%%%%%%%%%%%%%%%%%%%%%%%%%%%%%%%%

%\bibliographystyle{apj}
\bibliographystyle{mn2e}
\bibliography{tester_mnras}

\end{document}
