%%%%%%%%%%%%%%%%%%%%%%%%%%%%%%%%%%%%%%%%%%%%%%%%%%%%%%%%%%%%%%%%%%%%%%%%%%%%%
%%                                                                           
%%    The `compliation' paper for all z>5 spec-z quasars, here called
%%    "Very High Redshift Quasars" (VHzQs) and looking into their IR
%%    properties. Motivations include putting a list of all the VHzQ in one
%%    place, looking for signs of super-critical accretion and having a
%%    super-set of objects and their given parameters ahead of JWST GO Cycle 1.
%%									    
%%%%%%%%%%%%%%%%%%%%%%%%%%%%%%%%%%%%%%%%%%%%%%%%%%%%%%%%%%%%%%%%%%%%%%%%%%%%%
\documentclass[usenatbib]{mnras}

\usepackage{amsmath,, amssymb}
\usepackage{bm}% bold math
\usepackage{blindtext}
\usepackage{cancel, caption, color}
\usepackage{dcolumn}% Align table columns on decimal point
\usepackage{epsfig, epsf}
\usepackage{fancyhdr}
\usepackage[bottom,flushmargin,hang,multiple,para]{footmisc}
\usepackage{graphicx}
\usepackage{lscape, longtable, listings}
\usepackage{multirow}
\usepackage{natbib}
\usepackage{pdflscape}
%\usepackage{subcaption}
\usepackage{subfigure}
\usepackage{textcomp}
\usepackage{hyperref, ifthen}
\usepackage{verbatim}
\usepackage[usenames,dvipsnames]{xcolor}

%% This helps stop the friggin crazy erropr::
%% ! pdfTeX error (ext4): \pdfendlink ended up in different nesting level than \pd
%% fstartlink.
%% \AtBegShi@Output ...ipout \box \AtBeginShipoutBox 
\usepackage{etoolbox}
\makeatletter
\patchcmd\@combinedblfloats{\box\@outputbox}{\unvbox\@outputbox}{}{%
   \errmessage{\noexpand\@combinedblfloats could not be patched}%
}%
 \makeatother


%% http://en.wikibooks.org/wiki/LaTeX/Colors



%%%%%%%%%%%%%%%%%%%%%%%%%%%%%%%%%%%%%%%%%%%
%       define Journal abbreviations      %
%%%%%%%%%%%%%%%%%%%%%%%%%%%%%%%%%%%%%%%%%%%
\def\nat{Nat} \def\apjl{ApJ~Lett.} \def\apj{ApJ}
\def\apjs{ApJS} \def\aj{AJ} \def\mnras{MNRAS}
\def\prd{Phys.~Rev.~D} \def\prl{Phys.~Rev.~Lett.}
\def\plb{Phys.~Lett.~B} \def\jhep{JHEP} \def\nar{NewAR}
\def\npbps{NUC.~Phys.~B~Proc.~Suppl.} \def\prep{Phys.~Rep.}
\def\pasp{PASP} \def\aap{Astron.~\&~Astrophys.} \def\araa{ARA\&A}
\def\pasa{PASA}
\def\jcap{\ref@jnl{J. Cosmology Astropart. Phys.}}%
\def\physrep{Phys.~Rep.}


\newcommand{\preep}[1]{{\tt #1} }

%%%%%%%%%%%%%%%%%%%%%%%%%%%%%%%%%%%%%%%%%%%%%%%%%%%%%
%              define symbols                       %
%%%%%%%%%%%%%%%%%%%%%%%%%%%%%%%%%%%%%%%%%%%%%%%%%%%%%
\def \Mpc {~{\rm Mpc} }
\def \Om {\Omega_0}
\def \Omb {\Omega_{\rm b}}
\def \Omcdm {\Omega_{\rm CDM}}
\def \Omlam {\Omega_{\Lambda}}
\def \Omm {\Omega_{\rm m}}
\def \ho {H_0}
\def \qo {q_0}
\def \lo {\lambda_0}
\def \kms {{\rm ~km~s}^{-1}}
\def \kmsmpc {{\rm ~km~s}^{-1}~{\rm Mpc}^{-1}}
\def \hmpc{~\;h^{-1}~{\rm Mpc}} 
\def \hkpc{\;h^{-1}{\rm kpc}} 
\def \hmpcb{h^{-1}{\rm Mpc}}
\def \dif {{\rm d}}
\def \mlim {m_{\rm l}}
\def \bj {b_{\rm J}}
\def \mb {M_{\rm b_{\rm J}}}
\def \mg {M_{\rm g}}
\def \qso {_{\rm QSO}}
\def \lrg {_{\rm LRG}}
\def \gal {_{\rm gal}}
\def \xibar {\bar{\xi}}
\def \xis{\xi(s)}
\def \xisp{\xi(\sigma, \pi)}
\def \Xisig{\Xi(\sigma)}
\def \xir{\xi(r)}
\def \max {_{\rm max}}
\def \gsim { \lower .75ex \hbox{$\sim$} \llap{\raise .27ex \hbox{$>$}} }
\def \lsim { \lower .75ex \hbox{$\sim$} \llap{\raise .27ex \hbox{$<$}} }
\def \deg {^{\circ}}
%\def \sqdeg {\rm deg^{-2}}
\def \deltac {\delta_{\rm c}}
\def \mmin {M_{\rm min}}
\def \mbh  {M_{\rm BH}}
\def \mdh  {M_{\rm DH}}
\def \msun {M_{\odot}}
\def \z {_{\rm z}}
\def \edd {_{\rm Edd}}
\def \lin {_{\rm lin}}
\def \nonlin {_{\rm non-lin}}
\def \wrms {\langle w_{\rm z}^2\rangle^{1/2}}
\def \dc {\delta_{\rm c}}
\def \wp {w_{p}(\sigma)}
\def \PwrSp {\mathcal{P}(k)}
\def \DelSq {$\Delta^{2}(k)$}
\def \WMAP {{\it WMAP \,}}
\def \cobe {{\it COBE }}
\def \COBE {{\it COBE \;}}
\def \HST  {{\it HST \,\,}}
\def \Spitzer  {{\it Spitzer \,}}
\def \ATLAS {VST-AA$\Omega$ {\it ATLAS} }
\def \BEST   {{\tt best} }
\def \TARGET {{\tt target} }
\def \TQSO   {{\tt TARGET\_QSO}}
\def \HIZ    {{\tt TARGET\_HIZ}}
\def \FIRST  {{\tt TARGET\_FIRST}}
\def \zc {z_{\rm c}}
\def \zcz {z_{\rm c,0}}

\newcommand{\ltsim}{\raisebox{-0.6ex}{$\,\stackrel
        {\raisebox{-.2ex}{$\textstyle <$}}{\sim}\,$}}
\newcommand{\gtsim}{\raisebox{-0.6ex}{$\,\stackrel
        {\raisebox{-.2ex}{$\textstyle >$}}{\sim}\,$}}
\newcommand{\simlt}{\raisebox{-0.6ex}{$\,\stackrel
        {\raisebox{-.2ex}{$\textstyle <$}}{\sim}\,$}}
\newcommand{\simgt}{\raisebox{-0.6ex}{$\,\stackrel
        {\raisebox{-.2ex}{$\textstyle >$}}{\sim}\,$}}

\newcommand{\Msun}{M_\odot}
\newcommand{\Lsun}{L_\odot}
\newcommand{\lsun}{L_\odot}
\newcommand{\Mdot}{\dot M}

\newcommand{\sqdeg}{deg$^{-2}$}
\newcommand{\hi}{H\,{\sc i}\ }
\newcommand{\lya}{Ly$\alpha$\ }
%\newcommand{\lya}{Ly\,$\alpha$\ }
\newcommand{\lyaf}{Ly\,$\alpha$\ forest}
%\newcommand{\eg}{e.g.~}
%\newcommand{\etal}{et~al.~}
\newcommand{\lyb}{Ly$\beta$\ }
\newcommand{\cii}{C\,{\sc ii}\ }
\newcommand{\ciii}{C\,{\sc iii}]\ }
\newcommand{\civ}{C\,{\sc iv}\ }
\newcommand{\SiII}{Si\,{\sc ii}\ }
\newcommand{\SiIV}{Si\,{\sc iv}\ }
\newcommand{\mgii}{Mg\,{\sc ii}\ }
\newcommand{\feii}{Fe\,{\sc ii}\ }
\newcommand{\feiii}{Fe\,{\sc iii}\ }
\newcommand{\caii}{Ca\,{\sc ii}\ }
\newcommand{\halpha}{H\,$\alpha$\ }
\newcommand{\hbeta}{H\,$\beta$\ }
\newcommand{\hgamma}{H\,$\gamma$\ }
\newcommand{\hdelta}{H\,$\delta$\ }
\newcommand{\oi}{[O\,{\sc i}]\ }
\newcommand{\oii}{[O\,{\sc ii}]\ }
\newcommand{\oiii}{[O\,{\sc iii}]\ }
\newcommand{\heii}{He\,{\sc ii}\ }
%\newcommand{\heii}{[He\,{\sc ii}]\ }
\newcommand{\nv}{N\,{\sc v}\ }
\newcommand{\nev}{Ne\,{\sc v}\ }
\newcommand{\neiii}{[Ne\,{\sc iii}]\ }
\newcommand{\alii}{Al\,{\sc ii}\ }
\newcommand{\aliii}{Al\,{\sc iii}\ }
\newcommand{\siiii}{Si\,{\sc iii}]\ }


\begin{document}

\title[Very high-$z$ Quasars]
        {Near and Mid-infrared properties of known $z\geq5$ Quasars}
\author[Ross \& Cross]
       {Nicholas P. Ross$^{1}$\thanks{Corresponding Author: npross@roe.ac.uk} and Nicholas J. G. Cross$^{1}$
\\ 
$^1$Institute for Astronomy, University of Edinburgh, Royal Observatory, Edinburgh, EH9 3HJ, United Kingdom\\
}

\maketitle
\begin{abstract}
We assemble a catalogue of 463 spectroscopically confirmed very high
($z\geq5.00$) redshift quasars and report their near ($zZyYJHK_{s}$
and $K$) infrared and mid-infrared (WISE) properties.  Using archival
WFCAM/UKIRT and VIRCAM/VISTA data we check for photometric variability
in the near-infrared that might be expected from super-Eddington
accretion and find {\it blah}.  Extrapolating the known quasar
luminosity function we suggest that $x$\% of the possibly detected
$z\geq5$ quasars in the current datasets have been discovered.  All
the data, analysis codes and plots used and generated here can be
found at: \href{https://github.com/d80b2t/VHzQ}{\tt
github.com/d80b2t/VHzQ}.
\end{abstract}


\begin{keywords}
Astronomical data bases: surveys -- 
Quasars: general -- 
galaxies: evolution -- 
galaxies: infrared.
\end{keywords}



%%%%%%%%%%%%%%%%%%%%%%%%%%%%%%%%%%%%%%%%%%%%%%%%%%%%%%%%%%%%%%%%%%
%%%%%%%%%%%%%%%%%%%%%%%%%%%%%%%%%%%%%%%%%%%%%%%%%%%%%%%%%%%%%%%%%%
%%
%%  S E C T I O  N   1         S E C T I O  N   1           S E C T I O  N   1       S E C T I O  N   1
%%  S E C T I O  N   1         S E C T I O  N   1           S E C T I O  N   1       S E C T I O  N   1
%%  S E C T I O  N   1         S E C T I O  N   1           S E C T I O  N   1       S E C T I O  N   1
%%
%%%%%%%%%%%%%%%%%%%%%%%%%%%%%%%%%%%%%%%%%%%%%%%%%%%%%%%%%%%%%%%%%%
%%%%%%%%%%%%%%%%%%%%%%%%%%%%%%%%%%%%%%%%%%%%%%%%%%%%%%%%%%%%%%%%%%
\section{Introduction}
Very high redshift quasars (VH$z$Q; defined here to have redshifts
$z\geq5.00$) are excellent probes of the early Universe. This includes
studies of the Epoch of Reionization for hydrogen \citep[see e.g.][for
reviews]{Fan2006review, Mortlock2016}, the formation and build-up of
supermassive black holes \citep[e.g., ][]{Rees1984, WyitheLoeb2003,
Volonteri2010, Agarwal2016, Valiante2018, Latif2018, Wise2019} and
early metal enrichment \citep[see e.g., ][]{Simcoe2012, Chen2017,
Bosman2017}.

Super-critical accretion, where $\dot{M} > \dot{M}_{\rm Edd}$, is a
viable mechanism to explain the high, potentially super-Eddington,
luminosity and rapid growth of supermassive black holes in the early
universe \citep[e.g.,][]{AlexanderNatarajan2014, MadauHaardtDotti2014,
Volonteri2015, Pezzulli2016, Lupi2016, Pezzulli2017, Takeo2018}. Thus,
one could expect VH$z$Qs to potentially vary in luminosity as they go
through phases of super-critical accretion. These signatures of
photometric variability should be looked for, noting the rest-frame
optical emission is redshifted into the observed near-infrared (NIR)
at redshifts $z>5$. Fortunately, data are now in place from deep,
wide-field NIR instruments and surveys such as the Wide Field Camera
(WFCAM) instrument on the United Kingdom Infra-Red Telescope (UKIRT)
in the Northern Hemisphere and the VISTA InfraRed CAMera (VIRCAM) on
the Visible and Infrared Survey Telescope for Astronomy (VISTA) in the
Southern Hemisphere, that are necessary for identifying VH$z$Qs.

Quasars are known to be prodigious emitters of infrared emission,
thought to be from the thermal emission of dust grains heated by
continuum emission from the accretion disc
\citep[e.g.,][]{Richards2006b, Leipski2014, Hill2014,
Hickox2017}. Observations in the mid-infrared, e.g. $\sim$3-30$\mu$m
allow discrimination between AGN\footnote{Historically, ``quasars''
and ``Active Galactic Nuclei (AGN)'' have described different
luminosity/classes of objects. In recognition of the fact that both
terms describe accreting supermassive black holes, we use these terms
interchangeably, with a preference for quasar, since we are generally
in the higher-$L$ regime \citep[e.g.][]{Haardt2016book}.}  and passive
galaxies due to the 1.6$\mu$m ``bump'' entering the MIR at
$z\approx0.8-0.9$ \citep[e.g., ][]{Wright1994, Sawicki2002, Lacy2004,
Stern2005, Richards2006b, Timlin2016} as well as between AGN and
star-forming galaxies due to the presence of Polycyclic Aromatic
Hydrocarbon (PAHs) at $\lambda >3\mu$m \citep[e.g., ][]{Yan2007,
Tielens2008}.

\citet{Jiang2006dust} and \citet{Jiang2010} report on the discovery of
a quasar without hot-dust emission in a sample of 21 $z\approx6$
quasars. Such apparently hot-dust-free quasars have no counterparts at
low redshift. Moreover, those authors demonstrate that the hot-dust
abundance in the 21 quasars builds up in tandem with the growth of the
central black hole. But understanding how dust first forms and appears
in the central engine remains an open question \citep{WangR2008,
WangR2011}.

WISE mapped the sky in 4 passbands, in bands centered at wavelengths
of 3.4, 4.6, 12, and 23$\mu$m. The all sky `ALLWISE' catalogue
release, contains nearly 750 million detections at
high-significance\footnote{\href{wise2.ipac.caltech.edu/docs/release/allwise/expsup/sec2\_1.html}{wise2.ipac.caltech.edu/docs/release/allwise/expsup/sec2\_1.html}},
of which over 4.5M AGN candidates have been identified with 90\%
reliability \citep{Assef2018}.  \citet{Blain2013} presented WISE
mid-infrared (MIR) detections of 17 (55\%) of the then known 31
quasars at $z > 6$. However, \citet{Blain2013} was compiled with the
WISE `All-Sky' data release, as opposed to the superior AllWISE
catalogues. That sample only examined the 31 known $z>6$ quasars; our
sample has 170 objects with redshift $z \geq 6.00$ (with 108 detected
in WISE). \citet{Banados2016} reports WISE W1, W2, W3 and W4
magnitudes for the Panoramic Survey Telescope and Rapid Response
System 1 \citep[Pan-STARRS1, PS1;][]{Kaiser2002, Kaiser2010}, but with
no further investigation into the reddest WISE waveband for the
VH$z$Qs.

Critically, we now have available to us new W1 and W2 photometry from
the `unWISE Source Catalog' \citep[][]{Schlafly_Meisner2018}, a
WISE-selected catalogue that is based on significantly deeper imaging
and has a more extensive modeling of crowded regions than the ALLWISE
release. For the first time in a catalogue, unWISE takes advantage of
the ongoing mid-IR Near-Earth Object Wide-Field Infrared Survey
Explorer Reactivation mission \citep[NEOWISE-R; ][]{Mainzer2014}, and
achieves depths $\sim$0.7 mag deeper than ALLWISE (in W1/2).  This
additional depth is a significant advantage in the detection and study
of VH$z$Qs in the 3-5 micron regime.

Here we present for the first time the combined near-infrared
properites (from UKIRT and VIRCAM) and the new mid-infrared unWISE for
all the spectroscopically known $z\geq5.00$ quasars. Our motivations
are numerous and include: {\it (i)} establishing the first complete
catalogue of $z>5.00$ quasars since the pioneering work from SDSS;
{\it (ii)} utilizing all the WFCAM and VISTA near-infrared photometry
available for the quasars; {\it (iii)} making the first study of near-
and mid-IR variability of the VHzQ population and {\it (iv)}
establishing the photometric properties for upcoming surveys and
telescopes including the Large Synoptic Survey Telescope
(LSST)\footnote{\href{https://www.lsst.org}{{lsst.org}}}, ESA {\it
Euclid}\footnote{\href{https://sci.esa.int/euclid/}{sci.esa.int/euclid/}}
and the {\it James Webb Space Telescope}
(JWST)\footnote{\href{https://www.jwst.nasa.gov/}{jwst.nasa.gov};}$^,$\footnote{\href{https://sci.esa.int/jwst/}{sci.esa.int/jwst};}$^,$\footnote{\href{https://www.asc-csa.gc.ca/eng/satellites/jwst/}{www.asc-csa.gc.ca/eng/satellites/jwst};}$^,$\footnote{\href{https://jwst.stsci.edu/}{jwst.stsci.edu}}. We
chose redshift $z=5.00$ as our lower redshift limit due to a
combination of garnishing a large sample, adequately spanning physical
properties (e.g. luminosity, age of the Universe) and to highlight the
parts of $L-z$ parameter space where $z>5$ quasars still wait to be
discovered.

This paper is organized as follows. In Section 2, we present the
assembled list of the 463 $z\geq5.00$ VH$z$Qs that we have
compiled. We then give a high-level overview of the photometric
surveys and datasets we use and present the photometry of the
VH$z$Qs. In Section 3 we investigate the variability properties of the
VH$z$Qs, looking for evidence of super-critical accretion. In Section
4 we calculate how many $z>5$ quasars we should expect to find in
current datasets. We conclude in Section 5 and present all the
necessary details to obtain our dataset in the Appendices.

We present all our photometry and magnitudes on the AB zero-point
system \citep{Oke_Gunn1983, Fukugita1996}.  This includes the
near-infrared, as well as the mid-infrared magnitudes. These
magnitudes are {\it not} Galactic extinction corrected.
%Appendix~\ref{sec:filters} gives the AB to Vega transforms for a wide range of optical, NIR and MIR filters. 
We use a flat $\Lambda$CDM cosmology with $H_{\rm 0} = 67.7$ km s$-1$ Mpc$−1$, $\Omega_{\rm
M} = 0.307$, and $\Omega_{\Lambda} = 0.693$ (Planck Collaboration et
al. 2016) to be consistent with \citet{Banados2016} and all logarithms
are to the base 10.




%%%%%%%%%%%%%%%%%%%%%%%%%%%%%%%%%%%%%%%%%%%%%%%%%%%%%%%%%%%%%%%%%%
%%%%%%%%%%%%%%%%%%%%%%%%%%%%%%%%%%%%%%%%%%%%%%%%%%%%%%%%%%%%%%%%%%
%%
%%    S E C T I O  N   2         S E C T I O  N   2           S E C T I O  N   2       S E C T I O  N   2
%%    S E C T I O  N   2         S E C T I O  N   2           S E C T I O  N   2       S E C T I O  N   2
%%    S E C T I O  N   2         S E C T I O  N   2           S E C T I O  N   2       S E C T I O  N   2
%%
%%%%%%%%%%%%%%%%%%%%%%%%%%%%%%%%%%%%%%%%%%%%%%%%%%%%%%%%%%%%%%%%%%
%%%%%%%%%%%%%%%%%%%%%%%%%%%%%%%%%%%%%%%%%%%%%%%%%%%%%%%%%%%%%%%%%%
\begin{figure}
%%trim=l b r t
%  \includegraphics[width=18.6cm, clip,trim=14mm 4mm 10mm 10mm]
  \includegraphics[width=8.6cm, clip,trim=32mm 4mm 32mm 10mm]
  {/cos_pc19a_npr/programs/quasars/highest_z/SEDs/filters_vs_QSOstars_z6pnt7.pdf}
  \centering
  \vspace{-12pt}
  \caption[]
  {The spectral bands used by different survey telescopes and that are relevant here.
    The $grizy$ filters are from the Pan-STARRS survey. The $JHK$ are from 
    UKIRT/WFCAM, while $K_{s}$ is a VISTA/VIRCAM filter. The 
    The narrow W-band centered at $\lambda\approx14,500$\AA\ is a CFHT/Wircam filter. 
    The WISE passbands  W1-4 are also presented.
    The quasar spectrum is a composite based on \citet{VdB2001} and 
    \citet{Banados2016}. The L and T dwarf spectra are from \citet{Cushing2006}. 
  }
  \label{fig:filters}
\end{figure}

\vspace{-16pt}
\section{Method and Data}
Quasars are generally identified by photometric selection followed by
spectroscopic confirmation. Here, we reverse this method obtaining
first a list of spectroscopic quasars and then obtain photometric
information.

We have compiled a list of 463 quasars with redshifts $z\geq5.00$. We
use all the $z\geq5.00$ quasars that have been discovered,
spectroscopically confirmed and published as of 2018 December 31 (MJD
58483). We then obtain optical, near-infrared and mid-infrared
photometry for the spectral dataset. 
%The optical data comes from the Panoramic Survey Telescope and Rapid Response System (Pan-STARRS) survey \citep{Chambers2016}. 
The near-infrared data comes from two
sources: first, the WFCAM \citep[][]{Casali2007} on the UKIRT,
primarly, but not exclusively, as part of the UKIRT Infrared Deep Sky
Survey \citep[UKIDSS; ][]{Lawrence2007}.  And second, data from the
VIRCAM on the VISTA \citep[][]{Emerson2006, Dalton2006}. The
mid-infrared, $\lambda=3-30\mu$m wavelength data is from the the
Wide-Field Infrared Survey Explorer \citep[WISE;][]{Wright2010,
Cutri2013} mission. For reference, Figure~\ref{fig:filters} displays
the wavelength and normalised transmission of the filters in question.

\subsection{Spectroscopy} 
We compile the list of all known, spectroscopically confirmed
quasars from the literature.  This list was complied from a range of
surveys and papers.  Specifically, we use data from:
\citet{Banados2014, Banados2016, Banados2018}, 
\citet{Becker2015}, 
\citet{Calura2014}, 
\citet{Carilli2007, Carilli2010}, 
\citet{Carnall2015}, 
\citet{Cool2006}, 
\citet{DeRosa2011}, 
\citet{Fan2000, Fan2001c, Fan2003, Fan2004, Fan2006, Fan2018}, 
\citet{Goto2006}, 
\citet{Ikeda2017}, 
\citet{Jiang2008, Jiang2009, Jiang2015, Jiang2016},   
\citet{Kashikawa2015}, 
\citet{Koptelova2017}, 
\citet{Kim2015, Kim2018},  
\citet{Kurk2007, Kurk2009}, 
\citet{Leipski2014}, 
\citet{Mahabal2005}, 
\citet{Matsuoka2016,  Matsuoka2018a, Matsuoka2018b},   
\citet{Mazzucchelli2017}, 
\citet{Morganson2012}, 
\citet{Mortlock2009, Mortlock2011},
\citet{McGreer2006, McGreer2013},  
\citet{Reed2015, Reed2017}, 
\citet{Stern2007},  
\citet{Tang2017}, 
\citet{Venemans2007, Venemans2012, Venemans2013, Venemans2015a, Venemans2015b, Venemans2016},
\citet{WangF2016, WangF2017, WangF2018a, WangF2018b},
\citet{Willott2007, Willott2009, Willott2010a, Willott2013b, Willott2015}, 
\citet{Wu2015} 
\citet{YangJ2018a, YangJ2018b}  
and 
\citet{Zeimann2011}. 

Most of these objects are easily identified by their broad Ly$\alpha$
emission line, \nv emission and characteristic shape blueward of
1215\AA\ in the rest-frame. As we shall see, some of the recently
discovered objects are close to the galaxy luminosity function
characteristic luminosity $M^{*}$, and some have relatively weak or
maybe even completely absorbed Ly$\alpha$ \citep[e.g. Figures 7 and 10
in][]{Banados2016}. We leave aside detailed investigation and
discussion into spectral features and line strengths, and take as
given the published spectra and redshift identifications.

The breakdown of how many VH$z$Q each survey reports is given in
Table~\ref{tab:surveys}. The Sloan Digital Sky Survey (SDSS) and the
Pan-STARRS1 (PS1; PSO in Table~\ref{tab:surveys}) survey and alone
identified over half (54.6\%) of the VH$z$Q population. Data from the
Hyper Suprime-Cam (HSC) on the Subaru telecope is responsible for
13.6\% of our dataset (HSC+SHELLQs in Table~\ref{tab:surveys}). The
combination of surveys is also vital for identifying VH$z$Qs. The
UKIDSS Large Area Survey (ULAS) on its own, or in combination with
other surveys is responsible for 6.5\% of the sample (SUV+ULAS)
including the highest-$z$ object. Where more than one survey is used
for the high-redshift identification (e.g. via shorter-band veto and
longer wavelength detection) we follow the discovery paper naming
convention.


\begin{table}
\begin{tabular}{l r r l}
\hline  \hline
Survey              & \# VH$z$Qs & (\%) & Survey reference  \\
\hline  
  ATLAS             &     4    &   ( 0.86)    &  \citet{Shanks2015} \\
  CFHQS            &   20    &   ( 4.32)    &  \citet{Willott2007} \\
  DELS               &   16    &   ( 3.46)    &  \citet{Dey2018} \\
  ELAIS              &     1    &   ( 0.22)    &  \citet{Vaisanen2000} \\
  FIRST              &     1    &   ( 0.22)    &  \citet{Becker1995} \\
  HSC                 &    8    &   ( 1.73)    & \citet{Miyazaki2018} \\
  IMS                 &     5    &   ( 1.08)     &  \citet{Kim2015} \\
  MMT               &   12    &   ( 2.59)     &  \citet{McGreer2013} \\
  NDWFS           &     1    &   ( 0.22)    &  \citet{JD1999} \\
  PSO                 &   83   &   (17.93)   &   \citet{Kaiser2002, Kaiser2010} \\
  RD                   &     1   &   ( 0.22)    &  \citet{Mahabal2005} \\
  SDSS                &  170  &    (36.72)    & \citet{EDR} \\
 SDWISE$^{b}$    &   27    &  ( 5.83)    &   \citet{WangF2016} \\
%% SHELLQs    &  $^{c}$63  &  (13.6)   &  \citet{Matsuoka2016} \\  %% if you include 8 objects from the first HSC paper...
  SHELLQs         &    55    &   (11.88)  &  \citet{Matsuoka2016}     \\  
  SUV$^{c}$       &   20     &    ( 4.32)  & \citet{YangJ2017} \\
  UHS               &    1      &  ( 0.22)     &  \citet{WangF2017} \\
  ULAS               &   10   &   ( 2.16)     & \citet{Lawrence2007} \\
  VDES$^{d}$       &   17  &    ( 3.67)     &  \citet{Reed2017} \\
  VHS                 &     1  &     ( 0.22)    & \citet{WangF2018b} \\
  VIK                 &     9    &  ( 1.94)    &  \citet{Edge2013} \\
  VIMOS           &    1      &  ( 0.22)     &   \citet{LeFevre2003} \\
\hline  \hline
\end{tabular}
\caption{The source and number of the VH$z$Q, with the key survey reference also given. 
  Recent survey name and acronyms include: 
  $^{a}$DESI Legacy Imaging Survey; 
  $^{b}$SDWISE = SDSS+WISE; 
  $^{a}$SUV  = SDSS-ULAS/VHS; 
  $^{c}$VDES = VHS/VIKING+DES; 
%  $^{c}$Includes 8 objects with a HyperSuprimeCam \citep[HSC; ][]{Miyazaki2018} designation.
}
      \label{tab:surveys}
\end{table}

\begin{figure}
%%trim=l b r t
  \includegraphics[width=8.0cm, clip, trim=10mm 0mm 0mm 0mm]
  {/cos_pc19a_npr/programs/quasars/highest_z/Nofz/Nofz_0pnt075bins_20181211.png}
  \centering
  \vspace{-12pt}
  \caption[]
  {The redshift distribution $N(z)$ of the VH$z$Q sample. 
    The bins are $\delta-z=0.075$ in width. }
  \label{fig:Nofz}
\end{figure}

The redshifts for the VH$z$Qs generally come from the measurement of
broad UV/optical emission lines. There are far infra-red emission
lines e.g. \cii~158 micron available for several objects, but at the
level of our current analysis broadline redshifts are sufficient.

\begin{table}
\centering
\begin{tabular}{l r  r}
\hline \hline
$z \geq$  & Age / Myr & No. of objects \\
\hline 
7.50         &    700         &   1   \\
7.00         &    767         &   4   \\
6.78         &    800         &  14   \\
6.50         &    845         &  40   \\
6.19         &   900          &  86   \\
6.00         &   937          &  170   \\
5.70         & 1000          &  267   \\
5.00         & 1180          &  463   \\
\hline \hline
\end{tabular}
\caption{The number of objects at or above a given redshift. 
The age of the Universe in megayears is also given. }
      \label{tab:ages}
\end{table}

The number of objects at or above various redshifts, along with the 
corresponding age of the Universe is given in Table~\ref{tab:ages}. 

The $N(z)$ redshift histogram is given for the sample in Figure~\ref{fig:Nofz}. 
We split the contribution up by survey. For clarity we show the individual 
surveys of SDSS, PS1, HSC, the ULAS detection, and tally the remaining 
surveys together (``various''). 


\subsection{Near-infrared photometry}~\label{sec:NIR_data} 
The near-infrared data in this paper comes from the Wide Field
Astronomy Unit's \href{https://www.roe.ac.uk/ifa/wfau/}{(WFAU)}
Science Archives for UKIRT-WFCAM, the WFCAM Science Archive
\citep[WSA; ][]{Hambly2008} and VISTA-VIRCAM, the VISTA Science
Archive \citep[VSA; ][]{Cross2012}. These archives were developed for
the VISTA Data Flow System \citep[VDFS][]{VDFS}.

We access both the WSA and the VSA and include all non-proprietary
WFCAM data, which covers all public surveys and PI projects from
Semester 05A to 2017-Jan-01, and all non-proprietary VISTA data, which
covers all public surveys and PI projects from science verification on
2009-Oct-15 to 2016-Apr-01.

Here we are not just quering the WSA or VSA data tables. We are taking
a list of objects (positions) are performing matched aperature
(``forced'') photometry on the NIR imaging data. As such, we generate
a set of tables that are different in subtle ways to the regular
``Detection'' tables.  The two most important tables for our needs are
the {\tt [w/v]serv1000MapRemeasurement} and {\tt
[w/v]serv1000MapRemeasAver}.

We produce and provide a two new databases with all the necessary
quantaties and measurements to fully reproduce our tables, figures and
results herein. Morever, these databases report considerably more
information than we report here. Full documentation can be found at
the \href{http://wsa.roe.ac.uk/www/wsa_browser.html}{WSA Schema
Browser} and the
\href{http://horus.roe.ac.uk/vsa/www/vsa_browser.html}{VSA Schema
Browser}.


  \subsubsection{Averaging matched photometry}
  The data was processed using a matched-aperture photometry method
  where flux is measured at the spectroscopic position of the quasar,
  without necessarily knowing if there is a formal detection in the NIR
  photometry beforehand. The matched-aperture pipeline is discussed in
  \citet{Cross2013} and with fuller details to appear in a forthcoming
  paper (Cross et al., 2019, in prep).
  
  We query the WSA and VSA performing matched-aperature photometry at
  the positions of our 463 VH$z$Qs. This database is world-readable and
  we give the full receipe and relevant SQL queries for accessing both
  databases in Appendix~\ref{sec:SQL} as well as online. 
  
  The photometry in a single epoch image often has low
  signal-to-noise.  The advantage of matched aperture photometry on
  quasars is that co-adding is relatively simple if each epoch is taken
  in the same aperture and the aperture photometry has been corrected to
  total. Indeed, the standard aperture corrections work well for point
  sources. Coadding using the matched-aperature photometry, where the
  individual epochs are taken from multiple projects with different
  pointings and orientations, should help with issues such as scattered
  light, pixel distortion and aperture corrections.
    
  We average the aperture corrected calibrated fluxes (e.g. {\bf
  aperJky3}), and then convert to magnitudes. Since we do not have a
  deep image for each set of averages, we cannot calculate non-aperture
  corrected values, so the photometry is only appropriate for
  point-sources.

  \begin{equation}
    \bar{F} = \frac{\sum_i^N (w_i\,F_i)}{\sum_i^N w_i}  
    \label{eq:avg}
  \end{equation}
  where $F_i$ is the $i^{th}$ epoch measurement of a parameter to be
  averaged such as the aperture corrected calibrated flux in a $1\arcsec$ aperture
  ({\bf aperJky3}) and $\bar{F}$ is the weighted mean average of this parameter.
  The weight for each epoch $w_i=1/(\sigma_{F})^2$ if the epoch is included and 
  $w_i=0$ if an epoch is excluded for quality control purposes. 
   
  We calculate a set of averaged catalogues, for each pointing and filter, based
  on the requirements in \verb+RequiredMapAverages+, in these cases over time
  spans of 7, 14, 30, 91, 183 days, 365 days, 730 days, over 10 epochs and
  over all epochs. The averaging process starts at the first epoch and works onwards 
  from there. Again, we present these measurements in the new SQL tables. 

  We detect 359 unique quasars in the WFCAM WSA database, 220 quasars are 
  detected in the VISTA VSA database with 130 objects in common with both
  WFCAM and VISTA data.  
  We give the necessary SQL queries syntax at 
  \href{https://github.com/d80b2t/VHzQ/blob/master/data/WSA_VSA/SAMPLE_SQL_QUERIES}{\tt d80b2t/VHzQ}. 


\subsection{MIR data}
The MIR data for this study comes from the Wide-field Infrared Survey
Explorer (WISE) mission, and we utlize data from the WISE cryogenic
and NEOWISE (Mainzer et al. 2011 ApJ, 731, 53) post-cryogenic survey
phases.

We use data from the the beginning of the WISE mission \citep[2010
January; ][]{Wright2010} through the fourth-year of NEOWISE-R
operations \citep[2017 December;]{Mainzer2011}.  More specifically, we
utilise the recently released ``unWISE Catalog''
\citet{Schlafly2019}. The unWISE effort\footnote{http://unwise.me/} is
the unblurred coadds of the WISE imaging using the AllWISE and
NEOWISE-R stacked data \citep{Lang2014, Meisner2018a, Meisner2018b}.

All fluxes in the unWISE catalog are reported there are in ``Vega
nanoMaggies'', with the Vega magnitude of a source is given by 
\begin{equation}
m_{\rm Vega} = 22.5 - 2.5\log(f),
\end{equation} 
where $f$ is the source flux. The absolute calibration for unWISE is
ultimately inherited from AllWISE through the calibration of
\citet{Meisner2017a}. This inheritance depends on details of the PSF
normalization at large radii, which is uncertain. Subtracting 4
millimag from the unWISE W1, and 32 millimag from unWISE W2 fluxes
improves the agreement between unWISE and AllWISE fluxes.

Thus to convert unWISE Vega magnitudes onto the AB system, we have:     
\begin{eqnarray*}
        {\rm W1}_{\rm AB, unWISE}  & = &   22.5 - 2.5 \log(f_{\rm W1}) - 0.004 + 2.699 \\
        {\rm W2}_{\rm AB, unWISE}  &  = &  22.5 - 2.5 \log(f_{\rm W2}) - 0.032 + 3.339. 
\end{eqnarray*}
For the our MIR variability investigations, we do not use the unWISE
coadds, but instead use the
\href{http://wise2.ipac.caltech.edu/docs/release/allwise/}{AllWISE}
catalogue and the
\href{http://wise2.ipac.caltech.edu/docs/release/neowise/neowise_2018_release_intro.html}{NEOWISE
2018 Data Release}. NEOWISE 2018 makes available the 3.4 and 4.6 μm
(W1 and W2) single-exposure images and extracted source information
that was acquired between 2016 December 13 and 2017 December 13 UTC,
which was the fourth year of survey operations of the Near-Earth
Object Wide-field Infrared Survey Explorer Reactivation Mission
(NEOWISE; Mainzer et al. 2014, ApJ, 792, 30). The fourth year NEOWISE
data products are concatenated with those from the first three years
into a single archive.

The WISE scan pattern leads to coverage of the full-sky approximately once every six months (a``sky pass''), but the satellite was placed in hibernation in 2011 February and then reactivated in 2013 October. Hence, our light curves have a cadence of 6 months with a 32 month sampling gap.

Table~\ref{tab:output_table} represents the culmination of this effort, and we now describe the assembly of their contents in more detail.  

%\begin{landscape}
\begin{table}
%\begin{center}
\begin{tabular}{|l|l|r|r|r|r|r|r|r|r|r|r|r|r|r|r|r|r|r|r|r|r|r|r|r|r|r|}
\hline
  \multicolumn{1}{|c|}{survey} &
  \multicolumn{1}{c|}{qsoName} &
  \multicolumn{1}{c|}{ra} &
  \multicolumn{1}{c|}{dec} &
  \multicolumn{1}{c|}{redshift} &
  \multicolumn{1}{c|}{Z} &
  \multicolumn{1}{c|}{errZ} &
  \multicolumn{1}{c|}{Y} &
  \multicolumn{1}{c|}{errY} &
  \multicolumn{1}{c|}{J} &
  \multicolumn{1}{c|}{errJ} &
  \multicolumn{1}{c|}{H} &
  \multicolumn{1}{c|}{errH} &
  \multicolumn{1}{c|}{K} &
  \multicolumn{1}{c|}{errK} &
  \multicolumn{1}{c|}{unW1mag} &
  \multicolumn{1}{c|}{unW1err} &
  \multicolumn{1}{c|}{unW1snr} &
  \multicolumn{1}{c|}{unW2mag} &
  \multicolumn{1}{c|}{unW2err} &
  \multicolumn{1}{c|}{unW2snr} &
  \multicolumn{1}{c|}{w3mpro} &
  \multicolumn{1}{c|}{w3sig} &
  \multicolumn{1}{c|}{w3snr} &
  \multicolumn{1}{c|}{w4mpro} &
  \multicolumn{1}{c|}{w4sig} &
  \multicolumn{1}{c|}{w4snr} \\
\hline
  PSO & J000.3401+26.8358 & 0.3401135 & 26.8358814 & 5.75 & -999.99999 & -999.99999 & -999.99999 & -999.99999 & 19.284586 & 0.062458 & -999.99999 & -999.99999 & -999.99999 & -999.99999 & 16.2799 & 0.0257 & 41.7265 & 15.5194 & 0.0499 & 21.251 & 12.594 & 0.49 & 2.2 & 8.756 & 0.0 & 1.1\\
  SDSS & J0002+2550 & 0.6641173 & 25.8430442 & 5.82 & -999.99999 & -999.99999 & -999.99999 & -999.99999 & 19.373274 & 0.069474 & -999.99999 & -999.99999 & -999.99999 & -999.99999 & 16.25 & 0.0261 & 41.0239 & 15.4177 & 0.0469 & 22.6804 & 12.416 & 0.42 & 2.6 & 8.683 & 0.0 & 1.2\\
  SDWISE & J0008+3616 & 2.2142917 & 36.2704138 & 5.17 & -999.99999 & -999.99999 & -999.99999 & -999.99999 & 19.331099 & 0.063265 & -999.99999 & -999.99999 & -999.99999 & -999.99999 & 16.0176 & 0.0208 & 51.7021 & 15.4339 & 0.0444 & 23.9368 & 12.043 & 0.0 & 1.8 & 8.786 & 0.0 & 1.1\\
  PSO & J002.3786+32.8702 & 2.3787018 & 32.8702618 & 6.1 & -999.99999 & -999.99999 & -999.99999 & -999.99999 & 20.994478 & 0.24893 & -999.99999 & -999.99999 & -999.99999 & -999.99999 & 17.9509 & 0.1056 & 9.7915 & -99.999 & -9.99 & -99.999 & -9.99 & -9.99 & -9.9 & -9.99 & -9.99 & -9.9\\
  SDSS & J0012+3632 & 3.136999 & 36.5378055 & 5.44 & -999.99999 & -999.99999 & -999.99999 & -999.99999 & 19.013725 & 0.048705 & -999.99999 & -999.99999 & -999.99999 & -999.99999 & 15.8214 & 0.0174 & 62.0132 & 15.2306 & 0.0363 & 29.4372 & 12.001 & 0.23 & 4.6 & 8.688 & 0.33 & 3.3\\
  PSO & J004.3936+17.0862 & 4.3936135 & 17.0863045 & 5.8 & -999.99999 & -999.99999 & -999.99999 & -999.99999 & 20.559095 & 0.202108 & -999.99999 & -999.99999 & -999.99999 & -999.99999 & 17.8337 & 0.1032 & 10.0252 & 16.6956 & 0.1451 & 6.9949 & -9.99 & -9.99 & -9.9 & -9.99 & -9.99 & -9.9\\
  PSO & J006.1240+39.2219 & 6.1240417 & 39.2219306 & 6.62 & -999.99999 & -999.99999 & -999.99999 & -999.99999 & 21.280636 & 0.421661 & -999.99999 & -999.99999 & -999.99999 & -999.99999 & 17.3638 & 0.0643 & 16.3805 & -99.999 & -9.99 & -99.999 & -9.99 & -9.99 & -9.9 & -9.99 & -9.99 & -9.9\\
  SDWISE & J0025-0145 & 6.3618333 & -1.7590306 & 5.07 & -999.99999 & -999.99999 & -999.99999 & -999.99999 & -999.99999 & -999.99999 & 17.744932 & 0.003998 & -999.99999 & -999.99999 & 14.8509 & 0.0088 & 122.7056 & 14.2325 & 0.0184 & 58.52 & 11.395 & 0.22 & 4.9 & 8.514 & 0.0 & 0.9\\
  PSO & J007.0273+04.9571 & 7.0273329 & 4.9571232 & 6.0 & -999.99999 & -999.99999 & 20.334393 & 0.055914 & 20.22697 & 0.073723 & 20.288778 & 0.108315 & 20.189209 & 0.105499 & 17.1783 & 0.0596 & 17.7258 & 16.6062 & 0.1348 & 7.5651 & 12.253 & 0.0 & 0.4 & 8.317 & 0.0 & 1.4\\
  SDWISE & J0031+0710 & 7.85775 & 7.1769222 & 5.33 & -999.99999 & -999.99999 & 20.027384 & 0.082396 & 20.196571 & 0.146073 & 19.485508 & 0.105933 & 19.609877 & 0.123397 & 16.6577 & 0.0389 & 27.437 & 15.6829 & 0.0634 & 16.6299 & 12.187 & 0.0 & -0.3 & 8.398 & 0.0 & 0.5\\
\hline
\end{tabular}
\caption{WSA. 
The first ten objects are given here as guidance to the format of the data table. The full table  can be found online.}
\label{tab:WSA_wWISE}
%  \end{center}
\end{table}
\normalsize 
\end{landscape}



%%%%%%%%%%%%%%%%%%%%%%%%%%%%%%%%%%%%%%%%%%%%%%%%%%%%%%%%%%%%%%%%%%%%%%%%%%%%%%%%%
%%
%%
%%     T A B L E    O N E                GENERAL PROPERTIES,          incl.   redshift   and   M_1450
%%
%%
%%%%%%%%%%%%%%%%%%%%%%%%%%%%%%%%%%%%%%%%%%%%%%%%%%%%%%%%%%%%%%%%%%%%%%%%%%%%%%%%
\begin{table*}
\begin{center}
\begin{tabular}
%{|l|l|l|l|r|r|r|r|r|r|r|r|r|r|r|r|r|r|r|r|r|r|r|r|l|}
%{ l l l l r r r r r r r r r r r r r r r r r r r r l }
{ l l   r r  r r   l   l l l   }
\hline \hline
  \multicolumn{1}{ c }{na} &
  \multicolumn{1}{c }{desig} &
  \multicolumn{1}{c }{ra\_hms} &
  \multicolumn{1}{c }{dec\_dms} &
  \multicolumn{1}{c }{ra} &
  \multicolumn{1}{c }{dec} &
  \multicolumn{1}{c }{redshift} &
  \multicolumn{1}{c }{mag} &
  \multicolumn{1}{c }{M1450} &
  \multicolumn{1}{c }{ref} \\
\hline
  PSO & J000.3401+26.8358   & 00:01:21.63 & +26:50:09.17 &   0.340113    &  +26.83588      &   5.75    & 19.52 & -27.16     & 1/1/1\\
  SDSS & J0002+2550             & 00:02:39.39 & +25:50:34.80 &   0.664117    &  +25.84304      &  5.82    & 19.39 & -27.31     & 5/22/1\\
  SDSS & J0005-0006             & 00:05:52.34 & -00:06:55.80 &    1.468083    &  -00.11549      & 5.85    & 20.98 & -25.73     &  5/12/1\\
  PSO & J002.1073-06.4345   & 00:08:25.77 & -06:26:04.60 &    2.107390    &  -06.43456            & 5.93   & 20.41 & -26.32      &   1;43/1/1\\
  SDWISE & J0008+3616         & 00:08:51.43 & +36:16:13.49 &   2.214292    &   +36.27041        & 5.17   & 19.12 & -27.34     &    Wang2016\\
  PSO & J002.3786+32.8702  & 00:09:30.89 & +32:52:12.94 &    2.378702   &   +32.87026     &  6.1     & 21.13  & -25.65    &  1/1/1\\
  SDSS & J0017-1000             & 00:17:14.68 & -10:00:55.4 &      4.311166   &    -10.01540    & 5.011  & 99.99 & -99.99     &   DR7\_W16\\
  PSO & J004.3936+17.0862  & 00:17:34.47 & +17:05:10.70 &    4.393614   &   +17.08631      & 5.8      & 20.69 & -26.01     &  1/1/1\\
  PSO & J004.8140-24.2991  & 00:19:15.38 & -24:17:56.98 &     4.814080   &   -24.29920          & 5.68      & 19.43 & -27.24      &    1/1/1\\
  VDES & J0020-3653            & 00:20:31.46 & -36:53:41.8 &       5.131124   &   -36.89495       & 6.9      & 99.99 & -99.99       &   DES-VHS\_inprep\\
\hline \hline
\end{tabular}
\caption{All 463 $z\geq5.00$ quasars that have been spectroscopically confirmed as of 2018 June. 
  The first ten objects are given here as guidance to the format of the data table. The full table  can be found online.} 
\label{tab:THE_TABLE}
  \end{center}
\end{table*}
\normalsize 


%\footnotesize
%\tiny
\begin{landscape}
        \begin{table}
    \begin{tabular}{ccccc ccccc cccc}
  \hline \hline
  survey   & qsoName &  ra  & dec & redshift  &  
  Z        & Y       &  J   &  H  &  K & 
  W1       & W2      & W3   & W4 
  \\
  \hline \hline
\footnotesize
PSO & J000.3401+26.8358 &    0.34011 &   26.83588 &  5.75   &   $-1000.00\pm-1000.000$  &  $-1000.00\pm-1000.000$  &  $19.28\pm0.062$  &  $-1000.00\pm-1000.000$   & $-1000.00\pm-1000.000$    &   $16.280\pm0.026$   &  $15.52\pm0.050$   &   $12.59\pm0.490$   &   $ 8.76\pm-9.900$   \\
SDSS & J0002+2550 &    0.66412 &   25.84304 &  5.82   &   $-1000.00\pm-1000.000$  &  $-1000.00\pm-1000.000$  &  $19.37\pm0.069$  &  $-1000.00\pm-1000.000$   & $-1000.00\pm-1000.000$    &   $16.250\pm0.026$   &  $15.42\pm0.047$   &   $12.42\pm0.420$   &   $ 8.68\pm-9.900$   \\
SDWISE & J0008+3616 &    2.21429 &   36.27041 &  5.17   &   $-1000.00\pm-1000.000$  &  $-1000.00\pm-1000.000$  &  $19.33\pm0.063$  &  $-1000.00\pm-1000.000$   & $-1000.00\pm-1000.000$    &   $16.018\pm0.021$   &  $15.43\pm0.044$   &   $12.04\pm-9.900$   &   $ 8.79\pm-9.900$   \\
PSO & J002.3786+32.8702 &    2.37870 &   32.87026 &  6.10   &   $-1000.00\pm-1000.000$  &  $-1000.00\pm-1000.000$  &  $20.99\pm0.249$  &  $-1000.00\pm-1000.000$   & $-1000.00\pm-1000.000$    &   $17.951\pm0.106$   &  $-100.00\pm-9.990$   &   $-9.99\pm-9.990$   &   $-9.99\pm-9.990$   \\
SDSS & J0012+3632 &    3.13700 &   36.53781 &  5.44   &   $-1000.00\pm-1000.000$  &  $-1000.00\pm-1000.000$  &  $19.01\pm0.049$  &  $-1000.00\pm-1000.000$   & $-1000.00\pm-1000.000$    &   $15.821\pm0.017$   &  $15.23\pm0.036$   &   $12.00\pm0.230$   &   $ 8.69\pm0.330$   \\
PSO & J004.3936+17.0862 &    4.39361 &   17.08630 &  5.80   &   $-1000.00\pm-1000.000$  &  $-1000.00\pm-1000.000$  &  $20.56\pm0.202$  &  $-1000.00\pm-1000.000$   & $-1000.00\pm-1000.000$    &   $17.834\pm0.103$   &  $16.70\pm0.145$   &   $-9.99\pm-9.990$   &   $-9.99\pm-9.990$   \\
PSO & J006.1240+39.2219 &    6.12404 &   39.22193 &  6.62   &   $-1000.00\pm-1000.000$  &  $-1000.00\pm-1000.000$  &  $21.28\pm0.422$  &  $-1000.00\pm-1000.000$   & $-1000.00\pm-1000.000$    &   $17.364\pm0.064$   &  $-100.00\pm-9.990$   &   $-9.99\pm-9.990$   &   $-9.99\pm-9.990$   \\
SDWISE & J0025-0145 &    6.36183 &   -1.75903 &  5.07   &   $-1000.00\pm-1000.000$  &  $-1000.00\pm-1000.000$  &  $-1000.00\pm-1000.000$  &  $17.74\pm0.004$   & $-1000.00\pm-1000.000$    &   $14.851\pm0.009$   &  $14.23\pm0.018$   &   $11.39\pm0.220$   &   $ 8.51\pm-9.900$   \\
PSO & J007.0273+04.9571 &    7.02733 &    4.95712 &  6.00   &   $-1000.00\pm-1000.000$  &  $20.33\pm0.056$  &  $20.23\pm0.074$  &  $20.29\pm0.108$   & $20.19\pm0.105$    &   $17.178\pm0.060$   &  $16.61\pm0.135$   &   $12.25\pm-9.900$   &   $ 8.32\pm-9.900$   \\
SDWISE & J0031+0710 &    7.85775 &    7.17692 &  5.33   &   $-1000.00\pm-1000.000$  &  $20.03\pm0.082$  &  $20.20\pm0.146$  &  $19.49\pm0.106$   & $19.61\pm0.123$    &   $16.658\pm0.039$   &  $15.68\pm0.063$   &   $12.19\pm-9.900$   &   $ 8.40\pm-9.900$   \\
  \hline \hline
    \end{tabular}
    \caption{The first ten of 463 very high-$z$ quasars with near and mid-infrared photometry.}
  \end{table}

\end{landscape}
%\normalsize 



%%%%%%%%%%%%%%%%%%%%%%%%%%%%%%%%%%%%%%%%%%%%%%%%%%%%%%%%%%%%%%%%%%
%%%%%%%%%%%%%%%%%%%%%%%%%%%%%%%%%%%%%%%%%%%%%%%%%%%%%%%%%%%%%%%%%%
%% 
%%     S E C T I O  N   3         S E C T I O  N   3           S E C T I O  N   3       S E C T I O  N   3
%%     S E C T I O  N   3         S E C T I O  N   3           S E C T I O  N   3       S E C T I O  N   3
%%     S E C T I O  N   3         S E C T I O  N   3           S E C T I O  N   3       S E C T I O  N   3
%%
%%%%%%%%%%%%%%%%%%%%%%%%%%%%%%%%%%%%%%%%%%%%%%%%%%%%%%%%%%%%%%%%%%
%%%%%%%%%%%%%%%%%%%%%%%%%%%%%%%%%%%%%%%%%%%%%%%%%%%%%%%%%%%%%%%%%%
\section{Results}
Having collated the sample of 463 VH$z$Qs, and obtained their optical,
near- and mid-infrared photometry we report here the various
photometric properites of the quasars.

First, we will concentrate on detection rate in the infrared, go on to
report on the color-redshift and color-color properties of our sample
and then report on how the current sample populates the
luminosity-redshift $Lz$-plane.

    %\subsection{Detection Rates in the optical}

    \subsection{Detection Rates in the NIR}
    Table~\ref{tab:nir_detection} gives the detection rates for the 
    VH$z$Qs in the NIR $YJHK/K_{s}$-bands. 
    The first thing to note is that the coverage of the NIR surveys 
    for example from the UKIDSS LAS and VISTA VHS, does
    not overlap the full area for where the VH$z$Qs are detected. 

    \begin{table}
          \centering

      \begin{tabular}{l r l}
        \hline  \hline
        Selection   & number detected (\%) \\
        \hline  
        Any band ($ZYJHK/K_{s}$   &  449  (97.0) \\
        $Z$-band    &  75  (16.2) \\
        $Y$-band    &  273  (59.0) \\
        $J$-band    &  447  (96.5) \\
        $H$-band    &  269  (58.1) \\
        $K$ or $Ks$-band    &  322  (69.5) \\
        \hline  \hline
      \end{tabular}
      \caption{Detection rate of VH$z$Qs in the near-infrared.}
      \label{tab:nir_detection}
    \end{table}

     There are 14 objects that have no NIR detections. 
     3 of these (PSOJ053.9605-15.7956, PSO J056.7168-16.4769 and DELSJ0411-0907) 
     have been observed (by VHS) but are out of our queried time range. 
     6 objects have not been observed (or at least the data is not in the WSA/VSA 
     archives yet) and 5 objects at $\geq +60$ deg$^{2}$ are too far north for UKIRT 
     and cannot be observed. 


        \subsubsection{Comparing WFCAM and VISTA}
        There are 130 overlapping QSOs between WFCAM and VISTA.  Using the
        {\tt VegaToAB} value\footnote{What is this exactly??} to put these
        objects on the same AB system, and for each object compared the two
        measurements. First, the calculated weighted average (calibrated flux)
        in each filter of both and calculated the ratio and difference between
        each measurement and the average.  Then for each filter we calculated
        the weighted average of the differences (in mag) for each instrument
        to see if there were significant offsets. The results are given in
        Table~\ref{tab:WFCAM_vs_VISTA}.  The only filter with a significant
        offset is the $Y$-band. All of the VISTA averages are negative and all
        of the WFCAM ones are positive.  The $Ks$ versus $K$ band may be
        slightly dodgy, given the different shapes of the filters.
        \begin{table}
          \centering
          \begin{tabular}{l r r}
            \hline  \hline
            abs(VIRCAM & \multirow{2}{*}{millimags} &  no. of  \\
            -  WFCAM)      &                                        &  objects \\
            \hline
            $Z$                 &  23.2 	& 3 \\
            $Y$                 &  57.3 	& 53 \\
            $J$                  &    2.1 	& 106 \\
            $H$                 &  45.8     &  96 \\
            $K_{\rm s}$/$K$ &  25.2     & 110 \\
            \hline  \hline
          \end{tabular}
          \caption{Comparing the magnitudes in different WFCAM/UKIRT and VIRCAM/VISTA near-infrared bands.}
          \label{tab:WFCAM_vs_VISTA}
        \end{table}

%% On May 13, 2019, at 5:25 PM, Nicholas Cross <njc@roe.ac.uk> wrote:

%% I have also checked for QSOs with large differences in magnitude. At first I found loads of objects which appeared to have either the WFCAM or VISTA measurement more than 1 mag different from the mean. In fact initially there were 100 QSO/filter combinations. However, in many cases the average flux was less 0, or very close to zero, so I selected only objects where the average was >0. and >5 average flux error. This removed 80-90\%. I then found that many of the others had one of WFCAM or VISTA where the error bars were very large, so I removed ones where deltaMag<2.*deltaMagErr in either. This left me with 1 object or 2 if I reduced the change to 0.2 mag different from the mean:

The two with large differences are:

SDSSJ0349+0034, in K/Ks band: VSA = 18.36$\pm$0.10, WSA = 19.13$\pm$-0.24 and 

SDSSJ2220-0101, in J band: VSA = 19.38+/-0.04, WSA = 22.23+/-0.15


        We investigate the VH$z$Qs in the UltraVISTA field. 
        UltraVISTA is...
        %% 
        One detail to note is that depths quoted in the UltraVISTA Data Releases are for more realistic 
        1.8 arcsec diameter apertures (J. Dunlop, priv. comm.). 


        
    \subsection{Detection Rates in the MIR}
    Unlike the NIR coverage, the WISE
    satellite and mission performed an all-sky survey, so the location of
    evey VH$z$Q in our dataset is covered. However, the depth of the WISE
    ALLWISE survey depends heavily on sky location, with locations near
    the Ecliptic Poles having the highest number of exposures.
    
    Before reporting on the detection rates, we investigate this
    effect. Figure~\ref{fig:WISEmag_vs_coverage} shows the WISE magnitude
    versus signal-to-noise, colour coded by {\tt w$x$cov} the mean
    coverage depth, in each corresponding band. In the two shorter bands
    W1/2 we see the clear and expected trend for brighter objects to have
    larger SNR, and also for the higher signal to noise for objects with
    more exposures at a given magnitude. The behaviour for the W3/4 bands
    is different, with two populations clearly evident in W3 and although
    a bit more mixed, also in W4. With the suggested split at SNR$>2$, and
    no obvious R.A./Declination dependence seen, this behaviour is
    explained by the fact that there are non-detectiopns in W3/4 for
    objects (with high W1/2 SNR) that are reported in the ALLWISE
    catalogue.
    
    For the 278 VH$z$Q with coverage detections, the mean number of
    exposures for the W1/2 bands is 32.0 and 31.5, respectively, with a
    minimum number of exposures 17 and 12, and the maximum number or
    exposures being 114 (for both bands).  For the W3/4 filters, the
    corresponding mean, minimum and maximum exposure are 17.4 and 17.5,
    5.8 and 6.8 and 69 (for both bands). These values are direclty from
    the {\tt w$x$cov} enteries in the 
    \href {http://wise2.ipac.caltech.edu/docs/release/allwise/expsup/sec2_1a.html#w1cov}{WISE ALLWISE catalogue}.

    \begin{figure}
      %% trim=l b r t
      \includegraphics[width=8.6cm, clip,trim=6mm 6mm 0mm 6mm]
      {/cos_pc19a_npr/programs/quasars/highest_z/detections/WISEmag_vs_coverage_2x2_v1.pdf}
      \centering
      \vspace{-14pt}
      \caption[]{WISE W1/2/3/4 magnitude against signal-to-noise, 
        colour coded by w$x$cov the mean coverage depth, in each corresponding band.
      }
      \label{fig:WISEmag_vs_coverage}
    \end{figure}
    
    Table~\ref{tab:mir_detection} gives the detection rates for the
    VH$z$Qs in the MIR WISE W1-4 bands. 
    \begin{table}
      \begin{tabular}{l r l}
        \hline  \hline
        Selection   & number detected (\%) \\
        \hline  
        W1 SNR $> 2.0$                                             &  275  (64.9) \\
        W2 SNR $> 2.0$                                            &   255 (60.1) \\
        W1 $\land$ W2 SNR $> 2.0$                         &  \\
        W3 SNR $> 2.0$                                            &  99    (23.3) \\
        W4 SNR $> 2.0$                                            &  29    (6.8) \\
        Any W1/2/3/4 SNR $>2.0$                           & \\
        W1/2 SNR $< 2.0$ $\land$ W3 SNR $>2.0$ & \\
        \hline  \hline
      \end{tabular}
      \caption{}
      \label{tab:mir_detection}
    \end{table}


    \begin{table}
      \begin{tabular}{l r l}
        \hline  \hline
         \multirow{2}{*}{Selection}   & number detected \\ 
                                                   & (\% of full specrta) \\
        \hline  
        From ``Source'', ``Rejects'',                    & 245, 40  (67.2) \\
        W1 SNR $> 2.0$                                       &  279  (65.8) \\
        W2 SNR $> 2.0$                                       &  258 (60.8) \\
        W1 $\land$ W2 SNR $> 2.0$                    & 253   (59.7)  \\
        W3 SNR $> 2.0$                                     &  97    (22.9) \\
        W4 SNR $> 2.0$                                     &  33    (7.8) \\
%        Any W1/2/3/4 SNR $>2.0$                            & \\
        W1/2 SNR $< 2.0$ $\land$ W3 SNR $>2.0$ &  3 (0.7)\\
        \hline  \hline
      \end{tabular}
      \caption{Data from the AllWISE Source Catalog and AllWISE Reject Table, from the 
\href{https://irsa.ipac.caltech.edu/cgi-bin/Gator/nph-scan?submit=Select&projshort=WISE} {{\tt NASA/IPAC  Infrared Science Archive}}}
      \label{tab:mir_detection}
    \end{table}
    
\begin{table}
\centering
\begin{tabular}{l r  r}
\hline \hline
$z \geq$  & Age / Myr & No. of objects \\
\hline 
7.50         &    700         &   1   \\
7.00         &    767         &   4   \\
6.78         &    800         &  14   \\
6.50         &    845         &  40   \\
6.19         &   900          &  86   \\
6.00         &   937          &  170   \\
5.70         & 1000          &  267   \\
5.00         & 1180          &  463   \\
\hline \hline
\end{tabular}
\caption{The number of objects at or above a given redshift. 
The age of the Universe in megayears is also given. }
      \label{tab:ages}
\end{table}



    \begin{figure}
      %% trim=l b r t
      \includegraphics[width=8.6cm, clip,trim=2mm 0mm 2mm 0mm]
      {/cos_pc19a_npr/programs/quasars/highest_z/detections/WISEsnrW1W2W3W4_2by3_v1.pdf}
      \centering
      \vspace{-14pt}
      \caption[]{WISE signal-to-noise measures for the four bands, as well
        as for (W1-W2) colour.  The points are colour coded by redshift.}
      \label{fig:WISEmag_vs_coverage}
    \end{figure}

    \citet{Blain2013} 

    Recently, \citet{Assef2018} released two large catalogues of AGN 
    candidates identified across 30,000 deg$^2$ of extragalactic sky 
    from the WISE AllWISE Data Release. The ``R90'' catalogue, is 
    contains 4.5M AGN candidates at 90\% reliability (and $\approx$150 
    AGN candidates per deg$^2$) while the ``C75'' catalog 
    consists of 20.9M AGN candidates at 75\% completeness (and 
    ($\approx$700 AGN candidates per deg$^2$).  Crossmatching 
    out catalogue of 463 VH$z$Qs with these catalogues, produces 
    42 matches with	the R90 sample and 98 matches with the C75 sample. 
    Both catalogues unsurprisingly match to the ultraluminous quasar 
    SDSS J0100+2802 \citep{Wu2015} while the C75, but not the R90 catalogue 
    mathes to ULAS J1120+0641 \citep{Mortlock2011}. Neither catalogue 
    matches J1342+0928 \citep{Banados2018}. 

    %\subsubsection{
    Very High-$z$ Quasars Detected in WISE W3 and W4.


    \begin{figure}
      \centering
      \includegraphics[width=8.5cm]
      {/cos_pc19a_npr/programs/quasars/highest_z/light_curves/MIR_LCs/NEOWISER_LC_histogramlog_20180827.png}
      \vspace{-16pt}
      \caption[]
      {Histogram showing the number of NEOWISE-R epochs and detections there are for each 
        VH$z$Q.} 
      \label{fig:MIR_LC_epochs}
    \end{figure}
    
\subsection{Variability}
VH$z$Qs, if accreting at, or above the Eddington Limit, might well have have large values of changing mass accretion rate, $\ddot{m_{\rm accr}}$. A consequence of this would be that these quasar exhibit signs of variability, most likely showing up in their UV/optical rest-frame spectra. We look for evidence of this variability signature in the NIR and MIR light-curves of the VH$z$Qs. As a guide, \civ enters the $Y$-band at redshift $z$=5.32 and exits at $z$=5.99, and enters the $J$-band at redshift $z=6.55$ and exits at $z$=7.57. \mgii enters the $H$-band at redshift $z=4.33$ and exits at $z=5.37$ and enters the $K$-band at redshift $z=6.25$ and exits at $7.50$.

Using the extended datasets described in Section~\ref{sec:NIR_data} and~\ref{sec:NIR_SQL}, we 

\begin{figure}
  \includegraphics[width=8.5cm]
  {/cos_pc19a_npr/programs/quasars/highest_z/light_curves/MIR_LCs/three_MIR_LC_egs_20180827.pdf}
  \centering
  \caption[]
  {Here we show the MIR NEOWISE-R for J0100+2802 \citep{Wu2015}, J0224-4711 and  J1626+2751. 
    Red points are the W1 band; cyan points the W2 band.} 
  \label{fig::MIR_LC_3egs}
\end{figure}

Figure~\ref{fig:MIR_LC_epochs} gives the number of NEOWISE-R epochs and detections there are for each VH$z$Q, while 
Figure~\ref{fig:MIR_LC_3egs} presents three examples of the MIR lightcurves and
associated colour changes. Here we show J0100+2802 \citep{Wu2015}, J0224-4711 and  J1626+2751. 
{\bf NJC: What about NIR light-curves / combined light-curves}

At least 8 original observations across at least 30 days in either WSA or VSA per band, calibrated flux greater than zero over the whole time, and the error on the average flux is 8 times lower than this. 

Thenn a requirement of SNR$>3$ per `average' epoch, where `average epoch' is then dependenet o nthe averaging timescales (whcih as we see can vary). 

Then calculate the clipped median and standard deviation, 
\begin{equation}
1.48 \times \sigma_{median} / \bar{\epsilon}
\end{equation}
where $\bar{\epsilon}$ is the mean of the error in each point in the light curve, divided by the total number of points. 
%%
With this definition, although some of the objects have very well sampled, 
`average' light curves where you are averaging over e.g. a month, or two weeks. 

Longer time-series, better SNR and moreover, $(1+z)$. 
Just selecting things that would
%%
Doing this independtly in each band...
%Variation

%% 
%% Object numbers: 54 VSA   MMTJ0215-0529,
%% objID: 162 is both VSA and WSA and  SDSSJ0959+0227
%%
objiD 164 is SDSSJ1000+0234, faint; very faint. 
SHELLQsJ0220-0432: (qso59) show this too. 
CFHQSJ0216-0455


\begin{figure}[H]
  \begin{subfigure}{}
    \centering
    \includegraphics[width=8.5cm]{../light_curves/MMTJ0215-0529LC_20190214.png}
    \caption{MMTJ0215-0529}
    \label{fig:MMTJ0215-0529}
  \end{subfigure}%
  \begin{subfigure}{}
    \centering
    \includegraphics[width=8.5cm]{../light_curves/CFHQSJ0216-0455LC_20190214.png}
    \caption{CFHQSJ0216-0455}
    \label{fig:CFHQSJ0216-0455}
  \end{subfigure}
  \begin{subfigure}{}\quad
    \centering
    \includegraphics[width=8.5cm]{../light_curves/SHELLQsJ0220-0432LC_20190214.png}
    \caption{SHELLQsJ0220-0432}
    \label{fig:SHELLQsJ0220-0432}
  \end{subfigure}
  \medskip
\end{figure}


%%%%%%%%%%%%%%%%%%%%%%%%%%%%%%%%%%%%%%%%%%%%%%%%%%%%%%%%%%%%%%%%%%
%%%%%%%%%%%%%%%%%%%%%%%%%%%%%%%%%%%%%%%%%%%%%%%%%%%%%%%%%%%%%%%%%%
%% 
%%     S E C T I O N    4         S E C T I O N    4           S E C T I O N    4       S E C T I O N   4
%%     S E C T I O N    4         S E C T I O N    4           S E C T I O N    4       S E C T I O N   4
%%     S E C T I O N    4         S E C T I O N    4           S E C T I O N    4       S E C T I O N   4
%%
%%%%%%%%%%%%%%%%%%%%%%%%%%%%%%%%%%%%%%%%%%%%%%%%%%%%%%%%%%%%%%%%%%
%%%%%%%%%%%%%%%%%%%%%%%%%%%%%%%%%%%%%%%%%%%%%%%%%%%%%%%%%%%%%%%%%%
\section{Surface Density of VH$z$Qs}
One interesting question to ask is given the compilation of VH$z$Qs assembled here, 
are there bright $z>5.00$ quasars that are in {\it current} photometric datasets, and if 
so, how many are still to be discovered and confirmed spectroscopcially?

VHS, Area 1: 150$<$RA/deg$<200$ annd $-40<$Decl./deg$<-20$:
$Y$=21.1 mag, $J$=20.6 mag, $H$=20.3 mag, $K_{s}$=19.8 mag (all AB). 
Area 2: 10$<$RA/deg$<$60, $-60<$Dec$<-40$:
$J$=21.2 mag, $H$=20.6 mag, $K_{s}$=20.2 mag. 

VIKING is more uniform, so I calculated the averages over the whole area:
$Z$=22.7 mag, $Y$=21.9 mag, $J$=21.4 mag, $H$=21.2 mag, $Ks$=21.1 mag.


\begin{figure}[h]
  \centering
  \includegraphics[width=8.5cm]{../data/WSA_VSA/VHS_H_abMagLim.png}
  \caption{The calculated average {\tt AB MAGLIM} in the VHS.
    As one can see, VHS is not completely uniform. The are 3 areas with different filter sets, 
    and the depths changed in 2 of these areas, particularly in $J$ and $K_{s}$. 
    Y and H are almost uniform, but the coverage is much less.}
  \label{fig:1}
\end{figure}

Following \citet{Fan2001b} and \citet{McGreer2013}, we use use an
exponential decline to describe the space density of VH$z$Qs at high
redshifts, e.g.
\begin{equation}
\rho(z, {\rm M}_{1450} \propto 10^{k (z)}
\end{equation}
where $z$ is the sample redshift. 
%Fan, X., Strauss, M. A., Schneider, D. P., et al. 2001b, AJ, 121, 54

UHS J-band depth 19.6 \citep[Vega; ][]{Dye2018}

VISTA HEMISPHERE SURVEY DATA RELEASE 1
Release date (will be set by ESO)
PROPOSAL ESO No.: 179.A-2010
PRINCIPAL INVESTIGATOR: Richard McMahon
Authors: R. McMahon, M. Banerji, N. Lodieu for the VHS Collaboration

The aim of the Vista Hemisphere Survey (VHS) is to carry out a near
Infra-Red survey, which when combined with other VISTA Public Surveys
will result in coverage of the whole southern celestial hemisphere
($\sim$20,000 deg$^2$) to a depth 30 times fainter than 2MASS/DENIS in
at least two wavebands (J and K$_s$), with an exposure time of 60 seconds
per waveband to produce median 5σ point source (Vega) limits of $J =
20.2$ and $K_{S} = 18.1$. In the South Galactic Cap, $\sim$5000 deg$^{2}$ will be
imaged deeper with an exposure time of 120 seconds and also including
the H band producing median 5$\sigma$ point limits of: J = 20.6; H = 19.8; K$_{s}$ = 18.5. 
In this 5000 deg$^{2}$ region of sky deep multi-band optical (grizY)
imaging data will be provided by the Dark Energy Survey (DES). The
remainder of the high galactic latitude ($|b|>$30$^{\circ}$) sky will be imaged
in YJHK for 60sec per band to be combined with ugriz waveband
observations from the VST ATLAS survey. \\ 

VHS                   YJHK            99.99, 21.2,  21.2,  20.6,  20.0                (9) \\

{\tt ABmagLimits} for each survey in the database. \\



%%%%%%%%%%%%%%%%%%%%%%%%%%%%%%%%%%%%%%%%%%%%%%%%%%%%%%%%%%%%%%%%%%
%%%%%%%%%%%%%%%%%%%%%%%%%%%%%%%%%%%%%%%%%%%%%%%%%%%%%%%%%%%%%%%%%%
%%
%%  S E C T I O  N   7         S E C T I O  N   7           S E C T I O  N   7       S E C T I O  N   7
%%  S E C T I O  N   7         S E C T I O  N   7           S E C T I O  N   7       S E C T I O  N   7
%%  S E C T I O  N   7         S E C T I O  N   7           S E C T I O  N   7       S E C T I O  N   7
%%
%%%%%%%%%%%%%%%%%%%%%%%%%%%%%%%%%%%%%%%%%%%%%%%%%%%%%%%%%%%%%%%%%%
%%%%%%%%%%%%%%%%%%%%%%%%%%%%%%%%%%%%%%%%%%%%%%%%%%%%%%%%%%%%%%%%%%
\section{Discussion and Conclusions}
\label{sec:conclusions}
In this study, we have, for the first time, ompiled the list of all
$z>5$ spectroscopically confirmed quasars. We have assemble the NIR
($y/Y, J, H, K/K_{s}$) and MIR (WISE W1/2/3/4) photometry for these
objects, given their detection rates and SEDs. We find that: 

%%
We can gain a good appreciation for what these missions will discover
by collating the datasets we currently have. 

\begin{itemize}
    \item Lorem ipsum dolor sit amet, consectetur adipiscing
      elit. Aliquam porta sodales est, vel cursus risus porta non. Vivamus
      vel pretium velit. Sed fringilla suscipit felis, nec iaculis lacus
      convallis ac. 
    \item Fusce pellentesque condimentum dolor, quis vehicula
      tortor hendrerit sed. Class aptent taciti sociosqu ad litora torquent
      per conubia nostra, per inceptos himenaeos. Etiam interdum tristique
      diam eu blandit. Donec in lacinia libero.
    \item Sed elit massa, eleifend non sodales a, commodo ut felis. Sed id
      pretium felis. Vestibulum et turpis vitae quam aliquam convallis. Sed
      id ligula eu nulla ultrices tempus. Phasellus mattis erat quis metus
      dignissim malesuada. Nulla tincidunt quam volutpat nibh facilisis
      euismod. Cras vel auctor neque. Nam quis diam risus.
\end{itemize}
Nunc lacus nibh, convallis ac lobortis ut, tempus ac lectus. Maecenas
eu elit massa. Nulla vel lacus lorem. Proin et lobortis
tortor. Phasellus ultrices nisl non enim porttitor dictum. Curabitur
nec nunc ac nibh ornare elementum. Nunc ultrices hendrerit
ultricies. Aliquam dapibus semper est et gravida. Etiam cursus, massa
eget tempor elementum, lectus urna feugiat nisi, eget sagittis.

\subsection*{Author Contributions}   
N.P.R. initiated the project, compiled the list of $z>5.00$ quasars, wrote most of the analysis code, developed the the plotting scripts, and developed and wrote the initial and subsequent drafts of the manuscript.
%%
N.J.G.C. supplied the critical near-infrared expertise and database for which the bulk of the project relies. N.J.G.C. also contributed directly to the writing of the manuscript.
%%



\subsection*{Availability of Data and computer analysis codes} 
All materials, databases, data tables and code are fully available at: 
\href{https://github.com/d80b2t/VHzQ}{\tt https://github.com/d80b2t/VHzQ}


\section*{Acknowledgements}
NPR acknowledges support from the STFC and the Ernest Rutherford Fellowship scheme. 

We thank Mike Read at the ROE WFAU for help with the WFCAM Science Archiv (WSA), and 
also the VISTA Science Archive (VSA). We thank Bernie Shiao at STScI for help with the Pan-STARRS1 DR1 CasJobs interface. 

This paper heavily used \href{http://www.star.bris.ac.uk/~mbt/topcat/}{TOPCAT} (v4.4)
\citep[][]{Taylor2005, Taylor2011}.
%%
This research made use of \href{http://www.astropy.org}{\tt Astropy}, 
a community-developed core Python package for Astronomy 
\citep{AstropyCollaboration2013, AstropyCollaboration2018}. 

The Pan-STARRS1 Surveys (PS1) and the PS1 public science archive have
been made possible through contributions by the Institute for
Astronomy, the University of Hawaii, the Pan-STARRS Project Office,
the Max-Planck Society and its participating institutes, the Max
Planck Institute for Astronomy, Heidelberg and the Max Planck
Institute for Extraterrestrial Physics, Garching, The Johns Hopkins
University, Durham University, the University of Edinburgh, the
Queen's University Belfast, the Harvard-Smithsonian Center for
Astrophysics, the Las Cumbres Observatory Global Telescope Network
Incorporated, the National Central University of Taiwan, the Space
Telescope Science Institute, the National Aeronautics and Space
Administration under Grant No. NNX08AR22G issued through the Planetary
Science Division of the NASA Science Mission Directorate, the National
Science Foundation Grant No. AST-1238877, the University of Maryland,
Eotvos Lorand University (ELTE), the Los Alamos National Laboratory,
and the Gordon and Betty Moore Foundation.

This project used data obtained with the Dark Energy Camera (DECam)
and the NOAO Data Lab, The Data Lab is operated by the National
Optical Astronomy Observatory, the national center for ground-based
nighttime astronomy in the United States operated by the Association
of Universities for Research in Astronomy (AURA) under cooperative
agreement with the National Science Foundation.

This publication makes use of data products from the Wide-field
Infrared Survey Explorer, which is a joint project of the University
of California, Los Angeles, and the Jet Propulsion
Laboratory/California Institute of Technology, and NEOWISE, which is a
project of the Jet Propulsion Laboratory/California Institute of
Technology. WISE and NEOWISE are funded by the National Aeronautics
and Space Administration.

CasJobs was originally developed by the Johns Hopkins University/
Sloan Digital Sky Survey (JHU/SDSS) team. With their permission, MAST
used version 3.5.16 to construct CasJobs-based tools for GALEX,
Kepler, the Hubble Source Catalog, and PanSTARRS.

This research has made use of the SVO Filter Profile Service
(http://svo2.cab.inta-csic.es/theory/fps/) supported from the Spanish
MINECO through grant AyA2014-55216 
%%
The SVO Filter Profile Service\footnote{Rodrigo, C., Solano, E., Bayo, A. http://ivoa.net/documents/Notes/SVOFPS/index.html}
describes the Spanish VO Filter Profile Service. 
The Filter Profile Service Access Protocol. Rodrigo, C., Solano, E. http://ivoa.net/documents/Notes/SVOFPSDAL/index.html

%\newpage


\appendix


\section{Near-Infrared WFCAM Science Archive SQL queries}\label{sec:SQL}
Here we give the receipe and SQL that returned the near-infrared photometry 
for the VH$z$Qs from the  WFCAM Science Archive. 

The data are on the WFCAM Science Archive: \href{wsa.roe.ac.uk}{\tt wsa.roe.ac.uk}. 
Access the User Login form \href{WFCAM Science Archive}{\tt wsa.roe.ac.uk/login.html} 
with these credentials::
\begin{itemize}
    \item Username: {\tt WSERV1000} 
    \item password: {\tt highzqso} 
    \item community: {\tt nonsurvey}
\end{itemize}
Then going to the
\href{http://wsa.roe.ac.uk:8080/wsa/SQL_form.jsp}{{\tt Free Form SQL
Query}} page the Database release {\tt WSERV1000v20180716} can be
accessed which contains all the data we use here.

We {\it nota bene} a few things. First, the quantity {\tt aperJky3}
and {\tt aperJky3Err} are found in the {\tt wserv1000MapRemeasAver}
and {\tt wserv1000MapRemeasurement}, so care has to be taken to return
unique column names (otherwise e.g.
\href{http://docs.astropy.org/en/stable/io/fits/}{astropy.io.fits}
will crash).  As such, we alias {\tt aver.aperJky3} to {\tt
aperJky3Aver} and likewise for the error quantity. Aliases will be
necessary in some cases anyway, because some queries can be done
sensibly on multiple instances of the same table. Other times, one may
join tables on quantities such as {\tt catalogueID} or {\tt
apertureID}, where you are meaning the same thing, but aliases would
again be sensible.

Second, the {\tt RA} and {\tt DEC} values returned by the WSA are in radians, if
used directly. To return values in degrees, use a selection with an alias, e.g. 
{\tt RA as RADeg} and {\tt DEC as DECDeg}.

\onecolumn
\input{SQL_examples_WSA}
\twocolumn


\section{Near-Infrared VISTA Science Archive SQL queries}\label{sec:SQL}
In a very similar manner to the WSA, we give here the details on how to access
the VISTA Science Archive (VSA)

At the \href{http://horus.roe.ac.uk/vsa/login.html}{VSA Login}, enter 
with these credentials::
\begin{itemize}
    \item Username: {\tt VSERV1000} 
    \item password: {\tt highzqso} 
    \item community: {\tt proprietary}
\end{itemize}
Then head to the \href{http://horus.roe.ac.uk:8080/vdfs/VSQL_form.jsp}{Freeform SQL Query} page where the database release to use is {\tt VSERV1000v20180716}. 

%\onecolumn
%\input{SQL_examples_VSA}
%\twocolumn





%%%%%%%%%%%%%%%%%%%%%%%%%%%%%%%%%%%%%%%%%%%%%%%%%%%%%%%%%%%%%%%%%%%%
%%%%%%%%%%%%%%%%%%%%%%%%%%%%%%%%%%%%%%%%%%%%%%%%%%%%%%%%%%%%%%%%%%%%

%\bibliographystyle{apj}
\bibliographystyle{mn2e}
\bibliography{tester_mnras}

\end{document}
